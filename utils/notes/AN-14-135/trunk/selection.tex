
%\fixme{This section needs to be rewritten to point to the relevant AN instead of copying and pasting them}

A complete reconstruction of the individual particles emerging from each collision event is obtained
via a particle-flow (PF) technique~\cite{CMS-PAS-PFT-09-001}, which uses the information from all CMS subdetectors. 
%to identify and reconstruct individual particles in the collision event. 
The identification of the particles produced in the event is the very same as the one described in~\cite{ZZXSPaper}, 
including the final state radiation treatment.
Also the logic to build up the ZZ candidate is the same as the Higgs boson analysis~\cite{HiggsLegacyPaper}.

The CMS standard selection of runs and luminosity sections is applied, which requires high quality data with a 
good functioning of the different sub-detectors. Events are selected online from the presence of a pair of electrons
or muons, or a triplet of electrons. Triggers requiring an electron and a muon are also used. The
minimal transverse momenta ($p_T$) of the first and second lepton are 17 and 8 GeV, respectively,
for the double lepton triggers, while they are 15, 8 and 5 GeV for the triple electron trigger.
More details can be found in~\cite{MuonTrigger}. The trigger efficiency for $ZZ$ events within the acceptance of this
analysis is greater than 98\%. % (to check, \fixme{}).\\

Events with identified and isolated primary electrons or muons are first selected by the offline
selection. We require a $Z$ candidate formed with a pair of leptons of the same flavor and
opposite charge. The FSR photons are kept if $|m_{\ell\ell\gamma} - m_Z | < |m_{\ell\ell} - m_Z | $ and $m_{\ell\ell\gamma} < 100$ GeV and
in the following the presence of the photons in the $\ell\ell$ kinematics is implicit.\\ 
The lepton pair with the invariant mass closest to the nominal $Z$ mass is denoted $Z_1$. A second $\ell^+\ell^-$ pair is required and denoted $Z_2$. If more than one $Z_2$ candidate satisfies all criteria, the pair of leptons with the highest scalar sum of $p_T$ is chosen. Both $Z$ candidates have to satisfy $60 < m_Z < 120$ GeV.
Among the four selected leptons forming the $Z_1$ and the $Z_2$ candidates at least one should have $p_T \ge 20$ GeV and another
one have $p_T \ge 10$ GeV  and any opposite-charge pair of leptons chosen among the four selected
leptons satisfy $m_{\ell\ell} \ge 4$ GeV. 
%Finally the phase space is reduced to the signal region by requiring $m_{4\ell} \ge 100$ GeV.

%\fixme{say here where the jets come from}
Jets are reconstructed using the anti-$k_{T}$ clustering 
algorithm~\cite{antiKT}, with a size parameter of $R = 0.5$, by summing the four-momenta of individual
PF particles according to the \texttt{FASTJET} package of reference~\cite{FASTJET}. Once reconstructed, jets overlapping
 (within $\Delta R = 0.5$) with any of the two leptons coming form the decay of the Z boson are removed from the jet collection. Jets satisfy the selection criteria recommended by Jet-MET group~\cite{JetID}. Jet energy corrections are applied as a function of the jet $p_T$ and $\eta$~\cite{JECandJER}. Moreover, they are required to have $p_T > 30$ GeV, 
 to reduce the pileup contamination as well as large uncertainty on the energy measurement, and $|\eta| < $ 4.7,
to ensure a good quality of the tracking information.
Since the jet energy resolution in data is worse than in simulation, the $p_T$ values of simulated
jets need to be smeared to describe data. Following the prescription of the CMS Jet-MET group~\cite{JETRES},
the corrected $p_T$ of the reconstructed jets are randomly smeared using a Gaussian distribution with a width 
of $\sqrt{c^2 -1}\cdot \sigma_{MC}$, 
where $c$ is the scaling factor (see Table~\ref{tab:c_JER}). This method only allows one to
worsen the resolution ($c > 1$). To determine the jet resolution in simulation, $\sigma_{MC}$, recommended tools have been used~\cite{JETRES}. 
\begin{table*}[htbH]
\begin{center}
\topcaption{Scaling factors applied on MC jets to described the data resolution\label{tab:c_JER}}
\begin{tabular}{cccccccc}
\hline $|\eta|$-region & $0.0-0.5$ & $0.5-1.1$ & $1.1-1.7$ & $1.7-2.3$ & $2.3-2.8$ & $2.8-3.2$ & $3.2-5.0$\\
\hline $c$ & 1.079 & 1.099 & 1.121 & 1.208 & 1.254 & 1.395 & 1.056\\
\hline
\end{tabular}
\end{center}
\end{table*}

Ancillary quantities, like data/MC efficiency scale factor and jet-to-lepton fake rate, are the same as the one measured in~\cite{ZZXSPaper}.
