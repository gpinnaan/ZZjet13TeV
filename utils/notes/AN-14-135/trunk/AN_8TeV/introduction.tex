The study of vector boson pair production in proton-proton collisions is crucial both in order to check the gauge structure
of the Standard Model (SM) and to search for new physics. In addition to this, the production of vector boson pairs, and their 
decay particles, enter indeed as irreducible backgrounds in many Higgs boson analyses and new physics searches and it is thus extremely 
important to measure these processes with high precision. 
Both ATLAS and CMS Collaborations at the LHC measured the cross-section for the $Z$ boson pair production decaying in charged-lepton final states. The CMS Collaboration performed the measurement in the 
$ZZ \to \ell\ell\ell'\ell'$ decay channels, where $\ell = e,\mu$ and $\ell' = e,\mu,\tau$ with the data corresponding
to an integrated luminosity of 5.1 (19.6) fb$^{-1}$ at $\sqrt{s}= 7(8)$ TeV~\cite{ZZXS7TeVPaper, ZZXSPaper}. The measured total cross section is 
$\sigma(pp \to ZZ) =  6.24^{+0.86}_{-0.80}\ (stat.)^{+0.41}_{-0.32}(syst.) \pm 0.14 (lumi.)$ pb at $\sqrt{s}= 7$ TeV and $7.7 \pm
0.5 (stat.) ^{+0.5}_{-0.4}(syst.) \pm 0.4 (theo.) \pm 0.2 (lumi.)$ at $\sqrt{s}= 8$ TeV, for both Z bosons in the mass
range $60 < m_{Z} < 120$ GeV. The ATLAS Collaboration measured a total cross section of $6.7 \pm 0.7 (stat.) ^{+0.4}_{-0.3}(syst.)
\pm 0.3 (lumi.)$ pb using $ZZ\to \ell\ell\ell'\ell'$ and  $ZZ\to \ell\ell\nu\nu$ final states with a data sample corresponding
to an integrated luminosity of 4.6 fb$^{-1}$ at $\sqrt{s}= 7$ TeV and $66 < m_{Z} < 116$ GeV. The ATLAS
 result of the total ZZ production cross section  at $\sqrt{s}= 8$ TeV is
 $\sigma(pp \to ZZ) =  7.1^{+0.5}_{-0.4}\ (stat.)\pm 0.3 (syst.) 
\pm 0.2 (lumi.)$ pb, measured in the $ZZ\to \ell\ell\ell'\ell'$ final state, with $66 < m_{Z} < 116$ GeV and using
a data sample of 20.3 fb$^{-1}$. Measurements of the $ZZ$ cross sections performed at the Tevatron are
summarized in ~\cite{ZZTevatronCDF,ZZTevatronD0}. All measurements are in agreement with the corresponding SM predictions.


In this note we present the measurement of the differential cross section of two Z bosons produced
in association with jets, extracted from pp collision data at $\sqrt{s}$~=~8~TeV. The measurement is done
considering the processes with leptonic decays of the Z bosons in either electrons or muons, using datasets
containing an integrated luminosity of $19.7$~fb~$^{-1}$.

This analysis extends the measurement done in~\cite{ZZXSPaper}, using the latest reprocessed data, to event quantities that have jets as basic observable. 
The main goal is to measure the differential cross-section in the number of jets accompaining the two Z bosons, as well as the measurement of the 
differential cross-section as a function of the invariant mass of the two energy-leading jets ($m_{jj}$), the  difference in (pseudo-)rapidity ($\Delta \eta(j_1 , j_2)$) between them, their transverse momentum and pseudorapidity ($p_T^{j1,j2}$ and $\eta^{j1,j2}$).
Very useful information can be extracted from these differential cross-sections: the cross-section dependency to the jet multiplicity has never been measured so far, and can tell us how well
we understand the QCD correction to the ZZ production. The measurement in the $\Delta \eta(j_1 , j_2)$ variable will be instrumental for the vector boson scattering analysis in this channel, 
while the $m_{jj}$ will be also the basis for future multi-boson final state searches (namely where an additional electroweak boson decays hadronically). The latter is also a key variable for the investigation of anomalous Quartic Gauge couplings (aQGC). Furthermore, the dipendency to the transverse momentum and pseudorapidity of the leading and sub-leading jet is investigated in order to obtain further information.

The implant of the analysis is largely the same as~\cite{ZZXSPaper} and, for what concerns the reconstruction of the two Z bosons, it is explicitly
based on the algorithms and the techniques used for the $H\rightarrow ZZ \rightarrow 4\ell$ analysis~\cite{HiggsLegacyPaper} (as it was done for~\cite{ZZXSPaper}), with the obvious change in the mass window ($60 < m_{Z_{1,2}} < 120~\mathrm{GeV}$). %used for the lowest quality Z.


This note is organized as follow: section~\ref{sec:samples} summarizes the Monte Carlo and the dataset used for this analysis, section~\ref{sec:selection} describes the event selection, 
while section~\ref{sec:background} deals with the background estimation and the related uncertainties. Section~\ref{sec:systunc} lists the several sources of systematic uncertainty and, finally, in section~\ref{sec:results} we present the result of this analysis,
including the unfolding of the detector resolution for the differential cross-section that we measured.
 
