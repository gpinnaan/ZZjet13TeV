%(The same as AN2013-134: check values!! FIXME)\\
\label{sec:systunc}
The systematic uncertainties for trigger efficiency (FIXME\%) are evaluated from data. The uncertainty in the LHC integrated luminosity of the data sample is FIXME\%~\cite{CMS-PAS-LUM-13-001}. Theoretical uncertainties in the pp~$\to \Z\Z \to \ell\ell\ell'\ell'$ acceptance are evaluated using \texttt{MCFM}  by varying the renormalization and factorization scales, up and down, by a factor of two with respect to the default values $\mu_R = \mu_F = m_\Z$. The variations in the acceptance are FIXME\% (NLO $\cPq\cPaq \to \Z\Z$) and FIXME\% ($\cPg\cPg \to \Z\Z$), and can be neglected. Uncertainties related to the choice of the PDF and the strong coupling constant are evaluated following the \texttt{PDF4LHC} ~\cite{PDF4LHCRec, PDF4LHCRep} prescription and using \texttt{CT10},  \texttt{MSTW08}, and  \texttt{NNPDF}~\cite{NNPDF} PDF sets. They are found to be FIXME\%.\\
The systematic uncertainty propagated through the unfolding procedure discussed in Section~\ref{sec:results} arises from several sources. The reducible background uncertainties in Z + jets, WZ + jets and \ttbar yields reflect the statistical uncertainties of the control regions in the data and correspond to a FIXME\%  uncertainty on the cross section measurement, depending on the jet multiplicity. The irreducible background uncertainty is smaller and varies between FIXME\% and FIXME\% of the cross section estimate. A systematic uncertainty is also associated to the estimate of  $\cPq\cPaq \to \Z\Z$ and $\cPg\cPg \to \Z\Z$ cross sections used in the MC simulation and is found to be very small ($<$~FIXME\%). The JES uncertainty is estimated separately for data (FIXME\%) and MC (FIXME\%) and it is the largest contribution affecting the measurement, while the uncertainty due to JER is less than FIXME\%. Uncertainties arising from lepton identification, isolation, tracking and impact parameter range between FIXME\% and FIXME\% as a function of the jet multiplicity. Finally, systematic uncertainties due to the unfolding procedure are obtained using two complementary methods: by employing two different sets of MC samples (\texttt{MadGraph}~+~\texttt{MCFM}~+~\texttt{Phantom} or \texttt{Powheg}~+~\texttt{MCFM}~+~\texttt{Phantom}) to estimate the response matrix (FIXME\%) and by building up a new response matrix whose elements are weighted for the ratio between the unfolded data and the MC truth (FIXME\%). Statistical uncertainties from the MC samples make negligible contributions to the uncertainties on the response matrix.
\begin{table*}[htbH]
\begin{center}
\topcaption{Estimated systematic uncertainties on the signal yield used to compute the inclusive cross section measurement\label{tab:syst}}
\begin{tabular}{lclc}
\hline Systematic source & Uncertainty value\\
\hline Trigger & FIXME\%  \\
Lepton ID, ISO and Tracking & FIXME\%\\
Luminosity & FIXME\%\\
Reducible background & $\sim$FIXME\%\\
Irreducible background & $<$FIXME\%\\
Acceptance for PDF & FIXME\%\\
\hline
\end{tabular}
\end{center}
\end{table*}

% The uncertainty in the unfolding procedure discussed in Section 7 arises from differences between
% SHERPA and POWHEG for the unfolding factors (2–3%), from scale and PDF uncertainties
% (4–5%), and from experimental uncertainties (4–5%).

