% Customizable fields and text areas start with % >> below.
% Lines starting with the comment character (%) are normally removed before release outside the collaboration, but not those comments ending lines

% svn info. These are modified by svn at checkout time.
% The last version of these macros found before the maketitle will be the one on the front page,
%so only the main file is tracked.
% Do not edit by hand!
\RCS$Revision: 343919 $
\RCS$HeadURL: svn+ssh://svn.cern.ch/reps/tdr2/notes/AN-14-135/trunk/AN-XX-YYY.tex $
\RCS$Id: AN-XX-YYY.tex 343919 2016-05-23 16:05:58Z lfinco $
%%%%%%%%%%%%% local definitions %%%%%%%%%%%%%%%%%%%%%
%\usepackage{rotating}
%\usepackage{float}
%\usepackage[utf8]{inputenc}
% This allows for switching between one column and two column (cms@external) layouts
% The widths should  be modified for your particular figures. You'll need additional copies if you have more than one standard figure size.
\newlength\cmsFigWidth
\ifthenelse{\boolean{cms@external}}{\setlength\cmsFigWidth{0.85\columnwidth}}{\setlength\cmsFigWidth{0.4\textwidth}}
\ifthenelse{\boolean{cms@external}}{\providecommand{\cmsLeft}{top}}{\providecommand{\cmsLeft}{left}}
\ifthenelse{\boolean{cms@external}}{\providecommand{\cmsRight}{bottom}}{\providecommand{\cmsRight}{right}}
%%%%%%%%%%%%%%%  Title page %%%%%%%%%%%%%%%%%%%%%%%%
\cmsNoteHeader{AN-XX-YYY} % This is over-written in the CMS environment: useful as preprint no. for export versions
% >> Title: please make sure that the non-TeX equivalent is in PDFTitle below
\title{Measurement of the differential cross section for $pp\rightarrow ZZ \rightarrow 4 \ell$ produced in association with jets in pp collisions at $\sqrt{s}=13$~TeV}

% >> Authors
%Author is always "The CMS Collaboration" for PAS and papers, so author, etc, below will be ignored in those cases
%For multiple affiliations, create an address entry for the combination
%To mark authors as primary, use the \author* form

\address[unito]{Universi\`a degli Studi di Torino and INFN.}
\address[infnto]{INFN Torino.}
\address[hig]{The Higgs to four leptons team.}
\author[unito]{N. Amapane}
\author[unito]{R. Bellan} 
\author[unito]{R. Covarelli} 
\author[unito]{L. Finco}
\author[unito]{R. Gomez Ambrosio}
\author[infnto]{C. Mariotti}  
\author[unito]{G. Pinna}
\author[hig]{many other people}

% >> Date
% The date is in yyyy/mm/dd format. Today has been
% redefined to match, but if the date needs to be fixed, please write it in this fashion.
% For papers and PAS, \today is taken as the date the head file (this one) was last modified according to svn: see the RCS Id string above.
% For the final version it is best to "touch" the head file to make sure it has the latest date.
\date{\today}

% >> Abstract
% Abstract processing:
% 1. **DO NOT use \include or \input** to include the abstract: our abstract extractor will not search through other files than this one.
% 2. **DO NOT use %**                  to comment out sections of the abstract: the extractor will still grab those lines (and they won't be comments any longer!).
% 3. For PASs: **DO NOT use tex macros**         in the abstract: CDS MathJax processor used on the abstract doesn't understand them _and_ will only look within $$. The abstracts for papers are hand formatted so macros are okay.
\abstract{This note reports the measurement of the differential cross sections of two Z bosons produced in association with jets in pp collisions at a center-of-mass energy of $\sqrt{s}=13~\mathrm{TeV}$. This analysis is based on a data sample collected with the CMS experiment at the LHC, corresponding to an integrated luminosity of $12.9~\mathrm{fb}^{-1}$. Measurements are performed in the leptonic decay modes $\mathrm{ZZ}\to\ell\ell\ell'\ell'$, where $\ell,\ell' = e, \mu$. 
}

% >> PDF Metadata
% Do not comment out the following hypersetup lines (metadata). They will disappear in NODRAFT mode and are needed by CDS.
% Also: make sure that the values of the metadata items are sensible and are in plain text:
% (1) no TeX! -- for \sqrt{s} use sqrt(s) -- this will show with extra quote marks in the draft version but is okay).
% (2) no %.
% (3) No curly braces {}.
\hypersetup{%
pdfauthor={R. Bellan, L. Finco},%
pdftitle={Measurement of the differential cross section for ZZ rightarrow 4 ell produced in association with jets in pp collisions at sqrt s = 13 TeV},%
pdfsubject={CMS},%
pdfkeywords={CMS, physics, software, computing}}

\maketitle %maketitle comes after all the front information has been supplied
% >> Text
%%%%%%%%%%%%%%%%%%%%%%%%%%%%%%%%  Begin text %%%%%%%%%%%%%%%%%%%%%%%%%%%%%
%% **DO NOT REMOVE THE BIBLIOGRAPHY** which is located before the appendix.
%% You can take the text between here and the bibiliography as an example which you should replace with the actual text of your document.
%% If you include other TeX files, be sure to use "\input{filename}" rather than "\input filename".
%% The latter works for you, but our parser looks for the braces and will break when uploading the document.
%%%%%%%%%%%%%%%

\tableofcontents

%%%%%%%%%%%%%%%%%%%%%%%%%%%%%%%%  Begin text %%%%%%%%%%%%%%%%%%%%%%%%%%%%%


%% --------------------------------------------------------- %%
\section{Introduction}
\label{sec:introduction}
In this note we present the measurement of the differential cross section of two Z bosons produced in association with jets in pp collisions at $\sqrt{s}= 13~\mathrm{TeV}$, extending the analysis performed at $\sqrt{s}= 8~\mathrm{TeV}$ in~\cite{CMS-PAS-SMP-15-012}. The dependence of the cross section on jet multiplicity tests the current understanding of the QCD corrections to ZZ production. Moreover, the study of the invariant mass of the two $p_T$-leading jets ($m_{jj}$) establishes the basis for future multi-boson final state searches and for the investigation of phenomena involving interactions with four bosons in a single vertex. Finally, the measurement of the separation in pseudorapidity between the two $p_T$-leading jets ($\Delta\eta_{jj}$) is instrumental for the vector boson scattering analysis in this channel, the key process to probe the nature of the electroweak symmetry breaking mechanism, which becomes potentially accessible at 13 TeV.\\
 The measurement is performed considering the \Z bosons decaying leptonically to either electrons or muons. The analysis is based on a data set that corresponds to an integrated luminosity of $12.9~\mathrm{fb}^{-1}$, collected by the CMS experiment at $\sqrt{s} =$~ 13 TeV.\\

The implant of the analysis is largely the same as~\cite{CMS-PAS-SMP-15-012} and, for what concerns the reconstruction of the two Z bosons, it is explicitly
based on the algorithms and the techniques used for the $H\rightarrow ZZ \rightarrow 4\ell$ analysis~\cite{HiggsLegacyPaper} (as it was done for~\cite{CMS-PAS-SMP-15-012}), with the obvious change in the mass window ($60 < m_{Z_{1,2}} < 120~\mathrm{GeV}$). %used for the lowest quality Z.


This note is organized as follow: section~\ref{sec:samples} summarizes the Monte Carlo and the dataset used for this analysis, section~\ref{sec:selection} describes the event selection, 
while section~\ref{sec:background} deals with the background estimation and the related uncertainties. Section~\ref{sec:systunc} lists the several sources of systematic uncertainty and, finally, in section~\ref{sec:results} we present the result of this analysis,
including the unfolding of the detector resolution for the differential cross-section that we measured.
 

%% --------------------------------------------------------- %%


%% --------------------------------------------------------- %%
\section{The CMS Detector and MC Samples}
\label{sec:samples}
Several Monte Carlo (MC) event generators are used to simulate the signal and background contributions. 
The MC samples are used to optimize the event selection, evaluate the signal efficiency and
acceptance, extract the unfolding response matrices and estimate the irreducible background yields. 
The full set of MC samples used in this analysis is reported in table~\ref{tab:listofsamples}.


%\begin{table*}[htbH]\footnotesize
\begin{sidewaystable}\footnotesize
\begin{center}\scriptsize
\topcaption{List of signal and background samples used in the analysis.}
\label{tab:listofsamples}
\begin{tabular}{lll}
\hline  Process & Cross Sections [pb] & Sample\\
\hline  \multicolumn{3}{l}{Signal Samples}\\
$ZZ\to 4\ell$ (\texttt{MadGraph}) & 0.071 (no $\tau$) & \texttt{ZZJetsTo4L\_TuneZ2star\_8TeV-madgraph-tauola/Summer12\_DR53X-PU\_S10\_START53\_V7A-v1/AODSIM/PAT\_CMG\_V5\_15\_0/} \\ 
$q\bar{q}\to ZZ\to 4\mu$ (\texttt{Powheg}) & 0.07691 & \texttt{ZZTo4mu\_8TeV-powheg-pythia6/Summer12\_DR53X-PU\_S10\_START53\_V7A-v1}\\
$q\bar{q}\to ZZ\to 4e$  (\texttt{Powheg}) & 0.07691 & \texttt{ZZTo4e\_8TeV-powheg-pythia6/Summer12\_DR53X-PU\_S10\_START53\_V7A-v1 }\\ 
$q\bar{q} \to ZZ\to 2e2\mu$  (\texttt{Powheg})& 0.1767 & \texttt{ZZTo2e2mu\_8TeV-powheg-pythia6/Summer12\_DR53X-PU\_S10\_START53\_V7A-v1}\\ 
$gg\to ZZ\to 4\mu$  (\texttt{MCFM})&0.000592340 &\texttt{GluGluTo4mu\_SMHContinInterf\_M-125p6\_8TeV-MCFM67-pythia6/Summer12\_DR53X-PU\_S10\_START53\_V19-v1 }\\
$gg\to ZZ\to 4e$  (\texttt{MCFM}) &0.000592000 &\texttt{GluGluTo4e\_SMHContinInterf\_M-125p6\_8TeV-MCFM67-pythia6/Summer12\_DR53X-PU\_S10\_START53\_V19-v1 }\\ 
$gg\to ZZ\to 2e2\mu$  (\texttt{MCFM})& 0.00118409 & \texttt{GluGluTo2e2mu\_SMHContinInterf\_M-125p6\_8TeV-MCFM67-pythia6/Summer12\_DR53X-PU\_S10\_START53\_V19-v1}\\
$ZZ\to 4\mu +2jets$  (\texttt{Phantom})&2.49E-04 & \texttt{ZZTo4muJJ\_SMHContinInterf\_M-125p6\_8TeV-phantom-pythia6/Summer12\_DR53X-PU\_S10\_START53\_V19-v1}\\ 
$ZZ\to 4e +2jets$  (\texttt{Phantom})& 2.48E-04&\texttt{ZZTo4eJJ\_SMHContinInterf\_M-125p6\_8TeV-phantom-pythia6/Summer12\_DR53X-PU\_S10\_START53\_V19-v1}\\
$ZZ\to 2e2\mu +2jets$  (\texttt{Phantom})&5.43E-04 &\texttt{ZZTo2e2muJJ\_SMHContinInterf\_M-125p6\_8TeV-phantom-pythia6/Summer12\_DR53X-PU\_S10\_START53\_V19-v1 }\\ 
$WZZ + jets$ (\texttt{MadGraph}) & 0.01968& \texttt{WZZNoGstarJets\_8TeV-madgraph/Summer12\_DR53X-PU\_S10\_START53\_V7A-v1}\\
$ZZZ + jets$ (\texttt{MadGraph}) &0.005527 &\texttt{ZZZNoGstarJets\_8TeV-madgraph/Summer12\_DR53X-PU\_S10\_START53\_V7A-v1 }\\
$t\bar{t}+H$ (\texttt{Pythia}) & 0.0001081681 & \texttt{TTbarH\_HToZZTo4L\_M-126\_8TeV-pythia6/Summer12\_DR53X-PU\_S10\_START53\_V7C-v1}\\
\hline  \multicolumn{3}{l}{Irreducible Background Samples}\\
\hline$t\bar{t}+Z+Jets$ (\texttt{MadGraph})&0.2057 &\texttt{TTZJets\_8TeV-madgraph\_v2/Summer12\_DR53X-PU\_S10\_START53\_V7A-v1 }\\ 
$t\bar{t}+WW+Jets$ (\texttt{MadGraph})&0.00208989 & \texttt{TTWWJets\_8TeV-madgraph/Summer12\_DR53X-PU\_S10\_START53\_V7A-v1}\\ 
$WWZ + jets$ (\texttt{MadGraph}) &0.05795 & \texttt{WWZNoGstarJets\_8TeV-madgraph/Summer12\_DR53X-PU\_S10\_START53\_V7A-v1}\\
\hline  \multicolumn{3}{l}{Reducible Background Samples}\\
\hline $Z$ (\texttt{MadGraph})& 3503.71& \texttt{DYJetsToLL\_M-50\_TuneZ2Star\_8TeV-madgraph-tarball/Summer12\_DR53X-PU\_S10\_START53\_V7A-v1 }\\
$WZ$ (\texttt{MadGraph})&1.057 & \texttt{WZJetsTo3LNu\_TuneZ2\_8TeV-madgraph-tauola/Summer12\_DR53X-PU\_S10\_START53\_V7A-v1}\\
$WWW$ (\texttt{MadGraph})&0.08058 &\texttt{WWWJets\_8TeV-madgraph/Summer12\_DR53X-PU\_S10\_START53\_V7A-v1}\\
$t\bar{t}+\gamma+Jets$ (\texttt{MadGraph})&1.93236 & \texttt{TTGJets\_8TeV-madgraph/Summer12\_DR53X-PU\_S10\_START53\_V7A-v1}\\
$t\bar{t}\to 2\ell 2\nu 2B$ (\texttt{Powheg})&23.64 &\texttt{TTTo2L2Nu2B\_8TeV-powheg-pythia6/Summer12\_DR53X-PU\_S10\_START53\_V7A-v1 }\\
$t\bar{t}+W+Jets$ (\texttt{MadGraph})&0.232 &\texttt{TTWJets\_8TeV-madgraph/Summer12\_DR53X-PU\_S10\_START53\_V7A-v1 }\\
\hline
\end{tabular}
\end{center}
\end{sidewaystable}
%\end{table*}

The $qq/qg \to ZZ$ and $gg \to ZZ$ (tree-level only) processes are generated at leading-order (LO) for 0, 1 and 2 jets with
\texttt{MadGraph 5.1}~\cite{MadGraph5.1}, while the $gg \to ZZ$ (box diagrams) processes are generated at LO with \texttt{MCFM}~\cite{MCFM}.
For comparison we also use signal $qq/qg \to ZZ$ samples generated with \texttt{Powheg} (NLO for 0 jet)~\cite{PowhegMethod, PowhegBox} and \texttt{MadGraph5\_aMCatNLO} (NLO for 1 jet)~\cite{MGatNLO}. 
Vector boson scattering processes with two $Z$ bosons produced in association with 2 forward and backward jets are simulated using \texttt{Phantom}~\cite{Phantom}. 
Other diboson and triboson processes ($WZ$, $Zg$, $ZZZ$, $WZZ$, $WWZ$) and the $Z$ + jets samples are generated at 
LO with \texttt{MadGraph}, like events from $t\bar{t}$ production. The \texttt{PYTHIA 6.4}~\cite{Sjostrand:2006za} package is used for parton showering, hadronization, and the underlying event simulation. 
The default set of parton distribution functions (PDF) used for LO generators
is \texttt{CTEQ6L}~\cite{CTEQ6L}, whereas \texttt{CT10}~\cite{CT10} is used for NLO generators. The $ZZ$ yields from simulation
are scaled according to the theoretical cross sections calculated with MCFM.

The data samples used in this analysis correspond to an integrated luminosity of 19.7 fb$^{-1}$ at $\sqrt{s}= 8$ TeV
collected in 2012. The integrated luminosity is measured using data from the HF
system and the pixel detector~\cite{CMS-PAS-LUM-13-001}. The uncertainty in the integrated luminosity measurement is
2.6\%.

The data sets processed in this analysis are summarized in table~\ref{tab:datasamples}, along with the trigger selection operated on them (based on the same selection presented in~\cite{HiggsLegacyPaper, ZZXSPaper}).

\begin{table*}
\begin{center}
\topcaption{List of data samples used in the analysis: 22Jan2013ReReco reprocessing of DoubleMu, DoubleEle, MuEG; total integrated luminosity
of $19.7~\mathrm{fb}^{-1}$.}
\label{tab:datasamples}
\begin{tabular}{l}
\hline  \textbf{Data Samples} \\
\hline  DoubleMu/Run2012A-22Jan2013-v1\\
DoubleMuParked/Run2012B-22Jan2013-v1\\
DoubleMuParked/Run2012C-22Jan2013-v1\\
DoubleMuParked/Run2012D-22Jan2013-v1\\
DoubleElectron/Run2012A-22Jan2013-v1\\
DoubleElectron/Run2012B-22Jan2013-v1\\
DoubleElectron/Run2012C-22Jan2013-v1\\
DoubleElectron/Run2012D-22Jan2013-v1\\
MuEG/Run2012A-22Jan2013-v1\\
MuEG/Run2012B-22Jan2013-v1\\
MuEG/Run2012C-22Jan2013-v1\\
MuEG/Run2012D-22Jan2013-v1\\
\hline
\end{tabular}
\end{center}
\end{table*}


%% --------------------------------------------------------- %%


%% --------------------------------------------------------- %%
\section{Event Recontruction and Selection}
\label{sec:selection}
%The identification and reconstruction of the leptons produced in the event is the same as the one described in~\cite{ZZXSPaper}, including the final state radiation (FSR) treatment, which corrects the energy and isolation variables of the leptons from Z boson candidates when the radiated photons are
%identified. In addition, jets are reconstructed using the anti-$k_{T}$ clustering algorithm~\cite{antiKT}, as implemented in the \texttt{FASTJET} package~\cite{FASTJET}, with a distance parameter of 0.5. In order to assure a good reconstruction efficiency, identification quality criteria are imposed on jets based on the energy fraction of the charged, electromagnetic, and neutral hadronic components.\\
%Once reconstructed, jets are required to have a spatial separation from lepton candidates of $\Delta R = \sqrt{(\Delta\eta)^2+(\Delta\phi)^2}> 0.5$. Several effects contribute to bias the measured jet energy, such as pileup interactions, detector noise, and detector response non-uniformities in $\eta$ and non-linearities in $p_T$. The jet energy scale (JES) calibration~\cite{JESCalib, Khachatryan:2016kdb} relies on corrections which are parameterized in terms of the uncorrected $p_T$ and $\eta$ of the jet, and applied as multiplicative factors scaling the four-momentum vector of each jet. Furthermore, the jet energy resolution (JER) in data is known to be worse than in the simulation; therefore the simulated resolution is degraded to compensate for this effect. 
%A minimum threshold of 30 GeV on the transverse
%momentum of jets is required to ensure that they are well measured and
%to reduce the pileup contamination. Only jets with $|\eta| < $ 4.7 are considered.\\
%The selection is the same as~\cite{ZZXSPaper} and is here summarized briefly for convenience. Events are selected online from the presence of a pair of electrons or muons, or a triplet of electrons. Triggers requiring an electron together with a muon are also used.
%Electrons are selected if they have $|\eta^e| < 2.5$ and transverse momentum greater than 7 GeV,  while muons are considered if they have $|\eta^{\mu}| < 2.4$ and $p^{\mu}_T > 5~\mathrm{GeV}$.
%Moreover, electrons and muons must be isolated from the other particles and are required to originate from the primary vertex of the collision.  Leptons coming from tau decays are not considered as signal leptons. Each of the two Z candidates is formed of a pair of leptons of the same flavor and opposite charge that satisfies $60 < m_{\ell\ell} < 120$ GeV.

%\begin{table*}[htbH]
%\begin{center}
%\topcaption{Scaling factors applied on MC jets to described the data resolution\label{tab:c_JER}}
%\begin{tabular}{cccccccc}
%\hline $|\eta|$-region & $0.0-0.5$ & $0.5-1.1$ & $1.1-1.7$ & $1.7-2.3$ & $2.3-2.8$ & $2.8-3.2$ & $3.2-5.0$\\
%\hline $c$ & 1.079 & 1.099 & 1.121 & 1.208 & 1.254 & 1.395 & 1.056\\
%\hline
%\end{tabular}
%\end{center}
%\end{table*}


%% --------------------------------------------------------- %%


%% --------------------------------------------------------- %%
\section{Background Estimation}
\label{sec:background}
%The irreducible background is very small and it is estimated using MC samples (see Fig.~\ref{fig:irr_bkg}).
The lepton identification and isolation requirements significantly suppress the background contribution, and the remnant portion of it is very small compared to the signal.
We can identify two main background components: an irreducible background from processes that produce four genuine high-$p_T$ isolated leptons, such as pp~$\to \ttbar\Z$, pp~$\to$ WWZ and pp~$\to \ttbar$~WW,  and a reducible background from processes with less than four leptons, but with jets that are identified as leptons. The irreducible background is very small and it is estimated using MC samples (see Fig.~\ref{fig:irr_bkg}).
The main background contribution arises mostly from a Z boson produced in association with jets, as well as from \ttbar, WZ and WWW + jets. This component is estimated with data analyzed in dedicated control regions and is based on the probability for particles from jet fragmentation, which satisfy predefined loose selection criteria, to also pass the final selection criteria. The procedure is described in detail in~\cite{ZZXSPaper}. In Figure~\ref{fig:red_bkg} the contribution of the reducible background is reported as a function of the 4-lepton invariant mass and the jet multiplicity, and it is estimated both from MC samples and by using the ``fake-rate'' method. The lack of statistics makes the data-driven estimate necessary.
\begin{figure}[hbtp]
  \begin{center}
   % \includegraphics[width=\cmsFigWidth]{Figures/Mass_Irreducible_Bkg_MadGraphSet}
    %\includegraphics[width=\cmsFigWidth]{Figures/Jets_Irreducible_Bkg_MadGraphSet}
\includegraphics[width=\cmsFigWidth]{Figures/Mass_Irreducible_Bkg_pow_SR_4l}
\includegraphics[width=\cmsFigWidth]{Figures/nJets_Irreducible_Bkg_mad_SR_4l}
     \caption{Number of events of the irreducible background component in the signal region as a function of the invariant mass of the 4 lepton system (\cmsLeft) and the reconstructed number of jets produced in the event (\cmsRight).}
    \label{fig:irr_bkg}
  \end{center}
\end{figure}
\begin{figure}[hbtp]
  \begin{center}
   % \includegraphics[width=\cmsFigWidth]{Figures/Mass_Reducible_Bkg_mad_SR_4l.png}
    %\includegraphics[width=\cmsFigWidth]{Figures/Jets_Reducible_Bkg_MadGraphSet}%FIXME 
    \includegraphics[width=\cmsFigWidth]{Figures/Mass_Reducible_Bkg_mad_SR_4l}
    \includegraphics[width=\cmsFigWidth]{Figures/nJets_Reducible_Bkg_mad_SR_4l}
     \caption{Reducible background component in the signal region as a function of the invariant mass of the 4 lepton system (\cmsLeft) and the reconstructed number of jets produced in the event (\cmsRight). Points represent the data-driven estimate, the stacked histogram represents the Monte Carlo predictions, characterized by a very poor statistics.}
    \label{fig:red_bkg}
  \end{center}
\end{figure}
%The reconstructed four-lepton invariant mass and the number of jets distributions are shown in Fig.~\ref{fig:sig_contr_Mad}.\\
%\begin{figure}[hbtp]
 % \begin{center}
  %  \includegraphics[width=\cmsFigWidth]{Figures/Mass_Final_State_MadGraphSet}
   % \includegraphics[width=\cmsFigWidth]{Figures/Jets_Final_State_MadGraphSet}
    % \caption{ Distribution of the reconstructed four-lepton mass (\cmsLeft). Distribution of the reconstructed number of jets produced in the event  (\cmsRight). Points represent the data, the stacked histogram represents the \texttt{MadGraph+MCFM+Phantom} predictions for $ZZ$ signal. The estimate of irreducible and reducible backgrounds is also reported.}
   % \label{fig:sig_contr_Mad}
 % \end{center}
%\end{figure}

%% --------------------------------------------------------- %%


%% --------------------------------------------------------- %%
\section{Systematic Uncertainties}
\label{sec:systunc}
%(The same as AN2013-134: check values!! FIXME)\\
\label{sec:systunc}
Systematic uncertainties for trigger efficiency (1.5\%) are evaluated from data~\cite{ZZXSPaper}. Uncertainties
arising from lepton identification, isolation, tracking and impact parameter are 1-5\% for muons and electrons~\cite{ZZXSPaper}. The uncertainty in the LHC integrated
luminosity of the data sample is 2.6\%~\cite{CMS-PAS-LUM-13-001}. Theoretical uncertainties in the $ZZ \to \ell\ell\ell'\ell'$ 
acceptance are evaluated using \texttt{MCFM} and by varying the renormalization and factorization scales, up and down, 
by a factor of two with respect to the default values $\mu_R = \mu_F = m_Z$. The variations in the acceptance are 0.1\% (NLO
$q\bar{q} \to ZZ$) and 0.4\% ($gg \to ZZ$), and can be neglected. Uncertainties related to the choice
of the PDF and the strong coupling constant are evaluated following the \texttt{PDF4LHC} ~\cite{PDF4LHCRec, PDF4LHCRep, CT10} prescription and 
using \texttt{CT10},  \texttt{MSTW08}, and  \texttt{NNPDF}~\cite{NNPDF} PDF sets. They are found to be 1.2\% in the wider fiducial region and 0.8\% in the tighter fiducial region (fiducial regions are defined in the next section).
%4\% (NLO $q\bar{q} \to ZZ$) and 5\% ($gg \to ZZ$)~\cite{ZZXSPaper}.

The reducible background uncertainties in $Z$ + jets, $WZ$ + jets and $t\bar{t}$ yields reflect the uncertainties in the measured values of the misidentification rates and the limited statistics of the control regions in the data, and vary between 30\% and 70\% on the background yield, which corresponds to a $\sim$1\% of uncertainty on the cross-section measurement. The irreducible background uncertainty is smaller and varies between 13\% and 26\% on the yield (less than 1\% on the cross-section estimate). Table~\ref{tab:syst} summarizes the sources  of systematic uncertainty and their values.

\begin{table*}[htbH]
\begin{center}
\topcaption{Estimated systematic uncertainties on the signal yield used to compute the inclusive cross section measurement\label{tab:syst}}
\begin{tabular}{lclc}
\hline Systematic source & Uncertainty value\\
\hline Trigger & 1.5\%  \\
Lepton ID, ISO and Tracking & 1-5\%\\
Luminosity & 2.6\%\\
Reducible background & $\sim$1\%\\
Irreducible background & $<$1\%\\
Acceptance for PDF & 0.8-1.2\%\\
\hline
\end{tabular}
\end{center}
\end{table*}

% The uncertainty in the unfolding procedure discussed in Section 7 arises from differences between
% SHERPA and POWHEG for the unfolding factors (2–3%), from scale and PDF uncertainties
% (4–5%), and from experimental uncertainties (4–5%).


%% --------------------------------------------------------- %%



%% --------------------------------------------------------- %%
\section{Results}
\label{sec:results}
The differential pp~$\to \Z\Z\to \ell\ell\ell'\ell'$ cross section is determined as a function of the jet multiplicity, the invariant mass of the two most energetic jets ($m_{jj}$), the 
difference in pseudorapidity between them ($\Delta\eta_{jj}$), their transverse momentum and pseudorapidity 
(leading jet $p_{T}(\eta)^{jet1}$ and sub-leading jet $p_{T}(\eta)^{jet2}$). Both jets with  $|\eta^{jet}|<4.7$ and 
central jets with $|\eta^{jet}|<2.4$ are considered separately. Measurements are performed in a fiducial region, called ``wide region'',  defined by requiring the invariant mass of each Z boson candidate between 60 and 120 GeV. Moreover, they are also obtained in a ``tight region'', defined by requiring also leptons with $p^e_T > 7~\mathrm{GeV}$ and $|\eta^e| < 2.5$ if electrons and $p^{\mu}_T > 5~\mathrm{GeV}$ and $|\eta^{\mu}| < 2.4$ if muons, at least one lepton with $p_T > 20~\mathrm{GeV}$ and another one with $p_T > 10~\mathrm{GeV}$.
Each distribution is corrected for event selection efficiencies and for detector resolution effects in order to be 
compared with predictions from event generators. The correction procedure is based on the iterative D'Agostini's method unfolding technique~\cite{DAgostini}, implemented in the \texttt{RooUnfold} toolkit~\cite{RooUnfold}. The algorithm uses a response matrix that correlates the observable with and without detector effects.\\
For each measured distribution, a response matrix is evaluated using two different sets of generators: the first 
one is composed of signal samples generated with \texttt{MadGraph} ($\cPq\cPaq \to \Z\Z$), \texttt{MCFM} ($\cPg\cPg \to \Z\Z$) 
and \texttt{Phantom} ($\cPq\cPaq \to \Z\Z$ + 2 jets). In the second one the \texttt{Powheg} sample ($\cPq\cPaq \to \Z\Z$) is used 
instead of the \texttt{MadGraph} one. The \texttt{MadGraph} set is the reference set, while the  \texttt{Powheg} 
one is used for comparison purposes and to extract the systematic uncertainty due to the Monte Carlo generator choice. \\
Distributions of the corrected number of events are then divided by the bin width and by the integrated luminosity and are normalized to the unity. The normalized differential distributions as a function of the jet multiplicity and as a function of the invariant mass of the 4-lepton system are reported in Figs.~\ref{fig:xs_jets} and~\ref{fig:xs_mass} for the wide (left) and tight (right) fiducial regions.
\begin{figure}[hbtp]
  \begin{center}
    \includegraphics[width=1.2\cmsFigWidth]{Figures/diffCrossSecZZTo4lJets_Unfolded_MadGraph_norm.pdf}     
    \includegraphics[width=1.2\cmsFigWidth]{Figures/diffCrossSecZZTo4lJets_Unfolded_MadGraph_tight_norm.pdf}     
   \caption{\footnotesize{Normalized differential cross sections of $\Z\Z\to 4\ell$ processes as a function of the multiplicity of jets obtained in the wide (left) and tight (right) fiducial regions.}}
   \label{fig:xs_jets}
  \end{center}
\end{figure}
\begin{figure}[hbtp]
  \begin{center}
    \includegraphics[width=1.2\cmsFigWidth]{Figures/diffCrossSecZZTo4lMass_Unfolded_MadGraph_norm.pdf}     
    \includegraphics[width=1.2\cmsFigWidth]{Figures/diffCrossSecZZTo4lMass_Unfolded_MadGraph_tight_norm.pdf}     
   \caption{\footnotesize{Normalized differential cross sections of $\Z\Z\to 4\ell$ processes as a function of the invariant mass of the 4-lepton system obtained in the wide (left) and tight (right) fiducial regions.}}
   \label{fig:xs_mass}
  \end{center}
\end{figure}

%% --------------------------------------------------------- %%

%% --------------------------------------------------------- %%
%\section{Enhanced EWK Region of ZZ + 2jets Events}
%\label{sec:EWKregion}
%The scattering of two massive vector bosons (VBS) $VV\to VV$, with $V=W$ or $Z$, is the key process to probe the nature of the electroweak symmetry breaking (EWSB). In the absence of a Standard Model (SM) Higgs boson, the longitudinally polarized VBS amplitude increases as a function of the center-of-mass energy $\sqrt{s}$ and violates unitarity at energies around 1 TeV. The recent discovery of a 125 GeV SM-like Higgs boson at LHC~\cite{AtlasHiggs, CMSHiggs} provides a plausible explanation for the mechanism that unitarizes this process. However, many physics scenarios predict enhancements in the VBS amplitude either from additional resonances, or if the observed SM-like Higgs boson is only partially responsible for the EWSB.\\
At hadron collider VBS can be represented by an interaction of gauge bosons radiated from initial state quarks yielding a final state with two bosons and two jets, in a purely electroweak process. Two classes of processes can generate a $VV + 2 jet$ final state: the first class, that includes also VBS processes, involves only electroweak interactions of order $\alpha_{EW}^{4}$ (\emph{electroweak production}), while the second class involves both strong and electroweak processes of order $\alpha_{S}^{2}\alpha_{EW}^{2}$ (\emph{strong production}). A fiducial region is defined in order to enhance the purity of electroweak $ZZ+2jets$ and remove most of the strong events, which are considered as background in this analysis. This region is a subset of the one we used to measure the inclusive cross section of ZZ processes (see section \ref{sec:results}), in which two $Z$ bosons have a mass between 60 and 120 GeV. In addition to that inclusive region, it requires: at least two jets with an invariant mass ($m_{jj}$) larger than 300 GeV and separated in pseudorapidity by $\Delta\eta_{jj} > 2.4$, the leading(sub-leading) jet with $p_T>100(70)~\mathrm{GeV}$ and the magnitude of the missing transverse energy $E_T < 60~\mathrm{GeV}$. Only one event passing all selection requirements is observed in the data, compared to a SM expectation of 0.14 signal events and 0.47 background events. Figure~\ref{fig:ewk_distr} (left) shows the expected and observed $m_{4\ell}$ distribution after all fiducial region selection criteria are applied, except $\Delta\eta > 2.4$. In this region 4 events are observed and their kinematics is reported in Table~\ref{tab:4ev}. Figure~\ref{fig:ewk_distr} (right) shows the $m_{jj}$ distribution after the whole selection. All three dilepton channels are summed in both plots.
\begin{figure}[hbtp]
  \begin{center}
    \includegraphics[width=\cmsFigWidth]{Figures/Mass_pt100pt70_met60_mjj200_mad.png}     
    \includegraphics[width=\cmsFigWidth]{Figures/Mjj_pt100pt70_met60_deta24_mad.png}     
       \caption{The $m_{4\ell}$ distribution (left) for events passing the region selections except for the $\Delta\eta_{jj} > 2.4$ selection. The $m_{jj}$ distribution (right) for events passing all the region selections. Points represent the data, the stacked filled histogram represents the predictions for $ZZ$ signal (from \texttt{Phantom}) and background contributions using \texttt{MadGraph} samples to describe $q\bar{q}(qg)\to ZZ\to 4\ell$ processes (while for the stacked histogram outlined in red the \texttt{Powheg} simulation is used).}
    \label{fig:ewk_distr}
  \end{center}
\end{figure}

\begin{table*}[htbH]
\begin{center}
\topcaption{Kinematic values of events observed in the fiducial region without requiring $\Delta\eta_{jj} > 2.4$. \label{tab:4ev}}
\begin{tabular}{lccccc}
\hline Event & $m_{4\ell}$ & $m_{jj}$  & $\Delta\eta_{jj}$ & $p_T^{jet1}$ & $p_T^{jet2}$\\
\hline 1 & 216.6 & 346.6 & 2.64 & 100.3 & 96.47 \\
\hline 2 & 277.7 & 495.2 & 1.20 & 289.5 & 164.2 \\
\hline 3 & 287.2 & 391.5 & 0.97 & 207.5 & 150.4 \\
\hline 4 & 442.5 & 421.9 & 1.59 & 252.3 & 102.9\\
\hline
\end{tabular}
\end{center}
\end{table*}


%% --------------------------------------------------------- %%


%% --------------------------------------------------------- %%
%\section{Conclusions}
%\label{sec:conclusions}
%The measurement of the $ZZ$ production total and differential cross-sections in pp collisions at $\sqrt{s}=8~\mathrm{TeV}$ 
has been presented in the leptonic decay channels  $pp \to ZZ\to \ell\ell\ell'\ell'$ with $\ell,\ell'=e,\mu$. The full 2012 dataset is used corresponding to an integrated luminosity of $19.7~\mathrm{fb}^{-1}$. Simple sequential sets of lepton reconstruction, identification and isolation cuts and a set of kinematic cuts are used following the selection introduced in the search for the Higgs boson decaying in 4 leptons (high mass selection). The main backgrounds are estimated using Monte Carlo samples and data-driven techniques and found to be very small for $60 < m_{Z_{1}} < 120$ and $60 < m_{Z_{2}} < 120$ GeV.\\
The measured combined cross section is $\sigma_{pp\to ZZ \to 4\ell} =20.50 \pm 1.26~\mathrm{(stat.)}^{+0.58}_{- 0.64}~\mathrm{(syst.)}\pm 0.53~\mathrm{(lumi.)}$,
 in agreement with the SM prediction of $20.21^{+3.3\%}_{-2.6\%}$ fb from~\cite{grazzini}. \\
Differential distributions for jet-related variables are presented and no significant discrepancy with respect to predictions is found.\\
A tighter fiducial region is defined in order to select electroweak $ZZ$ +2 jets events. The observation of one event in this region agrees with the SM expectation.

%% --------------------------------------------------------- %%

%\clearpage

%\appendix
%\section{Unfolding}
%\label{sec:unfolding}
%In high energy physics the measurement of physical observables is usually distorted by several effects, such as finite resolution and limited acceptance of the detector.
The observed spectrum of physics quantities is thus considered as a ``noisy distortion'' of the true one, i.e. the distribution one would observed under idealized 
conditions (ideal detector, no backgrounds...). An important task of the experimental method is therefore to extract the true distribution from the 
observed one, correcting for distortion and noise. This can be done by an unfolding procedure, that allows a direct comparison of the data obtained using different 
detectors with each other and with theoretical predictions.\\
\\
The measurement discussed here is based on the full data sample collected in proton-proton collisions during 2012 with the Compact Muon Solenoid Experiment (CMS) at the Large Hadron Collider (LHC), corresponding to the integrated luminosity of 19.7 fb$^{-1}$ at a center of mass energy of $\sqrt{s}$ = 8 TeV. The ZZ production cross section 
is measured differentially as a function of the invariant mass of the four-lepton system, the number of jets produced in the 
event, the invariant mass of the two most energetic jets ($m_{jj}$), the difference of pseudorapidity between them 
($\Delta\eta_{jj}$) and their transverse momentum and pseudorapidity. 
The measurements are performed in the leptonic decay modes ZZ $\to \ell\ell\ell'\ell'$ channel ($\ell,\ell' = e, \mu$). \\
\\
The unfolding procedure is based on the so-called ``response matrix'', derived using simulated Monte Carlo, which is a mapping between the true value (generated) 
of the observable and the reconstructed (measured) value, distorted by detector effects. Because of inevitable statistical uncertainties in the measured
distribution, the exact solution  that one would obtain inverting the response matrix (if it exists) can lead to unacceptable results, wildly oscillating and useless. 
In order to avoid this unstable behavior, a ``regularization condition'' can be imposed, requiring a smooth true distribution. The regularized unfolding methods 
investigated here, as implemented in the RooUnfold package~\cite{RooUnfold}, are the Singular Value Decomposition (SVD) method~\cite{SVD} and the iterative Bayesian unfolding~\cite{DAgostini}.

\subsection{Unfolding procedure}
The distribution of a measured observable be stored in a vector $N_{rec}$ of dimension $n$, where the $ith$ coordinate of the vector contains the number of entries in the corresponding bin of the histogram~\cite{SVD, SMP-14-016}. The measurement is affected by the finite experimental resolution and/or the limited acceptance of the detector, so that each event from the true distribution may find itself in a range of (not necessarily) adjacent bins, or nowhere at all.\\
\\
We generate the distribution $N_{gen}$ of dimension $m$ according to some idea of the physical process under study, and perform the detector simulation. At this stage, every event in a measured bin $i$ can be directly matched to the generated one $j$. A well defined system of linear equations is thus determined, describing the relations between the simulated true and measured distributions:
$$\sum_{j}A_{ij}N^{j}_{gen}=N^{i}_{rec}.$$
The $A_{ij}$ matrix is the response matrix that maps the binned generated distribution onto the measured one: each element $(i,j)$ is indeed related to the 
probability that the observable generated in the $jth$ bin would be measured in the $ith$ bin. The unfolding procedure applies the response matrix to the 
measured data distribution (in which background is subtracted), taking into account the measurement uncertainties due to statistical fluctuations 
in the finite measured sample through the covariance matrix:
$$\sum_{j}A_{ij}N^{j}_{gen}=N^{i}_{sig} := N_{data}^{i}-N^{i}_{bkg}.$$
The linear system of equations can be solved using the exact inverse of the response matrix on measured histogram and obtaining a data distribution 
``at particle level''. However a direct inversion of the matrix usually leads to completely unacceptable wildly oscillating results. In order to overcome this 
problem, different regularized algorithms can be used, such as the Singular Value Decomposition (SVD) and the iterative ``Baysian'' method, well described 
in~\cite{SVD} and~\cite{DAgostini}.\\
\\
%from WW
The response matrices used in the analysis are obtained from Monte Carlo samples that contain signal-only events with pile-up reweighting, lepton and trigger efficiency applied. Two sets of samples are employed, the \texttt{Madgraph} + \texttt{MCFM} + \texttt{Phantom} (as reference) and \texttt{Powheg} + \texttt{MCFM} + \texttt{Phantom} (as check) sets (for the $m_{ZZ}$ distribution the role of the two sets of samples is switched). Leptons are generated requiring  $m_{\ell^{+}\ell^{-}}> 4$ GeV in all samples but \texttt{MadGraph}, in which  $m_{\ell^{+}\ell^{-}} > 12$ GeV, and reconstructed  following the same steps of~\cite{HiggsLegacyPaper,ZZXSPaper}. Jets are generated with $p_{T} > 10$ GeV and reconstructed following the criteria recommended by Jet-MET group~\cite{JetID}. Both generated events not measured because of detection or selection inefficiency and reconstructed events not generated as signal are also taken into account in the unfolding procedure.\\
\\
In the following, first the choice of the binning of the distributions is discussed, in order to control migration from one bin to others. Then the performance of the unfolding is validated through studies on Monte Carlo samples and applied on data distribution. Finally the different sources of systematic errors are investigated and  uncertainties are evaluated.

\subsection{Bin-to-bin migration}
Because of the finite experimental resolution, events that are actually produced (generated) in one bin might be measured (reconstructed) in another bin, leading to migrations of events across bin boundaries. This bin-to-bin migration is studied in term of
purity and stability in each bin. As in~\cite{SMP-14-016}, the purity $p_i$ is defined as the number of events that are
generated and correctly reconstructed in a given bin $i$ divided by the number of events that are reconstructed in bin $i$, 
but generated anywhere. On the other hand, the stability is defined as the number of events that are
generated and correctly reconstructed in a given bin $i$ divided by the number of events that are generated in bin $i$, but 
reconstructed anywhere:
$$p_i=\frac{N^i_{gen\&reco}}{N_{reco}^i}; \ \ s_i=\frac{N^i_{gen\&reco}}{N_{gen}^i},$$

where $reco$ refers to reconstructed events fulfilling the full selection requirements described above and $gen$ refers to 
generated events satisfying the phase space requirements. 
Without migration effects, purity and stability would be equal to one. The purity
(stability) is sensitive to migrations into (out of) the bin. In order to keep bin-to-bin migrations
acceptably small, the bin widths for each observable are optimized such that for each bin purity
and stability are greater than about 70\%.  Figures~\ref{fig:ps_mass},~\ref{fig:ps_jets},~\ref{fig:ps_mjj},~\ref{fig:ps_deta},~\ref{fig:ps_jet1} and ~\ref{fig:ps_jet2} show the purity and stability for the final binning definition. The binning selection for each observable is define as:\\
\begin{itemize}
\item $m_{4\ell}$: $\{100,200,250,300,350,400,500,600,\ge 800\}$
\item $N\ jets$: $\{0,1,2, >2\}$
\item $m_{jj}$: $\{0,200,\ge 800\}$ 
\item $\Delta\eta_{jj}$: $\{0,2.4,\ge4.7\}$
\item $p_{T}^{jet1}$: $\{30,50,100,200,300,\ge 500)\}$
\item $p_{T}^{jet2}$: $\{30,100,200,\ge 500\}$
\item $\eta^{jet1}$,  $\eta^{jet2}$: $\{0,1.5,3,4.7\}$
\end{itemize}
\begin{figure}[hbtp]
  \begin{center}
    \includegraphics[width=0.8\cmsFigWidth]{Figures/Unfolding/BinMigration/PurityStability_4m_Mass_Pow}
    \includegraphics[width=0.8\cmsFigWidth]{Figures/Unfolding/BinMigration/PurityStability_4e_Mass_Pow}
    \includegraphics[width=0.8\cmsFigWidth]{Figures/Unfolding/BinMigration/PurityStability_2e2m_Mass_Pow}
    \caption{Purity and stability as a function of the 4-lepton invariant mass, for the $4\mu$ (left), $4e$ (center) and $2e2\mu$ (right) final states.}
    \label{fig:ps_mass}
  \end{center}
\end{figure}
\begin{figure}[hbtp]
  \begin{center}
    \includegraphics[width=0.8\cmsFigWidth]{Figures/Unfolding/BinMigration/PurityStability_4m_Jets_Mad}
    \includegraphics[width=0.8\cmsFigWidth]{Figures/Unfolding/BinMigration/PurityStability_4e_Jets_Mad}
    \includegraphics[width=0.8\cmsFigWidth]{Figures/Unfolding/BinMigration/PurityStability_2e2m_Jets_Mad}
    \includegraphics[width=0.8\cmsFigWidth]{Figures/Unfolding/BinMigration/PurityStability_4m_CentralJets_Mad}
    \includegraphics[width=0.8\cmsFigWidth]{Figures/Unfolding/BinMigration/PurityStability_4e_CentralJets_Mad}
    \includegraphics[width=0.8\cmsFigWidth]{Figures/Unfolding/BinMigration/PurityStability_2e2m_CentralJets_Mad}
 \caption{Purity and stability as a function of the number of jets (top) and central jets (bottom) in the event,  for the $4\mu$ (left), $4e$ (center) and $2e2\mu$ (right) final states.}
    \label{fig:ps_jets}
  \end{center}
\end{figure}
\begin{figure}[hbtp]
  \begin{center}
    \includegraphics[width=0.8\cmsFigWidth]{Figures/Unfolding/BinMigration/PurityStability_4m_Mjj_Mad}
    \includegraphics[width=0.8\cmsFigWidth]{Figures/Unfolding/BinMigration/PurityStability_4e_Mjj_Mad}
    \includegraphics[width=0.8\cmsFigWidth]{Figures/Unfolding/BinMigration/PurityStability_2e2m_Mjj_Mad}
    \includegraphics[width=0.8\cmsFigWidth]{Figures/Unfolding/BinMigration/PurityStability_4m_CentralMjj_Mad}
    \includegraphics[width=0.8\cmsFigWidth]{Figures/Unfolding/BinMigration/PurityStability_4e_CentralMjj_Mad}
    \includegraphics[width=0.8\cmsFigWidth]{Figures/Unfolding/BinMigration/PurityStability_2e2m_CentralMjj_Mad}
 \caption{Purity and stability as a function of the invariant mass of the two most energetic jets (top) and  central jets (bottom) in the event,  for the $4\mu$ (left), $4e$ (center) and $2e2\mu$ (right) final states.}
    \label{fig:ps_mjj}
  \end{center}
\end{figure}
\begin{figure}[hbtp]
  \begin{center}
    \includegraphics[width=0.8\cmsFigWidth]{Figures/Unfolding/BinMigration/PurityStability_4m_Deta_Mad}
    \includegraphics[width=0.8\cmsFigWidth]{Figures/Unfolding/BinMigration/PurityStability_4e_Deta_Mad}
    \includegraphics[width=0.8\cmsFigWidth]{Figures/Unfolding/BinMigration/PurityStability_2e2m_Deta_Mad}
    \includegraphics[width=0.8\cmsFigWidth]{Figures/Unfolding/BinMigration/PurityStability_4m_CentralDeta_Mad}
    \includegraphics[width=0.8\cmsFigWidth]{Figures/Unfolding/BinMigration/PurityStability_4e_CentralDeta_Mad}
    \includegraphics[width=0.8\cmsFigWidth]{Figures/Unfolding/BinMigration/PurityStability_2e2m_CentralDeta_Mad}
 \caption{Purity and stability as a function of the $\Delta\eta$ between the two most energetic jets (top) and  central jets (bottom) in the event,  for the $4\mu$ (left), $4e$ (center) and $2e2\mu$ (right) final states.}
    \label{fig:ps_deta}
  \end{center}
\end{figure}

\begin{figure}[hbtp]
  \begin{center}
    \includegraphics[width=0.8\cmsFigWidth]{Figures/Unfolding/BinMigration/PurityStability_4m_PtJet1_Mad}
    \includegraphics[width=0.8\cmsFigWidth]{Figures/Unfolding/BinMigration/PurityStability_4e_PtJet1_Mad}
    \includegraphics[width=0.8\cmsFigWidth]{Figures/Unfolding/BinMigration/PurityStability_2e2m_PtJet1_Mad}
    \includegraphics[width=0.8\cmsFigWidth]{Figures/Unfolding/BinMigration/PurityStability_4m_EtaJet1_Mad}
    \includegraphics[width=0.8\cmsFigWidth]{Figures/Unfolding/BinMigration/PurityStability_4e_EtaJet1_Mad}
    \includegraphics[width=0.8\cmsFigWidth]{Figures/Unfolding/BinMigration/PurityStability_2e2m_EtaJet1_Mad}
 \caption{Purity and stability as a function of the $p_T$ (top) and $\eta$ (bottom) of the most energetic jet in the event,  for the $4\mu$ (left), $4e$ (center) and $2e2\mu$ (right) final states.}
    \label{fig:ps_jet1}
  \end{center}
\end{figure}
\begin{figure}[hbtp]
  \begin{center}
    \includegraphics[width=0.8\cmsFigWidth]{Figures/Unfolding/BinMigration/PurityStability_4m_PtJet2_Mad}
    \includegraphics[width=0.8\cmsFigWidth]{Figures/Unfolding/BinMigration/PurityStability_4e_PtJet2_Mad}
    \includegraphics[width=0.8\cmsFigWidth]{Figures/Unfolding/BinMigration/PurityStability_2e2m_PtJet2_Mad}
    \includegraphics[width=0.8\cmsFigWidth]{Figures/Unfolding/BinMigration/PurityStability_4m_EtaJet2_Mad}
    \includegraphics[width=0.8\cmsFigWidth]{Figures/Unfolding/BinMigration/PurityStability_4e_EtaJet2_Mad}
    \includegraphics[width=0.8\cmsFigWidth]{Figures/Unfolding/BinMigration/PurityStability_2e2m_EtaJet2_Mad}
 \caption{Purity and stability as a function of the $p_T$ (top) and $\eta$ (bottom) of the second most energetic jet in the event,  for the $4\mu$ (left), $4e$ (center) and $2e2\mu$ (right) final states.}
    \label{fig:ps_jet2}
  \end{center}
\end{figure}

\clearpage
\subsection{Unfolding validation test on Monte Carlo}
Before looking at the data, it is recommended to test the unfolding procedure on MC events alone. 
As first step the consistency of the whole process is checked using the full \texttt{MadGraph} (or \texttt{Powheg}) set of samples,
both for the distribution to be unfolded and the response matrix. If everything is
correctly implemented the unfolded distribution and the generated one should be exactly the
same. Moreover, in order to get meaningful results, the distribution that has to be unfolded must be statistically independent 
from the 2-dimensional response histogram. The \texttt{MadGraph}(\texttt{Powheg}) set is thus split into two samples: one of them
is used to fill the response matrix, while the other one is used to build the distribution to unfold. Finally, in order to compare the effect of employing different signal samples and to be sure the procedure is
independent of the choice of a particular MC, the response matrix from the \texttt{MadGraph} set is applied on the distribution 
obtained using the \texttt{Powheg} set and vice versa. Tests are performed both in the standard and tight fiducial regions and results for the latter case are reported from Figure~\ref{fig:MCtest_Mass1} to Figure~\ref{fig:MCtest_EtaJet22} for the considered distributions, in the three different final states. In each plot the ratio of unfolded over generated
distribution is shown and, as expected, it is unity. \\
Closure tests show that the unfolding procedure doesn't introduce any additional bias and demonstrate its effectiveness. 

%% , as it is shown in figures~\ref{fig:FullMad_4e}, ~\ref{fig:FullMad_4m} and ~\ref{fig:FullMad_2e2m} (or in figures~\ref{fig:FullPow_4e}, ~\ref{fig:FullPow_4m} and ~\ref{fig:FullPow_2e2m} for \texttt{Powheg}) for the four considered distributions, in the three different final 
%% states. In each plot the ratio of unfolded over generated
%% distribution is shown and, as expected, it is unity.\\

%% Figures~\ref{fig:HalfMad_4e}, ~\ref{fig:HalfMad_4m} and ~\ref{fig:HalfMad_2e2m}(~\ref{fig:HalfPow_4e}, ~\ref{fig:HalfPow_4m} and ~\ref{fig:HalfPow_2e2m}) show the result of this closure test.\\
%%  Results are reported in figures~\ref{fig:MadMat_PowDist_4e} -~\ref{fig:MadMat_PowDist_4m} 
%% -~\ref{fig:MadMat_PowDist_2e2m} and~\ref{fig:PowMat_MadDist_4e} -~\ref{fig:PowMat_MadDist_4m} -~\ref{fig:PowMat_MadDist_2e2m}.\\


\begin{figure}[hbtp]
  \begin{center}
    \includegraphics[width=0.8\cmsFigWidth]{Figures/Unfolding/MCTests/Mass_ZZTo4e_MadMatrix_MadDistr_FullSample_fr}     
    \includegraphics[width=0.8\cmsFigWidth]{Figures/Unfolding/MCTests/Mass_ZZTo4m_MadMatrix_MadDistr_FullSample_fr}     
    \includegraphics[width=0.8\cmsFigWidth]{Figures/Unfolding/MCTests/Mass_ZZTo2e2m_MadMatrix_MadDistr_FullSample_fr}
     \includegraphics[width=0.8\cmsFigWidth]{Figures/Unfolding/MCTests/Mass_ZZTo4e_PowMatrix_PowDistr_FullSample_fr}     
    \includegraphics[width=0.8\cmsFigWidth]{Figures/Unfolding/MCTests/Mass_ZZTo4m_PowMatrix_PowDistr_FullSample_fr}     
    \includegraphics[width=0.8\cmsFigWidth]{Figures/Unfolding/MCTests/Mass_ZZTo2e2m_PowMatrix_PowDistr_FullSample_fr}      
    \includegraphics[width=0.8\cmsFigWidth]{Figures/Unfolding/MCTests/Mass_ZZTo4e_MadMatrix_MadDistr_HalfSample_fr}     
    \includegraphics[width=0.8\cmsFigWidth]{Figures/Unfolding/MCTests/Mass_ZZTo4m_MadMatrix_MadDistr_HalfSample_fr}     
    \includegraphics[width=0.8\cmsFigWidth]{Figures/Unfolding/MCTests/Mass_ZZTo2e2m_MadMatrix_MadDistr_HalfSample_fr}     
    \includegraphics[width=0.8\cmsFigWidth]{Figures/Unfolding/MCTests/Mass_ZZTo4e_PowMatrix_PowDistr_HalfSample_fr}     
    \includegraphics[width=0.8\cmsFigWidth]{Figures/Unfolding/MCTests/Mass_ZZTo4m_PowMatrix_PowDistr_HalfSample_fr}     
 \includegraphics[width=0.8\cmsFigWidth]{Figures/Unfolding/MCTests/Mass_ZZTo2e2m_PowMatrix_PowDistr_HalfSample_fr}        
      \caption{Unfolding tests. From top to bottom: \texttt{MadGraph} matrix applied on \texttt{MadGraph} distribution using the full set, \texttt{Powheg} matrix applied on \texttt{Powheg} distribution using the full set,  \texttt{MadGraph} matrix applied on \texttt{MadGraph} distribution using the two different halves of the total sample set, \texttt{Powheg} matrix applied on \texttt{Powheg} distribution using the two different halves of the total sample set. Results are reported as a function of the 4-lepton mass system for the $4e$ (left), $4\mu$ (center) and $2e2\mu$ (right) final states.}
    \label{fig:MCtest_Mass1}
  \end{center}
\end{figure}

\begin{figure}[hbtp]
  \begin{center}
    \includegraphics[width=0.8\cmsFigWidth]{Figures/Unfolding/MCTests/Mass_ZZTo4e_MadMatrix_PowDistr_FullSample_fr}     
    \includegraphics[width=0.8\cmsFigWidth]{Figures/Unfolding/MCTests/Mass_ZZTo4m_MadMatrix_PowDistr_FullSample_fr}     
    \includegraphics[width=0.8\cmsFigWidth]{Figures/Unfolding/MCTests/Mass_ZZTo2e2m_MadMatrix_PowDistr_FullSample_fr}
     \includegraphics[width=0.8\cmsFigWidth]{Figures/Unfolding/MCTests/Mass_ZZTo4e_PowMatrix_MadDistr_FullSample_fr}     
    \includegraphics[width=0.8\cmsFigWidth]{Figures/Unfolding/MCTests/Mass_ZZTo4m_PowMatrix_MadDistr_FullSample_fr}     
    \includegraphics[width=0.8\cmsFigWidth]{Figures/Unfolding/MCTests/Mass_ZZTo2e2m_PowMatrix_MadDistr_FullSample_fr}  
 \caption{Unfolding tests. \texttt{MadGraph} matrix applied on \texttt{Powheg} distribution using the full set (top), \texttt{Powheg} matrix applied on \texttt{MadGraph} distribution using the full set (bottom). Results are reported as a function of the 4-lepton mass system for the $4e$ (left), $4\mu$ (center) and $2e2\mu$ (right) final states.}
    \label{fig:MCtest_Mass2}
  \end{center}
\end{figure}
\clearpage
\begin{figure}[hbtp]
  \begin{center}
    \includegraphics[width=0.8\cmsFigWidth]{Figures/Unfolding/MCTests/Jets_ZZTo4e_MadMatrix_MadDistr_FullSample_fr}     
    \includegraphics[width=0.8\cmsFigWidth]{Figures/Unfolding/MCTests/Jets_ZZTo4m_MadMatrix_MadDistr_FullSample_fr}     
    \includegraphics[width=0.8\cmsFigWidth]{Figures/Unfolding/MCTests/Jets_ZZTo2e2m_MadMatrix_MadDistr_FullSample_fr}
     \includegraphics[width=0.8\cmsFigWidth]{Figures/Unfolding/MCTests/Jets_ZZTo4e_PowMatrix_PowDistr_FullSample_fr}     
    \includegraphics[width=0.8\cmsFigWidth]{Figures/Unfolding/MCTests/Jets_ZZTo4m_PowMatrix_PowDistr_FullSample_fr}     
    \includegraphics[width=0.8\cmsFigWidth]{Figures/Unfolding/MCTests/Jets_ZZTo2e2m_PowMatrix_PowDistr_FullSample_fr}      
    \includegraphics[width=0.8\cmsFigWidth]{Figures/Unfolding/MCTests/Jets_ZZTo4e_MadMatrix_MadDistr_HalfSample_fr}     
    \includegraphics[width=0.8\cmsFigWidth]{Figures/Unfolding/MCTests/Jets_ZZTo4m_MadMatrix_MadDistr_HalfSample_fr}     
    \includegraphics[width=0.8\cmsFigWidth]{Figures/Unfolding/MCTests/Jets_ZZTo2e2m_MadMatrix_MadDistr_HalfSample_fr}     
    \includegraphics[width=0.8\cmsFigWidth]{Figures/Unfolding/MCTests/Jets_ZZTo4e_PowMatrix_PowDistr_HalfSample_fr}     
    \includegraphics[width=0.8\cmsFigWidth]{Figures/Unfolding/MCTests/Jets_ZZTo4m_PowMatrix_PowDistr_HalfSample_fr}     
 \includegraphics[width=0.8\cmsFigWidth]{Figures/Unfolding/MCTests/Jets_ZZTo2e2m_PowMatrix_PowDistr_HalfSample_fr}        
      \caption{Unfolding tests. From top to bottom: \texttt{MadGraph} matrix applied on \texttt{MadGraph} distribution using the full set, \texttt{Powheg} matrix applied on \texttt{Powheg} distribution using the full set,  \texttt{MadGraph} matrix applied on \texttt{MadGraph} distribution using the two different halves of the total sample set, \texttt{Powheg} matrix applied on \texttt{Powheg} distribution using the two different halves of the total sample set. Results are reported as a function of the number of jets for the $4e$ (left), $4\mu$ (center) and $2e2\mu$ (right) final states.}
    \label{fig:MCtest_Jets1}
  \end{center}
\end{figure}

\begin{figure}[hbtp]
  \begin{center}
    \includegraphics[width=0.8\cmsFigWidth]{Figures/Unfolding/MCTests/Jets_ZZTo4e_MadMatrix_PowDistr_FullSample_fr}     
    \includegraphics[width=0.8\cmsFigWidth]{Figures/Unfolding/MCTests/Jets_ZZTo4m_MadMatrix_PowDistr_FullSample_fr}     
    \includegraphics[width=0.8\cmsFigWidth]{Figures/Unfolding/MCTests/Jets_ZZTo2e2m_MadMatrix_PowDistr_FullSample_fr}
     \includegraphics[width=0.8\cmsFigWidth]{Figures/Unfolding/MCTests/Jets_ZZTo4e_PowMatrix_MadDistr_FullSample_fr}     
    \includegraphics[width=0.8\cmsFigWidth]{Figures/Unfolding/MCTests/Jets_ZZTo4m_PowMatrix_MadDistr_FullSample_fr}     
    \includegraphics[width=0.8\cmsFigWidth]{Figures/Unfolding/MCTests/Jets_ZZTo2e2m_PowMatrix_MadDistr_FullSample_fr}  
 \caption{Unfolding tests. \texttt{MadGraph} matrix applied on \texttt{Powheg} distribution using the full set (top), \texttt{Powheg} matrix applied on \texttt{MadGraph} distribution using the full set (bottom). Results are reported as a function of the number of jets for the $4e$ (left), $4\mu$ (center) and $2e2\mu$ (right) final states.}
    \label{fig:MCtest_Jets2}
  \end{center}
\end{figure}
\clearpage
\begin{figure}[hbtp]
  \begin{center}
    \includegraphics[width=0.8\cmsFigWidth]{Figures/Unfolding/MCTests/CentralJets_ZZTo4e_MadMatrix_MadDistr_FullSample_fr}     
    \includegraphics[width=0.8\cmsFigWidth]{Figures/Unfolding/MCTests/CentralJets_ZZTo4m_MadMatrix_MadDistr_FullSample_fr}     
    \includegraphics[width=0.8\cmsFigWidth]{Figures/Unfolding/MCTests/CentralJets_ZZTo2e2m_MadMatrix_MadDistr_FullSample_fr}
     \includegraphics[width=0.8\cmsFigWidth]{Figures/Unfolding/MCTests/CentralJets_ZZTo4e_PowMatrix_PowDistr_FullSample_fr}     
    \includegraphics[width=0.8\cmsFigWidth]{Figures/Unfolding/MCTests/CentralJets_ZZTo4m_PowMatrix_PowDistr_FullSample_fr}     
    \includegraphics[width=0.8\cmsFigWidth]{Figures/Unfolding/MCTests/CentralJets_ZZTo2e2m_PowMatrix_PowDistr_FullSample_fr}      
    \includegraphics[width=0.8\cmsFigWidth]{Figures/Unfolding/MCTests/CentralJets_ZZTo4e_MadMatrix_MadDistr_HalfSample_fr}     
    \includegraphics[width=0.8\cmsFigWidth]{Figures/Unfolding/MCTests/CentralJets_ZZTo4m_MadMatrix_MadDistr_HalfSample_fr}     
    \includegraphics[width=0.8\cmsFigWidth]{Figures/Unfolding/MCTests/CentralJets_ZZTo2e2m_MadMatrix_MadDistr_HalfSample_fr}     
    \includegraphics[width=0.8\cmsFigWidth]{Figures/Unfolding/MCTests/CentralJets_ZZTo4e_PowMatrix_PowDistr_HalfSample_fr}     
    \includegraphics[width=0.8\cmsFigWidth]{Figures/Unfolding/MCTests/CentralJets_ZZTo4m_PowMatrix_PowDistr_HalfSample_fr}     
 \includegraphics[width=0.8\cmsFigWidth]{Figures/Unfolding/MCTests/CentralJets_ZZTo2e2m_PowMatrix_PowDistr_HalfSample_fr}        
      \caption{Unfolding tests. From top to bottom: \texttt{MadGraph} matrix applied on \texttt{MadGraph} distribution using the full set, \texttt{Powheg} matrix applied on \texttt{Powheg} distribution using the full set,  \texttt{MadGraph} matrix applied on \texttt{MadGraph} distribution using the two different halves of the total sample set, \texttt{Powheg} matrix applied on \texttt{Powheg} distribution using the two different halves of the total sample set. Results are reported as a function of the number of central jets for the $4e$ (left), $4\mu$ (center) and $2e2\mu$ (right) final states.}
    \label{fig:MCtest_CentralJets1}
  \end{center}
\end{figure}

\begin{figure}[hbtp]
  \begin{center}
    \includegraphics[width=0.8\cmsFigWidth]{Figures/Unfolding/MCTests/CentralJets_ZZTo4e_MadMatrix_PowDistr_FullSample_fr}     
    \includegraphics[width=0.8\cmsFigWidth]{Figures/Unfolding/MCTests/CentralJets_ZZTo4m_MadMatrix_PowDistr_FullSample_fr}     
    \includegraphics[width=0.8\cmsFigWidth]{Figures/Unfolding/MCTests/CentralJets_ZZTo2e2m_MadMatrix_PowDistr_FullSample_fr}
     \includegraphics[width=0.8\cmsFigWidth]{Figures/Unfolding/MCTests/CentralJets_ZZTo4e_PowMatrix_MadDistr_FullSample_fr}     
    \includegraphics[width=0.8\cmsFigWidth]{Figures/Unfolding/MCTests/CentralJets_ZZTo4m_PowMatrix_MadDistr_FullSample_fr}     
    \includegraphics[width=0.8\cmsFigWidth]{Figures/Unfolding/MCTests/CentralJets_ZZTo2e2m_PowMatrix_MadDistr_FullSample_fr}  
 \caption{Unfolding tests. \texttt{MadGraph} matrix applied on \texttt{Powheg} distribution using the full set (top), \texttt{Powheg} matrix applied on \texttt{MadGraph} distribution using the full set (bottom). Results are reported as a function of the number of central jets for the $4e$ (left), $4\mu$ (center) and $2e2\mu$ (right) final states.}
    \label{fig:MCtest_CentralJets2}
  \end{center}
\end{figure}
\clearpage 
\begin{figure}[hbtp]
  \begin{center}
    \includegraphics[width=0.8\cmsFigWidth]{Figures/Unfolding/MCTests/Mjj_ZZTo4e_MadMatrix_MadDistr_FullSample_fr}     
    \includegraphics[width=0.8\cmsFigWidth]{Figures/Unfolding/MCTests/Mjj_ZZTo4m_MadMatrix_MadDistr_FullSample_fr}     
    \includegraphics[width=0.8\cmsFigWidth]{Figures/Unfolding/MCTests/Mjj_ZZTo2e2m_MadMatrix_MadDistr_FullSample_fr}
     \includegraphics[width=0.8\cmsFigWidth]{Figures/Unfolding/MCTests/Mjj_ZZTo4e_PowMatrix_PowDistr_FullSample_fr}     
    \includegraphics[width=0.8\cmsFigWidth]{Figures/Unfolding/MCTests/Mjj_ZZTo4m_PowMatrix_PowDistr_FullSample_fr}     
    \includegraphics[width=0.8\cmsFigWidth]{Figures/Unfolding/MCTests/Mjj_ZZTo2e2m_PowMatrix_PowDistr_FullSample_fr}      
    \includegraphics[width=0.8\cmsFigWidth]{Figures/Unfolding/MCTests/Mjj_ZZTo4e_MadMatrix_MadDistr_HalfSample_fr}     
    \includegraphics[width=0.8\cmsFigWidth]{Figures/Unfolding/MCTests/Mjj_ZZTo4m_MadMatrix_MadDistr_HalfSample_fr}     
    \includegraphics[width=0.8\cmsFigWidth]{Figures/Unfolding/MCTests/Mjj_ZZTo2e2m_MadMatrix_MadDistr_HalfSample_fr}     
    \includegraphics[width=0.8\cmsFigWidth]{Figures/Unfolding/MCTests/Mjj_ZZTo4e_PowMatrix_PowDistr_HalfSample_fr}     
    \includegraphics[width=0.8\cmsFigWidth]{Figures/Unfolding/MCTests/Mjj_ZZTo4m_PowMatrix_PowDistr_HalfSample_fr}     
 \includegraphics[width=0.8\cmsFigWidth]{Figures/Unfolding/MCTests/Mjj_ZZTo2e2m_PowMatrix_PowDistr_HalfSample_fr}        
      \caption{Unfolding tests. From top to bottom: \texttt{MadGraph} matrix applied on \texttt{MadGraph} distribution using the full set, \texttt{Powheg} matrix applied on \texttt{Powheg} distribution using the full set,  \texttt{MadGraph} matrix applied on \texttt{MadGraph} distribution using the two different halves of the total sample set, \texttt{Powheg} matrix applied on \texttt{Powheg} distribution using the two different halves of the total sample set. Results are reported as a function of $m_{jj}$ for the $4e$ (left), $4\mu$ (center) and $2e2\mu$ (right) final states.}
    \label{fig:MCtest_Mjj1}
  \end{center}
\end{figure}

\begin{figure}[hbtp]
  \begin{center}
    \includegraphics[width=0.8\cmsFigWidth]{Figures/Unfolding/MCTests/Mjj_ZZTo4e_MadMatrix_PowDistr_FullSample_fr}     
    \includegraphics[width=0.8\cmsFigWidth]{Figures/Unfolding/MCTests/Mjj_ZZTo4m_MadMatrix_PowDistr_FullSample_fr}     
    \includegraphics[width=0.8\cmsFigWidth]{Figures/Unfolding/MCTests/Mjj_ZZTo2e2m_MadMatrix_PowDistr_FullSample_fr}
     \includegraphics[width=0.8\cmsFigWidth]{Figures/Unfolding/MCTests/Mjj_ZZTo4e_PowMatrix_MadDistr_FullSample_fr}     
    \includegraphics[width=0.8\cmsFigWidth]{Figures/Unfolding/MCTests/Mjj_ZZTo4m_PowMatrix_MadDistr_FullSample_fr}     
    \includegraphics[width=0.8\cmsFigWidth]{Figures/Unfolding/MCTests/Mjj_ZZTo2e2m_PowMatrix_MadDistr_FullSample_fr}  
 \caption{Unfolding tests. \texttt{MadGraph} matrix applied on \texttt{Powheg} distribution using the full set (top), \texttt{Powheg} matrix applied on \texttt{MadGraph} distribution using the full set (bottom). Results are reported as a function of $m_{jj}$ for the $4e$ (left), $4\mu$ (center) and $2e2\mu$ (right) final states.}
    \label{fig:MCtest_Mjj2}
  \end{center}
\end{figure}
\clearpage 
\begin{figure}[hbtp]
  \begin{center}
    \includegraphics[width=0.8\cmsFigWidth]{Figures/Unfolding/MCTests/CentralMjj_ZZTo4e_MadMatrix_MadDistr_FullSample_fr}     
    \includegraphics[width=0.8\cmsFigWidth]{Figures/Unfolding/MCTests/CentralMjj_ZZTo4m_MadMatrix_MadDistr_FullSample_fr}     
    \includegraphics[width=0.8\cmsFigWidth]{Figures/Unfolding/MCTests/CentralMjj_ZZTo2e2m_MadMatrix_MadDistr_FullSample_fr}
     \includegraphics[width=0.8\cmsFigWidth]{Figures/Unfolding/MCTests/CentralMjj_ZZTo4e_PowMatrix_PowDistr_FullSample_fr}     
    \includegraphics[width=0.8\cmsFigWidth]{Figures/Unfolding/MCTests/CentralMjj_ZZTo4m_PowMatrix_PowDistr_FullSample_fr}     
    \includegraphics[width=0.8\cmsFigWidth]{Figures/Unfolding/MCTests/CentralMjj_ZZTo2e2m_PowMatrix_PowDistr_FullSample_fr}      
    \includegraphics[width=0.8\cmsFigWidth]{Figures/Unfolding/MCTests/CentralMjj_ZZTo4e_MadMatrix_MadDistr_HalfSample_fr}     
    \includegraphics[width=0.8\cmsFigWidth]{Figures/Unfolding/MCTests/CentralMjj_ZZTo4m_MadMatrix_MadDistr_HalfSample_fr}     
    \includegraphics[width=0.8\cmsFigWidth]{Figures/Unfolding/MCTests/CentralMjj_ZZTo2e2m_MadMatrix_MadDistr_HalfSample_fr}     
    \includegraphics[width=0.8\cmsFigWidth]{Figures/Unfolding/MCTests/CentralMjj_ZZTo4e_PowMatrix_PowDistr_HalfSample_fr}     
    \includegraphics[width=0.8\cmsFigWidth]{Figures/Unfolding/MCTests/CentralMjj_ZZTo4m_PowMatrix_PowDistr_HalfSample_fr}     
 \includegraphics[width=0.8\cmsFigWidth]{Figures/Unfolding/MCTests/CentralMjj_ZZTo2e2m_PowMatrix_PowDistr_HalfSample_fr}        
      \caption{Unfolding tests. From top to bottom: \texttt{MadGraph} matrix applied on \texttt{MadGraph} distribution using the full set, \texttt{Powheg} matrix applied on \texttt{Powheg} distribution using the full set,  \texttt{MadGraph} matrix applied on \texttt{MadGraph} distribution using the two different halves of the total sample set, \texttt{Powheg} matrix applied on \texttt{Powheg} distribution using the two different halves of the total sample set. Results are reported as a function of $m_{jj}$ (with $|\eta^{jet}|<2.4$) for the $4e$ (left), $4\mu$ (center) and $2e2\mu$ (right) final states.}
    \label{fig:MCtest_CentralMjj1}
  \end{center}
\end{figure}

\begin{figure}[hbtp]
  \begin{center}
    \includegraphics[width=0.8\cmsFigWidth]{Figures/Unfolding/MCTests/CentralMjj_ZZTo4e_MadMatrix_PowDistr_FullSample_fr}     
    \includegraphics[width=0.8\cmsFigWidth]{Figures/Unfolding/MCTests/CentralMjj_ZZTo4m_MadMatrix_PowDistr_FullSample_fr}     
    \includegraphics[width=0.8\cmsFigWidth]{Figures/Unfolding/MCTests/CentralMjj_ZZTo2e2m_MadMatrix_PowDistr_FullSample_fr}
     \includegraphics[width=0.8\cmsFigWidth]{Figures/Unfolding/MCTests/CentralMjj_ZZTo4e_PowMatrix_MadDistr_FullSample_fr}     
    \includegraphics[width=0.8\cmsFigWidth]{Figures/Unfolding/MCTests/CentralMjj_ZZTo4m_PowMatrix_MadDistr_FullSample_fr}     
    \includegraphics[width=0.8\cmsFigWidth]{Figures/Unfolding/MCTests/CentralMjj_ZZTo2e2m_PowMatrix_MadDistr_FullSample_fr}  
 \caption{Unfolding tests. \texttt{MadGraph} matrix applied on \texttt{Powheg} distribution using the full set (top), \texttt{Powheg} matrix applied on \texttt{MadGraph} distribution using the full set (bottom). Results are reported as a function of $m_{jj}$ (with $|\eta^{jet}|<2.4$) for the $4e$ (left), $4\mu$ (center) and $2e2\mu$ (right) final states.}
    \label{fig:MCtest_CentralMjj2}
  \end{center}
\end{figure}
\clearpage 
\begin{figure}[hbtp]
  \begin{center}
    \includegraphics[width=0.8\cmsFigWidth]{Figures/Unfolding/MCTests/Deta_ZZTo4e_MadMatrix_MadDistr_FullSample_fr}     
    \includegraphics[width=0.8\cmsFigWidth]{Figures/Unfolding/MCTests/Deta_ZZTo4m_MadMatrix_MadDistr_FullSample_fr}     
    \includegraphics[width=0.8\cmsFigWidth]{Figures/Unfolding/MCTests/Deta_ZZTo2e2m_MadMatrix_MadDistr_FullSample_fr}
     \includegraphics[width=0.8\cmsFigWidth]{Figures/Unfolding/MCTests/Deta_ZZTo4e_PowMatrix_PowDistr_FullSample_fr}     
    \includegraphics[width=0.8\cmsFigWidth]{Figures/Unfolding/MCTests/Deta_ZZTo4m_PowMatrix_PowDistr_FullSample_fr}     
    \includegraphics[width=0.8\cmsFigWidth]{Figures/Unfolding/MCTests/Deta_ZZTo2e2m_PowMatrix_PowDistr_FullSample_fr}      
    \includegraphics[width=0.8\cmsFigWidth]{Figures/Unfolding/MCTests/Deta_ZZTo4e_MadMatrix_MadDistr_HalfSample_fr}     
    \includegraphics[width=0.8\cmsFigWidth]{Figures/Unfolding/MCTests/Deta_ZZTo4m_MadMatrix_MadDistr_HalfSample_fr}     
    \includegraphics[width=0.8\cmsFigWidth]{Figures/Unfolding/MCTests/Deta_ZZTo2e2m_MadMatrix_MadDistr_HalfSample_fr}     
    \includegraphics[width=0.8\cmsFigWidth]{Figures/Unfolding/MCTests/Deta_ZZTo4e_PowMatrix_PowDistr_HalfSample_fr}     
    \includegraphics[width=0.8\cmsFigWidth]{Figures/Unfolding/MCTests/Deta_ZZTo4m_PowMatrix_PowDistr_HalfSample_fr}     
 \includegraphics[width=0.8\cmsFigWidth]{Figures/Unfolding/MCTests/Deta_ZZTo2e2m_PowMatrix_PowDistr_HalfSample_fr}        
      \caption{Unfolding tests. From top to bottom: \texttt{MadGraph} matrix applied on \texttt{MadGraph} distribution using the full set, \texttt{Powheg} matrix applied on \texttt{Powheg} distribution using the full set,  \texttt{MadGraph} matrix applied on \texttt{MadGraph} distribution using the two different halves of the total sample set, \texttt{Powheg} matrix applied on \texttt{Powheg} distribution using the two different halves of the total sample set. Results are reported as a function of $\Delta\eta_{jj}$ for the $4e$ (left), $4\mu$ (center) and $2e2\mu$ (right) final states.}
    \label{fig:MCtest_Deta1}
  \end{center}
\end{figure}

\begin{figure}[hbtp]
  \begin{center}
    \includegraphics[width=0.8\cmsFigWidth]{Figures/Unfolding/MCTests/Deta_ZZTo4e_MadMatrix_PowDistr_FullSample_fr}     
    \includegraphics[width=0.8\cmsFigWidth]{Figures/Unfolding/MCTests/Deta_ZZTo4m_MadMatrix_PowDistr_FullSample_fr}     
    \includegraphics[width=0.8\cmsFigWidth]{Figures/Unfolding/MCTests/Deta_ZZTo2e2m_MadMatrix_PowDistr_FullSample_fr}
     \includegraphics[width=0.8\cmsFigWidth]{Figures/Unfolding/MCTests/Deta_ZZTo4e_PowMatrix_MadDistr_FullSample_fr}     
    \includegraphics[width=0.8\cmsFigWidth]{Figures/Unfolding/MCTests/Deta_ZZTo4m_PowMatrix_MadDistr_FullSample_fr}     
    \includegraphics[width=0.8\cmsFigWidth]{Figures/Unfolding/MCTests/Deta_ZZTo2e2m_PowMatrix_MadDistr_FullSample_fr}  
 \caption{Unfolding tests. \texttt{MadGraph} matrix applied on \texttt{Powheg} distribution using the full set (top), \texttt{Powheg} matrix applied on \texttt{MadGraph} distribution using the full set (bottom). Results are reported as a function of $\Delta\eta_{jj}$ for the $4e$ (left), $4\mu$ (center) and $2e2\mu$ (right) final states.}
    \label{fig:MCtest_Deta2}
  \end{center}
\end{figure}
\clearpage 
\begin{figure}[hbtp]
  \begin{center}
    \includegraphics[width=0.8\cmsFigWidth]{Figures/Unfolding/MCTests/CentralDeta_ZZTo4e_MadMatrix_MadDistr_FullSample_fr}     
    \includegraphics[width=0.8\cmsFigWidth]{Figures/Unfolding/MCTests/CentralDeta_ZZTo4m_MadMatrix_MadDistr_FullSample_fr}     
    \includegraphics[width=0.8\cmsFigWidth]{Figures/Unfolding/MCTests/CentralDeta_ZZTo2e2m_MadMatrix_MadDistr_FullSample_fr}
     \includegraphics[width=0.8\cmsFigWidth]{Figures/Unfolding/MCTests/CentralDeta_ZZTo4e_PowMatrix_PowDistr_FullSample_fr}     
    \includegraphics[width=0.8\cmsFigWidth]{Figures/Unfolding/MCTests/CentralDeta_ZZTo4m_PowMatrix_PowDistr_FullSample_fr}     
    \includegraphics[width=0.8\cmsFigWidth]{Figures/Unfolding/MCTests/CentralDeta_ZZTo2e2m_PowMatrix_PowDistr_FullSample_fr}      
    \includegraphics[width=0.8\cmsFigWidth]{Figures/Unfolding/MCTests/CentralDeta_ZZTo4e_MadMatrix_MadDistr_HalfSample_fr}     
    \includegraphics[width=0.8\cmsFigWidth]{Figures/Unfolding/MCTests/CentralDeta_ZZTo4m_MadMatrix_MadDistr_HalfSample_fr}     
    \includegraphics[width=0.8\cmsFigWidth]{Figures/Unfolding/MCTests/CentralDeta_ZZTo2e2m_MadMatrix_MadDistr_HalfSample_fr}     
    \includegraphics[width=0.8\cmsFigWidth]{Figures/Unfolding/MCTests/CentralDeta_ZZTo4e_PowMatrix_PowDistr_HalfSample_fr}     
    \includegraphics[width=0.8\cmsFigWidth]{Figures/Unfolding/MCTests/CentralDeta_ZZTo4m_PowMatrix_PowDistr_HalfSample_fr}     
 \includegraphics[width=0.8\cmsFigWidth]{Figures/Unfolding/MCTests/CentralDeta_ZZTo2e2m_PowMatrix_PowDistr_HalfSample_fr}        
      \caption{Unfolding tests. From top to bottom: \texttt{MadGraph} matrix applied on \texttt{MadGraph} distribution using the full set, \texttt{Powheg} matrix applied on \texttt{Powheg} distribution using the full set,  \texttt{MadGraph} matrix applied on \texttt{MadGraph} distribution using the two different halves of the total sample set, \texttt{Powheg} matrix applied on \texttt{Powheg} distribution using the two different halves of the total sample set. Results are reported as a function of $\Delta\eta_{jj}$ (with $|\eta^{jet}|<2.4$) for the $4e$ (left), $4\mu$ (center) and $2e2\mu$ (right) final states.}
    \label{fig:MCtest_CentralDeta1}
  \end{center}
\end{figure}

\begin{figure}[hbtp]
  \begin{center}
    \includegraphics[width=0.8\cmsFigWidth]{Figures/Unfolding/MCTests/CentralDeta_ZZTo4e_MadMatrix_PowDistr_FullSample_fr}     
    \includegraphics[width=0.8\cmsFigWidth]{Figures/Unfolding/MCTests/CentralDeta_ZZTo4m_MadMatrix_PowDistr_FullSample_fr}     
    \includegraphics[width=0.8\cmsFigWidth]{Figures/Unfolding/MCTests/CentralDeta_ZZTo2e2m_MadMatrix_PowDistr_FullSample_fr}
     \includegraphics[width=0.8\cmsFigWidth]{Figures/Unfolding/MCTests/CentralDeta_ZZTo4e_PowMatrix_MadDistr_FullSample_fr}     
    \includegraphics[width=0.8\cmsFigWidth]{Figures/Unfolding/MCTests/CentralDeta_ZZTo4m_PowMatrix_MadDistr_FullSample_fr}     
    \includegraphics[width=0.8\cmsFigWidth]{Figures/Unfolding/MCTests/CentralDeta_ZZTo2e2m_PowMatrix_MadDistr_FullSample_fr}  
 \caption{Unfolding tests. \texttt{MadGraph} matrix applied on \texttt{Powheg} distribution using the full set (top), \texttt{Powheg} matrix applied on \texttt{MadGraph} distribution using the full set (bottom). Results are reported as a function of $\Delta\eta_{jj}$ (with $|\eta^{jet}|<2.4$) for the $4e$ (left), $4\mu$ (center) and $2e2\mu$ (right) final states.}
    \label{fig:MCtest_CentralDeta2}
  \end{center}
\end{figure}
\clearpage 
\begin{figure}[hbtp]
  \begin{center}
    \includegraphics[width=0.8\cmsFigWidth]{Figures/Unfolding/MCTests/PtJet1_ZZTo4e_MadMatrix_MadDistr_FullSample_fr}     
    \includegraphics[width=0.8\cmsFigWidth]{Figures/Unfolding/MCTests/PtJet1_ZZTo4m_MadMatrix_MadDistr_FullSample_fr}     
    \includegraphics[width=0.8\cmsFigWidth]{Figures/Unfolding/MCTests/PtJet1_ZZTo2e2m_MadMatrix_MadDistr_FullSample_fr}
     \includegraphics[width=0.8\cmsFigWidth]{Figures/Unfolding/MCTests/PtJet1_ZZTo4e_PowMatrix_PowDistr_FullSample_fr}     
    \includegraphics[width=0.8\cmsFigWidth]{Figures/Unfolding/MCTests/PtJet1_ZZTo4m_PowMatrix_PowDistr_FullSample_fr}     
    \includegraphics[width=0.8\cmsFigWidth]{Figures/Unfolding/MCTests/PtJet1_ZZTo2e2m_PowMatrix_PowDistr_FullSample_fr}      
    \includegraphics[width=0.8\cmsFigWidth]{Figures/Unfolding/MCTests/PtJet1_ZZTo4e_MadMatrix_MadDistr_HalfSample_fr}     
    \includegraphics[width=0.8\cmsFigWidth]{Figures/Unfolding/MCTests/PtJet1_ZZTo4m_MadMatrix_MadDistr_HalfSample_fr}     
    \includegraphics[width=0.8\cmsFigWidth]{Figures/Unfolding/MCTests/PtJet1_ZZTo2e2m_MadMatrix_MadDistr_HalfSample_fr}     
    \includegraphics[width=0.8\cmsFigWidth]{Figures/Unfolding/MCTests/PtJet1_ZZTo4e_PowMatrix_PowDistr_HalfSample_fr}     
    \includegraphics[width=0.8\cmsFigWidth]{Figures/Unfolding/MCTests/PtJet1_ZZTo4m_PowMatrix_PowDistr_HalfSample_fr}     
 \includegraphics[width=0.8\cmsFigWidth]{Figures/Unfolding/MCTests/PtJet1_ZZTo2e2m_PowMatrix_PowDistr_HalfSample_fr}        
      \caption{Unfolding tests. From top to bottom: \texttt{MadGraph} matrix applied on \texttt{MadGraph} distribution using the full set, \texttt{Powheg} matrix applied on \texttt{Powheg} distribution using the full set,  \texttt{MadGraph} matrix applied on \texttt{MadGraph} distribution using the two different halves of the total sample set, \texttt{Powheg} matrix applied on \texttt{Powheg} distribution using the two different halves of the total sample set. Results are reported as a function of the $p_T$ of the leading jet, for the $4e$ (left), $4\mu$ (center) and $2e2\mu$ (right) final states.}
    \label{fig:MCtest_PtJet11}
  \end{center}
\end{figure}

\begin{figure}[hbtp]
  \begin{center}
    \includegraphics[width=0.8\cmsFigWidth]{Figures/Unfolding/MCTests/PtJet1_ZZTo4e_MadMatrix_PowDistr_FullSample_fr}     
    \includegraphics[width=0.8\cmsFigWidth]{Figures/Unfolding/MCTests/PtJet1_ZZTo4m_MadMatrix_PowDistr_FullSample_fr}     
    \includegraphics[width=0.8\cmsFigWidth]{Figures/Unfolding/MCTests/PtJet1_ZZTo2e2m_MadMatrix_PowDistr_FullSample_fr}
     \includegraphics[width=0.8\cmsFigWidth]{Figures/Unfolding/MCTests/PtJet1_ZZTo4e_PowMatrix_MadDistr_FullSample_fr}     
    \includegraphics[width=0.8\cmsFigWidth]{Figures/Unfolding/MCTests/PtJet1_ZZTo4m_PowMatrix_MadDistr_FullSample_fr}     
    \includegraphics[width=0.8\cmsFigWidth]{Figures/Unfolding/MCTests/PtJet1_ZZTo2e2m_PowMatrix_MadDistr_FullSample_fr}  
 \caption{Unfolding tests. \texttt{MadGraph} matrix applied on \texttt{Powheg} distribution using the full set (top), \texttt{Powheg} matrix applied on \texttt{MadGraph} distribution using the full set (bottom). Results are reported as a function of  the $p_T$ of the leading jet, for the $4e$ (left), $4\mu$ (center) and $2e2\mu$ (right) final states.}
    \label{fig:MCtest_PtJet12}
  \end{center}
\end{figure}
\clearpage
\begin{figure}[hbtp]
  \begin{center}
    \includegraphics[width=0.8\cmsFigWidth]{Figures/Unfolding/MCTests/PtJet2_ZZTo4e_MadMatrix_MadDistr_FullSample_fr}     
    \includegraphics[width=0.8\cmsFigWidth]{Figures/Unfolding/MCTests/PtJet2_ZZTo4m_MadMatrix_MadDistr_FullSample_fr}     
    \includegraphics[width=0.8\cmsFigWidth]{Figures/Unfolding/MCTests/PtJet2_ZZTo2e2m_MadMatrix_MadDistr_FullSample_fr}
     \includegraphics[width=0.8\cmsFigWidth]{Figures/Unfolding/MCTests/PtJet2_ZZTo4e_PowMatrix_PowDistr_FullSample_fr}     
    \includegraphics[width=0.8\cmsFigWidth]{Figures/Unfolding/MCTests/PtJet2_ZZTo4m_PowMatrix_PowDistr_FullSample_fr}     
    \includegraphics[width=0.8\cmsFigWidth]{Figures/Unfolding/MCTests/PtJet2_ZZTo2e2m_PowMatrix_PowDistr_FullSample_fr}      
    \includegraphics[width=0.8\cmsFigWidth]{Figures/Unfolding/MCTests/PtJet2_ZZTo4e_MadMatrix_MadDistr_HalfSample_fr}     
    \includegraphics[width=0.8\cmsFigWidth]{Figures/Unfolding/MCTests/PtJet2_ZZTo4m_MadMatrix_MadDistr_HalfSample_fr}     
    \includegraphics[width=0.8\cmsFigWidth]{Figures/Unfolding/MCTests/PtJet2_ZZTo2e2m_MadMatrix_MadDistr_HalfSample_fr}     
    \includegraphics[width=0.8\cmsFigWidth]{Figures/Unfolding/MCTests/PtJet2_ZZTo4e_PowMatrix_PowDistr_HalfSample_fr}     
    \includegraphics[width=0.8\cmsFigWidth]{Figures/Unfolding/MCTests/PtJet2_ZZTo4m_PowMatrix_PowDistr_HalfSample_fr}     
 \includegraphics[width=0.8\cmsFigWidth]{Figures/Unfolding/MCTests/PtJet2_ZZTo2e2m_PowMatrix_PowDistr_HalfSample_fr}        
      \caption{Unfolding tests. From top to bottom: \texttt{MadGraph} matrix applied on \texttt{MadGraph} distribution using the full set, \texttt{Powheg} matrix applied on \texttt{Powheg} distribution using the full set,  \texttt{MadGraph} matrix applied on \texttt{MadGraph} distribution using the two different halves of the total sample set, \texttt{Powheg} matrix applied on \texttt{Powheg} distribution using the two different halves of the total sample set. Results are reported as a function of the $p_T$ of the sub-leading jet, for the $4e$ (left), $4\mu$ (center) and $2e2\mu$ (right) final states.}
    \label{fig:MCtest_PtJet21}
  \end{center}
\end{figure}

\begin{figure}[hbtp]
  \begin{center}
    \includegraphics[width=0.8\cmsFigWidth]{Figures/Unfolding/MCTests/PtJet2_ZZTo4e_MadMatrix_PowDistr_FullSample_fr}     
    \includegraphics[width=0.8\cmsFigWidth]{Figures/Unfolding/MCTests/PtJet2_ZZTo4m_MadMatrix_PowDistr_FullSample_fr}     
    \includegraphics[width=0.8\cmsFigWidth]{Figures/Unfolding/MCTests/PtJet2_ZZTo2e2m_MadMatrix_PowDistr_FullSample_fr}
     \includegraphics[width=0.8\cmsFigWidth]{Figures/Unfolding/MCTests/PtJet2_ZZTo4e_PowMatrix_MadDistr_FullSample_fr}     
    \includegraphics[width=0.8\cmsFigWidth]{Figures/Unfolding/MCTests/PtJet2_ZZTo4m_PowMatrix_MadDistr_FullSample_fr}     
    \includegraphics[width=0.8\cmsFigWidth]{Figures/Unfolding/MCTests/PtJet2_ZZTo2e2m_PowMatrix_MadDistr_FullSample_fr}  
 \caption{Unfolding tests. \texttt{MadGraph} matrix applied on \texttt{Powheg} distribution using the full set (top), \texttt{Powheg} matrix applied on \texttt{MadGraph} distribution using the full set (bottom). Results are reported as a function of  the $p_T$ of the sub-leading jet, for the $4e$ (left), $4\mu$ (center) and $2e2\mu$ (right) final states.}
    \label{fig:MCtest_PtJet22}
  \end{center}
\end{figure}
\clearpage
\begin{figure}[hbtp]
  \begin{center}
    \includegraphics[width=0.8\cmsFigWidth]{Figures/Unfolding/MCTests/EtaJet1_ZZTo4e_MadMatrix_MadDistr_FullSample_fr}     
    \includegraphics[width=0.8\cmsFigWidth]{Figures/Unfolding/MCTests/EtaJet1_ZZTo4m_MadMatrix_MadDistr_FullSample_fr}     
    \includegraphics[width=0.8\cmsFigWidth]{Figures/Unfolding/MCTests/EtaJet1_ZZTo2e2m_MadMatrix_MadDistr_FullSample_fr}
     \includegraphics[width=0.8\cmsFigWidth]{Figures/Unfolding/MCTests/EtaJet1_ZZTo4e_PowMatrix_PowDistr_FullSample_fr}     
    \includegraphics[width=0.8\cmsFigWidth]{Figures/Unfolding/MCTests/EtaJet1_ZZTo4m_PowMatrix_PowDistr_FullSample_fr}     
    \includegraphics[width=0.8\cmsFigWidth]{Figures/Unfolding/MCTests/EtaJet1_ZZTo2e2m_PowMatrix_PowDistr_FullSample_fr}      
    \includegraphics[width=0.8\cmsFigWidth]{Figures/Unfolding/MCTests/EtaJet1_ZZTo4e_MadMatrix_MadDistr_HalfSample_fr}     
    \includegraphics[width=0.8\cmsFigWidth]{Figures/Unfolding/MCTests/EtaJet1_ZZTo4m_MadMatrix_MadDistr_HalfSample_fr}     
    \includegraphics[width=0.8\cmsFigWidth]{Figures/Unfolding/MCTests/EtaJet1_ZZTo2e2m_MadMatrix_MadDistr_HalfSample_fr}     
    \includegraphics[width=0.8\cmsFigWidth]{Figures/Unfolding/MCTests/EtaJet1_ZZTo4e_PowMatrix_PowDistr_HalfSample_fr}     
    \includegraphics[width=0.8\cmsFigWidth]{Figures/Unfolding/MCTests/EtaJet1_ZZTo4m_PowMatrix_PowDistr_HalfSample_fr}     
 \includegraphics[width=0.8\cmsFigWidth]{Figures/Unfolding/MCTests/EtaJet1_ZZTo2e2m_PowMatrix_PowDistr_HalfSample_fr}        
      \caption{Unfolding tests. From top to bottom: \texttt{MadGraph} matrix applied on \texttt{MadGraph} distribution using the full set, \texttt{Powheg} matrix applied on \texttt{Powheg} distribution using the full set,  \texttt{MadGraph} matrix applied on \texttt{MadGraph} distribution using the two different halves of the total sample set, \texttt{Powheg} matrix applied on \texttt{Powheg} distribution using the two different halves of the total sample set. Results are reported as a function of the $\eta$ of the leading jet, for the $4e$ (left), $4\mu$ (center) and $2e2\mu$ (right) final states.}
    \label{fig:MCtest_EtaJet11}
  \end{center}
\end{figure}

\begin{figure}[hbtp]
  \begin{center}
    \includegraphics[width=0.8\cmsFigWidth]{Figures/Unfolding/MCTests/EtaJet1_ZZTo4e_MadMatrix_PowDistr_FullSample_fr}     
    \includegraphics[width=0.8\cmsFigWidth]{Figures/Unfolding/MCTests/EtaJet1_ZZTo4m_MadMatrix_PowDistr_FullSample_fr}     
    \includegraphics[width=0.8\cmsFigWidth]{Figures/Unfolding/MCTests/EtaJet1_ZZTo2e2m_MadMatrix_PowDistr_FullSample_fr}
     \includegraphics[width=0.8\cmsFigWidth]{Figures/Unfolding/MCTests/EtaJet1_ZZTo4e_PowMatrix_MadDistr_FullSample_fr}     
    \includegraphics[width=0.8\cmsFigWidth]{Figures/Unfolding/MCTests/EtaJet1_ZZTo4m_PowMatrix_MadDistr_FullSample_fr}     
    \includegraphics[width=0.8\cmsFigWidth]{Figures/Unfolding/MCTests/EtaJet1_ZZTo2e2m_PowMatrix_MadDistr_FullSample_fr}  
 \caption{Unfolding tests. \texttt{MadGraph} matrix applied on \texttt{Powheg} distribution using the full set (top), \texttt{Powheg} matrix applied on \texttt{MadGraph} distribution using the full set (bottom). Results are reported as a function of  the $\eta$ of the leading jet, for the $4e$ (left), $4\mu$ (center) and $2e2\mu$ (right) final states.}
    \label{fig:MCtest_EtaJet12}
  \end{center}
\end{figure}
\clearpage
\begin{figure}[hbtp]
  \begin{center}
    \includegraphics[width=0.8\cmsFigWidth]{Figures/Unfolding/MCTests/EtaJet2_ZZTo4e_MadMatrix_MadDistr_FullSample_fr}     
    \includegraphics[width=0.8\cmsFigWidth]{Figures/Unfolding/MCTests/EtaJet2_ZZTo4m_MadMatrix_MadDistr_FullSample_fr}     
    \includegraphics[width=0.8\cmsFigWidth]{Figures/Unfolding/MCTests/EtaJet2_ZZTo2e2m_MadMatrix_MadDistr_FullSample_fr}
     \includegraphics[width=0.8\cmsFigWidth]{Figures/Unfolding/MCTests/EtaJet2_ZZTo4e_PowMatrix_PowDistr_FullSample_fr}     
    \includegraphics[width=0.8\cmsFigWidth]{Figures/Unfolding/MCTests/EtaJet2_ZZTo4m_PowMatrix_PowDistr_FullSample_fr}     
    \includegraphics[width=0.8\cmsFigWidth]{Figures/Unfolding/MCTests/EtaJet2_ZZTo2e2m_PowMatrix_PowDistr_FullSample_fr}      
    \includegraphics[width=0.8\cmsFigWidth]{Figures/Unfolding/MCTests/EtaJet2_ZZTo4e_MadMatrix_MadDistr_HalfSample_fr}     
    \includegraphics[width=0.8\cmsFigWidth]{Figures/Unfolding/MCTests/EtaJet2_ZZTo4m_MadMatrix_MadDistr_HalfSample_fr}     
    \includegraphics[width=0.8\cmsFigWidth]{Figures/Unfolding/MCTests/EtaJet2_ZZTo2e2m_MadMatrix_MadDistr_HalfSample_fr}     
    \includegraphics[width=0.8\cmsFigWidth]{Figures/Unfolding/MCTests/EtaJet2_ZZTo4e_PowMatrix_PowDistr_HalfSample_fr}     
    \includegraphics[width=0.8\cmsFigWidth]{Figures/Unfolding/MCTests/EtaJet2_ZZTo4m_PowMatrix_PowDistr_HalfSample_fr}     
 \includegraphics[width=0.8\cmsFigWidth]{Figures/Unfolding/MCTests/EtaJet2_ZZTo2e2m_PowMatrix_PowDistr_HalfSample_fr}        
      \caption{Unfolding tests. From top to bottom: \texttt{MadGraph} matrix applied on \texttt{MadGraph} distribution using the full set, \texttt{Powheg} matrix applied on \texttt{Powheg} distribution using the full set,  \texttt{MadGraph} matrix applied on \texttt{MadGraph} distribution using the two different halves of the total sample set, \texttt{Powheg} matrix applied on \texttt{Powheg} distribution using the two different halves of the total sample set. Results are reported as a function of the $\eta$ of the sub-leading jet, for the $4e$ (left), $4\mu$ (center) and $2e2\mu$ (right) final states.}
    \label{fig:MCtest_EtaJet21}
  \end{center}
\end{figure}

\begin{figure}[hbtp]
  \begin{center}
    \includegraphics[width=0.8\cmsFigWidth]{Figures/Unfolding/MCTests/EtaJet2_ZZTo4e_MadMatrix_PowDistr_FullSample_fr}     
    \includegraphics[width=0.8\cmsFigWidth]{Figures/Unfolding/MCTests/EtaJet2_ZZTo4m_MadMatrix_PowDistr_FullSample_fr}     
    \includegraphics[width=0.8\cmsFigWidth]{Figures/Unfolding/MCTests/EtaJet2_ZZTo2e2m_MadMatrix_PowDistr_FullSample_fr}
     \includegraphics[width=0.8\cmsFigWidth]{Figures/Unfolding/MCTests/EtaJet2_ZZTo4e_PowMatrix_MadDistr_FullSample_fr}     
    \includegraphics[width=0.8\cmsFigWidth]{Figures/Unfolding/MCTests/EtaJet2_ZZTo4m_PowMatrix_MadDistr_FullSample_fr}     
    \includegraphics[width=0.8\cmsFigWidth]{Figures/Unfolding/MCTests/EtaJet2_ZZTo2e2m_PowMatrix_MadDistr_FullSample_fr}  
 \caption{Unfolding tests. \texttt{MadGraph} matrix applied on \texttt{Powheg} distribution using the full set (top), \texttt{Powheg} matrix applied on \texttt{MadGraph} distribution using the full set (bottom). Results are reported as a function of  the $\eta$ of the sub-leading jet, for the $4e$ (left), $4\mu$ (center) and $2e2\mu$ (right) final states.}
    \label{fig:MCtest_EtaJet22}
  \end{center}
\end{figure}
\clearpage

%% %%%%%%%%%%%%%%%%%%%%%%%%%%%%%%%%%%%%%%%%%%%%%%%%%%%%FullMad%%%%%%%%%%%%%%%%%%%%%%%%%%%%%%%%%%%%%%%%%%%%%%%%
%% \begin{figure}[hbtp]
%%   \begin{center}
%%     \includegraphics[width=\cmsFigWidth]{Figures/Unfolding/MCTests/Mass_ZZTo4e_MadMatrix_MadDistr_FullSample}     
%%     \includegraphics[width=\cmsFigWidth]{Figures/Unfolding/MCTests/Deta_ZZTo4e_MadMatrix_MadDistr_FullSample}   
%%     \includegraphics[width=\cmsFigWidth]{Figures/Unfolding/MCTests/Jets_ZZTo4e_MadMatrix_MadDistr_FullSample}
%%     \includegraphics[width=\cmsFigWidth]{Figures/Unfolding/MCTests/CentralJets_ZZTo4e_MadMatrix_MadDistr_FullSample}
%%     \includegraphics[width=\cmsFigWidth]{Figures/Unfolding/MCTests/Mjj_ZZTo4e_MadMatrix_MadDistr_FullSample}
%%     \includegraphics[width=\cmsFigWidth]{Figures/Unfolding/MCTests/CentralMjj_ZZTo4e_MadMatrix_MadDistr_FullSample}
%%     \includegraphics[width=\cmsFigWidth]{Figures/Unfolding/MCTests/PtJet1_ZZTo4e_MadMatrix_MadDistr_FullSample}        
%%     \includegraphics[width=\cmsFigWidth]{Figures/Unfolding/MCTests/PtJet2_ZZTo4e_MadMatrix_MadDistr_FullSample}        
%%       \caption{Unfolding test: \texttt{MadGraph} matrix applied on \texttt{MadGraph} distribution, using the full set. Results are reported as a function of 
%% the 4-lepton system (top left), the $\Delta\eta$ between the two most energetic jets (top right), the number of jets (second line left) and central jets (second line right) in the event, the invariant mass of the two most energetic jets (third line left) and central jets (third line right), the transverse momentum of the leading (bottom left) and sub-leading (botton right) jets, for the $4e$ final state.}
%%     \label{fig:FullMad_4e}
%%   \end{center}
%% \end{figure}
%% \begin{figure}[hbtp]
%%   \begin{center}
%%     \includegraphics[width=\cmsFigWidth]{Figures/Unfolding/MCTests/Mass_ZZTo4m_MadMatrix_MadDistr_FullSample}     
%%     \includegraphics[width=\cmsFigWidth]{Figures/Unfolding/MCTests/Deta_ZZTo4m_MadMatrix_MadDistr_FullSample}   
%%     \includegraphics[width=\cmsFigWidth]{Figures/Unfolding/MCTests/Jets_ZZTo4m_MadMatrix_MadDistr_FullSample}
%%     \includegraphics[width=\cmsFigWidth]{Figures/Unfolding/MCTests/CentralJets_ZZTo4m_MadMatrix_MadDistr_FullSample}
%%     \includegraphics[width=\cmsFigWidth]{Figures/Unfolding/MCTests/Mjj_ZZTo4m_MadMatrix_MadDistr_FullSample}
%%     \includegraphics[width=\cmsFigWidth]{Figures/Unfolding/MCTests/CentralMjj_ZZTo4m_MadMatrix_MadDistr_FullSample}
%%     \includegraphics[width=\cmsFigWidth]{Figures/Unfolding/MCTests/PtJet1_ZZTo4m_MadMatrix_MadDistr_FullSample}
%%     \includegraphics[width=\cmsFigWidth]{Figures/Unfolding/MCTests/PtJet2_ZZTo4m_MadMatrix_MadDistr_FullSample}
%%       \caption{Unfolding test: \texttt{MadGraph} matrix applied on \texttt{MadGraph} distribution, using the full set. Results are reported as a function of 
%% the 4-lepton system (top left), the $\Delta\eta$ between the two most energetic jets (top right), the number of jets (second line left) and central jets (second line right) in the event, the invariant mass of the two most energetic jets (third line left) and central jets (third line right), the transverse momentum of the leading (bottom left) and sub-leading (botton right) jets, for the $4\mu$ final state.}
%%     \label{fig:FullMad_4m}
%%   \end{center}
%% \end{figure}
%% \begin{figure}[hbtp]
%%   \begin{center}
%%     \includegraphics[width=\cmsFigWidth]{Figures/Unfolding/MCTests/Mass_ZZTo2e2m_MadMatrix_MadDistr_FullSample}     
%%     \includegraphics[width=\cmsFigWidth]{Figures/Unfolding/MCTests/Deta_ZZTo2e2m_MadMatrix_MadDistr_FullSample}   
%%     \includegraphics[width=\cmsFigWidth]{Figures/Unfolding/MCTests/Jets_ZZTo2e2m_MadMatrix_MadDistr_FullSample}
%%     \includegraphics[width=\cmsFigWidth]{Figures/Unfolding/MCTests/CentralJets_ZZTo2e2m_MadMatrix_MadDistr_FullSample}
%%     \includegraphics[width=\cmsFigWidth]{Figures/Unfolding/MCTests/Mjj_ZZTo2e2m_MadMatrix_MadDistr_FullSample}
%%     \includegraphics[width=\cmsFigWidth]{Figures/Unfolding/MCTests/CentralMjj_ZZTo2e2m_MadMatrix_MadDistr_FullSample}
%%     \includegraphics[width=\cmsFigWidth]{Figures/Unfolding/MCTests/PtJet1_ZZTo2e2m_MadMatrix_MadDistr_FullSample}
%%     \includegraphics[width=\cmsFigWidth]{Figures/Unfolding/MCTests/PtJet2_ZZTo2e2m_MadMatrix_MadDistr_FullSample}
%%       \caption{Unfolding test: \texttt{MadGraph} matrix applied on \texttt{MadGraph} distribution, using the full set. Results are reported as a function of 
%% the 4-lepton system (top left), the $\Delta\eta$ between the two most energetic jets (top right), the number of jets (second line left) and central jets (second line right) in the event, the invariant mass of the two most energetic jets (third line left) and central jets (third line right), the transverse momentum of the leading (bottom left) and sub-leading (botton right) jets, for the $2e2\mu$ final state.}
%%     \label{fig:FullMad_2e2m}
%%   \end{center}
%% \end{figure}

%% %%%%%%%%%%%%%%%%%%%%%%%%%%%%%%%%%%%%%%%%%%%%%%%%%%%%FullPow%%%%%%%%%%%%%%%%%%%%%%%%%%%%%%%%%%%%%%%%%%%%%%%%
%% \begin{figure}[hbtp]
%%   \begin{center}
%%     \includegraphics[width=\cmsFigWidth]{Figures/Unfolding/MCTests/Mass_ZZTo4e_PowMatrix_PowDistr_FullSample}     
%%     \includegraphics[width=\cmsFigWidth]{Figures/Unfolding/MCTests/Deta_ZZTo4e_PowMatrix_PowDistr_FullSample}   
%%     \includegraphics[width=\cmsFigWidth]{Figures/Unfolding/MCTests/Jets_ZZTo4e_PowMatrix_PowDistr_FullSample}
%%     \includegraphics[width=\cmsFigWidth]{Figures/Unfolding/MCTests/CentralJets_ZZTo4e_PowMatrix_PowDistr_FullSample}
%%     \includegraphics[width=\cmsFigWidth]{Figures/Unfolding/MCTests/Mjj_ZZTo4e_PowMatrix_PowDistr_FullSample}
%%     \includegraphics[width=\cmsFigWidth]{Figures/Unfolding/MCTests/CentralMjj_ZZTo4e_PowMatrix_PowDistr_FullSample}
%%     \includegraphics[width=\cmsFigWidth]{Figures/Unfolding/MCTests/PtJet1_ZZTo4e_PowMatrix_PowDistr_FullSample}
%%     \includegraphics[width=\cmsFigWidth]{Figures/Unfolding/MCTests/PtJet2_ZZTo4e_PowMatrix_PowDistr_FullSample}
%%       \caption{Unfolding test: \texttt{Powheg} matrix applied on \texttt{Powheg} distribution, using the full set. Results are reported as a function of 
%% the 4-lepton system (top left), the $\Delta\eta$ between the two most energetic jets (top right), the number of jets (second line left) and central jets (second line right) in the event, the invariant mass of the two most energetic jets (third line left) and central jets (third line right), the transverse momentum of the leading (bottom left) and sub-leading (botton right) jets, for the $4e$ final state.}
%%     \label{fig:FullPow_4e}
%%   \end{center}
%% \end{figure}
%% \begin{figure}[hbtp]
%%   \begin{center}
%%     \includegraphics[width=\cmsFigWidth]{Figures/Unfolding/MCTests/Mass_ZZTo4m_PowMatrix_PowDistr_FullSample}     
%%     \includegraphics[width=\cmsFigWidth]{Figures/Unfolding/MCTests/Deta_ZZTo4m_PowMatrix_PowDistr_FullSample}   
%%     \includegraphics[width=\cmsFigWidth]{Figures/Unfolding/MCTests/Jets_ZZTo4m_PowMatrix_PowDistr_FullSample}
%%     \includegraphics[width=\cmsFigWidth]{Figures/Unfolding/MCTests/CentralJets_ZZTo4m_PowMatrix_PowDistr_FullSample}
%%     \includegraphics[width=\cmsFigWidth]{Figures/Unfolding/MCTests/Mjj_ZZTo4m_PowMatrix_PowDistr_FullSample}
%%     \includegraphics[width=\cmsFigWidth]{Figures/Unfolding/MCTests/CentralMjj_ZZTo4m_PowMatrix_PowDistr_FullSample}
%%     \includegraphics[width=\cmsFigWidth]{Figures/Unfolding/MCTests/PtJet1_ZZTo4m_PowMatrix_PowDistr_FullSample}
%%     \includegraphics[width=\cmsFigWidth]{Figures/Unfolding/MCTests/PtJet2_ZZTo4m_PowMatrix_PowDistr_FullSample}
%%       \caption{Unfolding test: \texttt{Powheg} matrix applied on \texttt{Powheg} distribution, using the full set. Results are reported as a function of 
%% the 4-lepton system (top left), the $\Delta\eta$ between the two most energetic jets (top right), the number of jets (second line left) and central jets (second line right) in the event, the invariant mass of the two most energetic jets (third line left) and  central jets (third line right), the transverse momentum of the leading (bottom left) and sub-leading (botton right) jets, for the $4\mu$ final state.}
%%     \label{fig:FullPow_4m}
%%   \end{center}
%% \end{figure}
%% \begin{figure}[hbtp]
%%   \begin{center}
%%     \includegraphics[width=\cmsFigWidth]{Figures/Unfolding/MCTests/Mass_ZZTo2e2m_PowMatrix_PowDistr_FullSample}     
%%     \includegraphics[width=\cmsFigWidth]{Figures/Unfolding/MCTests/Deta_ZZTo2e2m_PowMatrix_PowDistr_FullSample}   
%%     \includegraphics[width=\cmsFigWidth]{Figures/Unfolding/MCTests/Jets_ZZTo2e2m_PowMatrix_PowDistr_FullSample}
%%     \includegraphics[width=\cmsFigWidth]{Figures/Unfolding/MCTests/CentralJets_ZZTo2e2m_PowMatrix_PowDistr_FullSample}
%%     \includegraphics[width=\cmsFigWidth]{Figures/Unfolding/MCTests/Mjj_ZZTo2e2m_PowMatrix_PowDistr_FullSample}
%%     \includegraphics[width=\cmsFigWidth]{Figures/Unfolding/MCTests/CentralMjj_ZZTo2e2m_PowMatrix_PowDistr_FullSample}
%%     \includegraphics[width=\cmsFigWidth]{Figures/Unfolding/MCTests/PtJet1_ZZTo2e2m_PowMatrix_PowDistr_FullSample}
%%     \includegraphics[width=\cmsFigWidth]{Figures/Unfolding/MCTests/PtJet2_ZZTo2e2m_PowMatrix_PowDistr_FullSample}
%%       \caption{Unfolding test: \texttt{Powheg} matrix applied on \texttt{Powheg} distribution, using the full set. Results are reported as a function of 
%% the 4-lepton system (top left), the $\Delta\eta$ between the two most energetic jets (top right), the number of jets (second line left) and central jets (second line right) in the event, the invariant mass of the two most energetic jets (third line left) and  central jets (third line right), the transverse momentum of the leading (bottom left) and sub-leading (botton right) jets, for the $2e2\mu$ final state. }
%%     \label{fig:FullPow_2e2m}
%%   \end{center}
%% \end{figure}


%% %%%%%%%%%%%%%%%%%%%%%%%%%%%%%%%%%%%%%%%%%%%%%%%-HalfMad-%%%%%%%%%%%%%%%%%%%%%%%%%%%%%%%%%%%

%% \begin{figure}[hbtp]
%%   \begin{center}
%%     \includegraphics[width=\cmsFigWidth]{Figures/Unfolding/MCTests/Mass_ZZTo4e_MadMatrix_MadDistr_HalfSample}     
%%     \includegraphics[width=\cmsFigWidth]{Figures/Unfolding/MCTests/Deta_ZZTo4e_MadMatrix_MadDistr_HalfSample}   
%%     \includegraphics[width=\cmsFigWidth]{Figures/Unfolding/MCTests/Jets_ZZTo4e_MadMatrix_MadDistr_HalfSample}
%%     \includegraphics[width=\cmsFigWidth]{Figures/Unfolding/MCTests/CentralJets_ZZTo4e_MadMatrix_MadDistr_HalfSample}
%%     \includegraphics[width=\cmsFigWidth]{Figures/Unfolding/MCTests/Mjj_ZZTo4e_MadMatrix_MadDistr_HalfSample}
%%     \includegraphics[width=\cmsFigWidth]{Figures/Unfolding/MCTests/CentralMjj_ZZTo4e_MadMatrix_MadDistr_HalfSample}
%%     \includegraphics[width=\cmsFigWidth]{Figures/Unfolding/MCTests/PtJet1_ZZTo4e_MadMatrix_MadDistr_HalfSample}
%%     \includegraphics[width=\cmsFigWidth]{Figures/Unfolding/MCTests/PtJet2_ZZTo4e_MadMatrix_MadDistr_HalfSample}
%%       \caption{Unfolding test: \texttt{MadGraph} matrix applied on \texttt{MadGraph} distribution, using the two different halfs of the total sample.Results are reported as a function of 
%% the 4-lepton system (top left), the $\Delta\eta$ between the two most energetic jets (top right), the number of jets (second line left) and central jets (second line right) in the event, the invariant mass of the two most energetic jets (third line left) and  central jets (third line right), the transverse momentum of the leading (bottom left) and sub-leading (botton right) jets, for the $4e$ final state.}
%%     \label{fig:HalfMad_4e}
%%   \end{center}
%% \end{figure}
%% \begin{figure}[hbtp]
%%   \begin{center}
%%     \includegraphics[width=\cmsFigWidth]{Figures/Unfolding/MCTests/Mass_ZZTo4m_MadMatrix_MadDistr_HalfSample}     
%%     \includegraphics[width=\cmsFigWidth]{Figures/Unfolding/MCTests/Deta_ZZTo4m_MadMatrix_MadDistr_HalfSample}   
%%     \includegraphics[width=\cmsFigWidth]{Figures/Unfolding/MCTests/Jets_ZZTo4m_MadMatrix_MadDistr_HalfSample}
%%     \includegraphics[width=\cmsFigWidth]{Figures/Unfolding/MCTests/CentralJets_ZZTo4m_MadMatrix_MadDistr_HalfSample}
%%     \includegraphics[width=\cmsFigWidth]{Figures/Unfolding/MCTests/Mjj_ZZTo4m_MadMatrix_MadDistr_HalfSample}
%%     \includegraphics[width=\cmsFigWidth]{Figures/Unfolding/MCTests/CentralMjj_ZZTo4m_MadMatrix_MadDistr_HalfSample}
%%      \includegraphics[width=\cmsFigWidth]{Figures/Unfolding/MCTests/PtJet1_ZZTo4m_MadMatrix_MadDistr_HalfSample}
%%     \includegraphics[width=\cmsFigWidth]{Figures/Unfolding/MCTests/PtJet2_ZZTo4m_MadMatrix_MadDistr_HalfSample}
%%       \caption{Unfolding test: \texttt{MadGraph} matrix applied on \texttt{MadGraph} distribution, using the two different halves of the total sample. Results are reported as a function of 
%% the 4-lepton system (top left), the $\Delta\eta$ between the two most energetic jets (top right), the number of jets (second line left) and central jets (second line right) in the event, the invariant mass of the two most energetic jets (third line left) and  central jets (third line right), the transverse momentum of the leading (bottom left) and sub-leading (botton right) jets, for the $4\mu$ final state.}
%%     \label{fig:HalfMad_4m}
%%   \end{center}
%% \end{figure}
%% \begin{figure}[hbtp]
%%   \begin{center}
%%     \includegraphics[width=\cmsFigWidth]{Figures/Unfolding/MCTests/Mass_ZZTo2e2m_MadMatrix_MadDistr_HalfSample}     
%%     \includegraphics[width=\cmsFigWidth]{Figures/Unfolding/MCTests/Deta_ZZTo2e2m_MadMatrix_MadDistr_HalfSample}   
%%     \includegraphics[width=\cmsFigWidth]{Figures/Unfolding/MCTests/Jets_ZZTo2e2m_MadMatrix_MadDistr_HalfSample}
%%     \includegraphics[width=\cmsFigWidth]{Figures/Unfolding/MCTests/CentralJets_ZZTo2e2m_MadMatrix_MadDistr_HalfSample}
%%     \includegraphics[width=\cmsFigWidth]{Figures/Unfolding/MCTests/Mjj_ZZTo2e2m_MadMatrix_MadDistr_HalfSample}
%%     \includegraphics[width=\cmsFigWidth]{Figures/Unfolding/MCTests/CentralMjj_ZZTo2e2m_MadMatrix_MadDistr_HalfSample}
%%     \includegraphics[width=\cmsFigWidth]{Figures/Unfolding/MCTests/PtJet1_ZZTo2e2m_MadMatrix_MadDistr_HalfSample}
%%     \includegraphics[width=\cmsFigWidth]{Figures/Unfolding/MCTests/PtJet2_ZZTo2e2m_MadMatrix_MadDistr_HalfSample}
%%       \caption{Unfolding test: \texttt{MadGraph} matrix applied on \texttt{MadGraph} distribution, using the two different halves of the total sample. Results are reported as a function of 
%% the 4-lepton system (top left), the $\Delta\eta$ between the two most energetic jets (top right), the number of jets (second line left) and central jets (second line right) in the event, the invariant mass of the two most energetic jets (third line left) and  central jets (third line right), the transverse momentum of the leading (bottom left) and sub-leading (botton right) jets, for the $2e2\mu$ final state.}
%%     \label{fig:HalfMad_2e2m}
%%   \end{center}
%% \end{figure}


%%     %%%%%%%%%%%%%%%%%%%%%%%%%%%%%%%%%%%%%%%%%%%%%%%%%%%%%-HalfPow-%%%%%%%%%%%%%%%%%%%%%%%%%%%%%%%%%%%%%%%%%%%%%%%%%%%%%%%%%%%%%%%%%%%%%%%%%%%

%% \begin{figure}[hbtp]
%%   \begin{center}
%%     \includegraphics[width=\cmsFigWidth]{Figures/Unfolding/MCTests/Mass_ZZTo4e_PowMatrix_PowDistr_HalfSample}     
%%     \includegraphics[width=\cmsFigWidth]{Figures/Unfolding/MCTests/Deta_ZZTo4e_PowMatrix_PowDistr_HalfSample}   
%%     \includegraphics[width=\cmsFigWidth]{Figures/Unfolding/MCTests/Jets_ZZTo4e_PowMatrix_PowDistr_HalfSample}
%%     \includegraphics[width=\cmsFigWidth]{Figures/Unfolding/MCTests/CentralJets_ZZTo4e_PowMatrix_PowDistr_HalfSample}
%%     \includegraphics[width=\cmsFigWidth]{Figures/Unfolding/MCTests/Mjj_ZZTo4e_PowMatrix_PowDistr_HalfSample}
%%     \includegraphics[width=\cmsFigWidth]{Figures/Unfolding/MCTests/CentralMjj_ZZTo4e_PowMatrix_PowDistr_HalfSample}
%%     \includegraphics[width=\cmsFigWidth]{Figures/Unfolding/MCTests/PtJet1_ZZTo4e_PowMatrix_PowDistr_HalfSample}
%%     \includegraphics[width=\cmsFigWidth]{Figures/Unfolding/MCTests/PtJet2_ZZTo4e_PowMatrix_PowDistr_HalfSample}    
%%       \caption{Unfolding test: \texttt{Powheg} matrix applied on \texttt{Powheg} distribution, using the two different halves of the total sample. Results are reported as a function of 
%% the 4-lepton system (top left), the $\Delta\eta$ between the two most energetic jets (top right), the number of jets (second line left) and central jets (second line right) in the event, the invariant mass of the two most energetic jets (third line left) and  central jets (third line right), the transverse momentum of the leading (bottom left) and sub-leading (botton right) jets, for the $4e$ final state.}
%%     \label{fig:HalfPow_4e}
%%   \end{center}
%% \end{figure}
%% \begin{figure}[hbtp]
%%   \begin{center}
%%     \includegraphics[width=\cmsFigWidth]{Figures/Unfolding/MCTests/Mass_ZZTo4m_PowMatrix_PowDistr_HalfSample}     
%%     \includegraphics[width=\cmsFigWidth]{Figures/Unfolding/MCTests/Deta_ZZTo4m_PowMatrix_PowDistr_HalfSample}   
%%     \includegraphics[width=\cmsFigWidth]{Figures/Unfolding/MCTests/Jets_ZZTo4m_PowMatrix_PowDistr_HalfSample}
%%     \includegraphics[width=\cmsFigWidth]{Figures/Unfolding/MCTests/CentralJets_ZZTo4m_PowMatrix_PowDistr_HalfSample}
%%     \includegraphics[width=\cmsFigWidth]{Figures/Unfolding/MCTests/Mjj_ZZTo4m_PowMatrix_PowDistr_HalfSample}
%%     \includegraphics[width=\cmsFigWidth]{Figures/Unfolding/MCTests/CentralMjj_ZZTo4m_PowMatrix_PowDistr_HalfSample}
%%     \includegraphics[width=\cmsFigWidth]{Figures/Unfolding/MCTests/PtJet1_ZZTo4m_PowMatrix_PowDistr_HalfSample}
%%     \includegraphics[width=\cmsFigWidth]{Figures/Unfolding/MCTests/PtJet2_ZZTo4m_PowMatrix_PowDistr_HalfSample}
%%       \caption{Unfolding test: \texttt{Powheg} matrix applied on \texttt{Powheg} distribution, using the two different halves of the total sample. Results are reported as a function of 
%% the 4-lepton system (top left), the $\Delta\eta$ between the two most energetic jets (top right), the number of jets (second line left) and central jets (second line right) in the event, the invariant mass of the two most energetic jets (third line left) and  central jets (third line right), the transverse momentum of the leading (bottom left) and sub-leading (botton right) jets, for the $4\mu$ final state.}
%%     \label{fig:HalfPow_4m}
%%   \end{center}
%% \end{figure}
%% \begin{figure}[hbtp]
%%   \begin{center}
%%     \includegraphics[width=\cmsFigWidth]{Figures/Unfolding/MCTests/Mass_ZZTo2e2m_PowMatrix_PowDistr_HalfSample}     
%%     \includegraphics[width=\cmsFigWidth]{Figures/Unfolding/MCTests/Deta_ZZTo2e2m_PowMatrix_PowDistr_HalfSample}   
%%     \includegraphics[width=\cmsFigWidth]{Figures/Unfolding/MCTests/Jets_ZZTo2e2m_PowMatrix_PowDistr_HalfSample}
%%     \includegraphics[width=\cmsFigWidth]{Figures/Unfolding/MCTests/CentralJets_ZZTo2e2m_PowMatrix_PowDistr_HalfSample}
%%     \includegraphics[width=\cmsFigWidth]{Figures/Unfolding/MCTests/Mjj_ZZTo2e2m_PowMatrix_PowDistr_HalfSample}
%%     \includegraphics[width=\cmsFigWidth]{Figures/Unfolding/MCTests/CentralMjj_ZZTo2e2m_PowMatrix_PowDistr_HalfSample}
%%     \includegraphics[width=\cmsFigWidth]{Figures/Unfolding/MCTests/PtJet1_ZZTo2e2m_PowMatrix_PowDistr_HalfSample}
%%     \includegraphics[width=\cmsFigWidth]{Figures/Unfolding/MCTests/PtJet2_ZZTo2e2m_PowMatrix_PowDistr_HalfSample}
%%       \caption{Unfolding test: \texttt{Powheg} matrix applied on \texttt{Powheg} distribution, using the two different halves of the total sample.  Results are reported as a function of 
%% the 4-lepton system (top left), the $\Delta\eta$ between the two most energetic jets (top right), the number of jets (second line left) and central jets (second line right) in the event, the invariant mass of the two most energetic jets (third line left) and  central jets (third line right), the transverse momentum of the leading (bottom left) and sub-leading (botton right) jets, for the $2e2\mu$ final state.}
%%     \label{fig:HalfPow_2e2m}
%%   \end{center}
%% \end{figure}

%% %%%%%%%%%%%%%%%%%%%%%%%%%%%%%%%%%%%%%%%%%%%%%%%%%%%%%%%-MadOnPow-%%%%%%%%%%%%%%%%%%%%%%%%%%%%%%%%%%%%%%%%%%%%%%%%%%%%%%%%%%%%%%%%%%%%%%%%%%%%%%%%%
%% \begin{figure}[hbtp]
%%   \begin{center}
%%     \includegraphics[width=\cmsFigWidth]{Figures/Unfolding/MCTests/Mass_ZZTo4e_MadMatrix_PowDistr_FullSample}     
%%     \includegraphics[width=\cmsFigWidth]{Figures/Unfolding/MCTests/Deta_ZZTo4e_MadMatrix_PowDistr_FullSample}   
%%     \includegraphics[width=\cmsFigWidth]{Figures/Unfolding/MCTests/Jets_ZZTo4e_MadMatrix_PowDistr_FullSample}
%%     \includegraphics[width=\cmsFigWidth]{Figures/Unfolding/MCTests/CentralJets_ZZTo4e_MadMatrix_PowDistr_FullSample}
%%     \includegraphics[width=\cmsFigWidth]{Figures/Unfolding/MCTests/Mjj_ZZTo4e_MadMatrix_PowDistr_FullSample}
%%     \includegraphics[width=\cmsFigWidth]{Figures/Unfolding/MCTests/CentralMjj_ZZTo4e_MadMatrix_PowDistr_FullSample}
%%     \includegraphics[width=\cmsFigWidth]{Figures/Unfolding/MCTests/PtJet1_ZZTo4e_MadMatrix_PowDistr_FullSample}
%%     \includegraphics[width=\cmsFigWidth]{Figures/Unfolding/MCTests/PtJet2_ZZTo4e_MadMatrix_PowDistr_FullSample}    
%%   \caption{Unfolding test: \texttt{MadGraph} matrix applied on \texttt{Powheg} distribution, using the full set. Results are reported as a function of 
%% the 4-lepton system (top left), the $\Delta\eta$ between the two most energetic jets (top right), the number of jets (second line left) and central jets (second line right) in the event, the invariant mass of the two most energetic jets (third line left) and  central jets (third line right), the transverse momentum of the leading (bottom left) and sub-leading (botton right) jets, for the $4e$ final state.}
%%     \label{fig:MadMat_PowDist_4e}
%%   \end{center}
%% \end{figure}
%% \begin{figure}[hbtp]
%%   \begin{center}
%%     \includegraphics[width=\cmsFigWidth]{Figures/Unfolding/MCTests/Mass_ZZTo4m_MadMatrix_PowDistr_FullSample}     
%%     \includegraphics[width=\cmsFigWidth]{Figures/Unfolding/MCTests/Deta_ZZTo4m_MadMatrix_PowDistr_FullSample}   
%%     \includegraphics[width=\cmsFigWidth]{Figures/Unfolding/MCTests/Jets_ZZTo4m_MadMatrix_PowDistr_FullSample}
%%     \includegraphics[width=\cmsFigWidth]{Figures/Unfolding/MCTests/CentralJets_ZZTo4m_MadMatrix_PowDistr_FullSample}
%%     \includegraphics[width=\cmsFigWidth]{Figures/Unfolding/MCTests/Mjj_ZZTo4m_MadMatrix_PowDistr_FullSample}
%%     \includegraphics[width=\cmsFigWidth]{Figures/Unfolding/MCTests/CentralMjj_ZZTo4m_MadMatrix_PowDistr_FullSample}
%%     \includegraphics[width=\cmsFigWidth]{Figures/Unfolding/MCTests/PtJet1_ZZTo4m_MadMatrix_PowDistr_FullSample}
%%     \includegraphics[width=\cmsFigWidth]{Figures/Unfolding/MCTests/PtJet2_ZZTo4m_MadMatrix_PowDistr_FullSample}
%%       \caption{Unfolding test: \texttt{MadGraph} matrix applied on \texttt{Powheg} distribution, using the full set. Results are reported as a function of 
%% the 4-lepton system (top left), the $\Delta\eta$ between the two most energetic jets (top right), the number of jets (second line left) and central jets (second line right) in the event, the invariant mass of the two most energetic jets (third line left) and  central jets (third line right), the transverse momentum of the leading (bottom left) and sub-leading (botton right) jets, for the $4\mu$ final state.}
%%     \label{fig:MadMat_PowDist_4m}
%%   \end{center}
%% \end{figure}

%% \clearpage

%% \begin{figure}[hbtp]
%%   \begin{center}
%%     \includegraphics[width=\cmsFigWidth]{Figures/Unfolding/MCTests/Mass_ZZTo2e2m_MadMatrix_PowDistr_FullSample}     
%%     \includegraphics[width=\cmsFigWidth]{Figures/Unfolding/MCTests/Deta_ZZTo2e2m_MadMatrix_PowDistr_FullSample}   
%%     \includegraphics[width=\cmsFigWidth]{Figures/Unfolding/MCTests/Jets_ZZTo2e2m_MadMatrix_PowDistr_FullSample}
%%     \includegraphics[width=\cmsFigWidth]{Figures/Unfolding/MCTests/CentralJets_ZZTo2e2m_MadMatrix_PowDistr_FullSample}
%%     \includegraphics[width=\cmsFigWidth]{Figures/Unfolding/MCTests/Mjj_ZZTo2e2m_MadMatrix_PowDistr_FullSample}
%%     \includegraphics[width=\cmsFigWidth]{Figures/Unfolding/MCTests/CentralMjj_ZZTo2e2m_MadMatrix_PowDistr_FullSample}
%%     \includegraphics[width=\cmsFigWidth]{Figures/Unfolding/MCTests/PtJet1_ZZTo2e2m_MadMatrix_PowDistr_FullSample}
%%     \includegraphics[width=\cmsFigWidth]{Figures/Unfolding/MCTests/PtJet2_ZZTo2e2m_MadMatrix_PowDistr_FullSample}
%%       \caption{Unfolding test: \texttt{MadGraph} matrix applied on \texttt{Powheg} distribution, using the full set. Results are reported as a function of 
%% the 4-lepton system (top left), the $\Delta\eta$ between the two most energetic jets (top right), the number of jets (second line left) and central jets (second line right) in the event, the invariant mass of the two most energetic jets (third line left) and  central jets (third line right), the transverse momentum of the leading (bottom left) and sub-leading (botton right) jets, for the $2e2\mu$ final state.}
%%     \label{fig:MadMat_PowDist_2e2m}
%%   \end{center}
%% \end{figure}



%% %%%%%%%%%%%%%%%%%%%%%%%%%%%%%%%%%%%%%%%%%%%%-PowOnMad-%%%%%%%%%%%%%%%%%%%%%%%%%%%%%%%%%%%%%

%% \begin{figure}[hbtp]
%%   \begin{center}
%%     \includegraphics[width=\cmsFigWidth]{Figures/Unfolding/MCTests/Mass_ZZTo4e_PowMatrix_MadDistr_FullSample}     
%%     \includegraphics[width=\cmsFigWidth]{Figures/Unfolding/MCTests/Deta_ZZTo4e_PowMatrix_MadDistr_FullSample}   
%%     \includegraphics[width=\cmsFigWidth]{Figures/Unfolding/MCTests/Jets_ZZTo4e_PowMatrix_MadDistr_FullSample}
%%     \includegraphics[width=\cmsFigWidth]{Figures/Unfolding/MCTests/CentralJets_ZZTo4e_PowMatrix_MadDistr_FullSample}
%%     \includegraphics[width=\cmsFigWidth]{Figures/Unfolding/MCTests/Mjj_ZZTo4e_PowMatrix_MadDistr_FullSample}
%%     \includegraphics[width=\cmsFigWidth]{Figures/Unfolding/MCTests/CentralMjj_ZZTo4e_PowMatrix_MadDistr_FullSample}
%%     \includegraphics[width=\cmsFigWidth]{Figures/Unfolding/MCTests/PtJet1_ZZTo4e_PowMatrix_MadDistr_FullSample}
%%     \includegraphics[width=\cmsFigWidth]{Figures/Unfolding/MCTests/PtJet2_ZZTo4e_PowMatrix_MadDistr_FullSample}
%%       \caption{Unfolding test: \texttt{Powheg} matrix applied on \texttt{MadGraph} distribution, using the full set. Results are reported as a function of 
%% the 4-lepton system (top left), the $\Delta\eta$ between the two most energetic jets (top right), the number of jets (second line left) and central jets (second line right) in the event, the invariant mass of the two most energetic jets (third line left) and  central jets (third line right), the transverse momentum of the leading (bottom left) and sub-leading (botton right) jets, for the $4e$ final state.}
%%     \label{fig:PowMat_MadDist_4e}
%%   \end{center}
%% \end{figure}
%% \begin{figure}[hbtp]
%%   \begin{center}
%%     \includegraphics[width=\cmsFigWidth]{Figures/Unfolding/MCTests/Mass_ZZTo4m_PowMatrix_MadDistr_FullSample}     
%%     \includegraphics[width=\cmsFigWidth]{Figures/Unfolding/MCTests/Deta_ZZTo4m_PowMatrix_MadDistr_FullSample}   
%%     \includegraphics[width=\cmsFigWidth]{Figures/Unfolding/MCTests/Jets_ZZTo4m_PowMatrix_MadDistr_FullSample}
%%     \includegraphics[width=\cmsFigWidth]{Figures/Unfolding/MCTests/CentralJets_ZZTo4m_PowMatrix_MadDistr_FullSample}
%%     \includegraphics[width=\cmsFigWidth]{Figures/Unfolding/MCTests/Mjj_ZZTo4m_PowMatrix_MadDistr_FullSample}
%%     \includegraphics[width=\cmsFigWidth]{Figures/Unfolding/MCTests/CentralMjj_ZZTo4m_PowMatrix_MadDistr_FullSample}
%%     \includegraphics[width=\cmsFigWidth]{Figures/Unfolding/MCTests/PtJet1_ZZTo4m_PowMatrix_MadDistr_FullSample}
%%     \includegraphics[width=\cmsFigWidth]{Figures/Unfolding/MCTests/PtJet2_ZZTo4m_PowMatrix_MadDistr_FullSample}
%%       \caption{Unfolding test: \texttt{Powheg} matrix applied on \texttt{MadGraph} distribution, using the full set. Results are reported as a function of 
%% the 4-lepton system (top left), the $\Delta\eta$ between the two most energetic jets (top right), the number of jets (second line left) and central jets (second line right) in the event, the invariant mass of the two most energetic jets (third line left) and  central jets (third line right), the transverse momentum of the leading (bottom left) and sub-leading (botton right) jets, for the $4\mu$ final state.}
%%     \label{fig:PowMat_MadDist_4m}
%%   \end{center}
%% \end{figure}

%% %\clearpage

%% \begin{figure}[hbtp]
%%   \begin{center}
%%     \includegraphics[width=\cmsFigWidth]{Figures/Unfolding/MCTests/Mass_ZZTo2e2m_PowMatrix_MadDistr_FullSample}     
%%     \includegraphics[width=\cmsFigWidth]{Figures/Unfolding/MCTests/Deta_ZZTo2e2m_PowMatrix_MadDistr_FullSample}   
%%     \includegraphics[width=\cmsFigWidth]{Figures/Unfolding/MCTests/Jets_ZZTo2e2m_PowMatrix_MadDistr_FullSample}
%%     \includegraphics[width=\cmsFigWidth]{Figures/Unfolding/MCTests/CentralJets_ZZTo2e2m_PowMatrix_MadDistr_FullSample}
%%     \includegraphics[width=\cmsFigWidth]{Figures/Unfolding/MCTests/Mjj_ZZTo2e2m_PowMatrix_MadDistr_FullSample}
%%     \includegraphics[width=\cmsFigWidth]{Figures/Unfolding/MCTests/CentralMjj_ZZTo2e2m_PowMatrix_MadDistr_FullSample}
%%     \includegraphics[width=\cmsFigWidth]{Figures/Unfolding/MCTests/PtJet1_ZZTo2e2m_PowMatrix_MadDistr_FullSample}
%%     \includegraphics[width=\cmsFigWidth]{Figures/Unfolding/MCTests/PtJet2_ZZTo2e2m_PowMatrix_MadDistr_FullSample}
%%       \caption{Unfolding test: \texttt{Powheg} matrix applied on \texttt{MadGraph} distribution, using the full set. Results are reported as a function of 
%% the 4-lepton system (top left), the $\Delta\eta$ between the two most energetic jets (top right), the number of jets (second line left) and central jets (second line right) in the event, the invariant mass of the two most energetic jets (third line left) and  central jets (third line right), the transverse momentum of the leading (bottom left) and sub-leading (botton right) jets, for the $4\mu$ final state.}
%%     \label{fig:PowMat_MadDist_2e2m}
%%   \end{center}
%% \end{figure}


\subsubsection{Choice of the unfolding algorithm}
In order to choose the proper algorithm, results obtained using the SVD method and the so-called iterative ``Bayesian'' method are compared.\\
\\
The SVD method depends on the choice of the regularization parameter ($k_{reg}$), which is needed since matrix inversion is sensitive to the statistical fluctuations and is in the range $[1,N_{bin}]$. The optimal value of $k_{reg}$ is the one for which the errors associated to the unfolding procedure are small when compared to the statistical ones. This parameter can be seen as a cut-off for quickly oscillating terms corresponding to data statistical fluctuations,  as described in detail in~\cite{SVD}.\\  %da ampliare?? 
Choosing a too small regularization parameter gets ride of these spurious fluctuations but the result tends to be biased by the MC truth. On the other hand, a too large regularization parameter will decrease the MC dependence but give too large importance to data fluctuations which will be interpreted as real shape.\\    
\\
The Bayes method using an iterative approach requires a choice for the maximum number of iterations to be done that, as discussed in~\cite{DAgostini}, 
should not lead to different results. In this analysis, the iteration procedure is stopped as soon as the new unfolded distribution is compatible
with the one obtained from the previous step. This is quantified by computing the $\chi ^2/n.d.f.$ of the change in the unfolded distribution for every iteration step.
The number of iterations in the Bayes unfolding procedure is set as the one when the next iteration is compatible with the current one by computing:
\begin{itemize}
\item $\sum_{i = 1}^{N_{bin}} (b_n^i - b_{n+1}^i)^2$, the sum of the deviations
\item $\chi_{n}^2 = \sum_{i = 1}^{N_{bin}} \frac{(b_n^i - b_{n+1}^i)^2}{(db_n^i)^2}$
 \item $\chi_{n+1}^2 = \sum_{i = 1}^{N_{bin}} \frac{(b_n^i - b_{n+1}^i)^2}{(db_{n+1}^i)^2}$
\end{itemize}
where $b^i_n$ is the content of the $ith$-bin of the histogram obtained using $n$ iterations and $db_{n}^i$ is the systematic uncertainty associated to  $b^i_n$.\\
\\
The unfolded $m_{ZZ}$-distributions are obtained using the SVD algorithm. In this case the regularization parameter is chosen to be $k_{reg} = 4$, the default value
($N_{bin}/2$), since as shown in Figure~\ref{fig:Mass_biased} the distributions obtained using $k_{reg} = 2$ are clearly biased by the MC truth. On the other hand, all the other jet-related variables distributions are unfolded using the Bayesian algorithm, with four iterations. As shown in Figure~\ref{fig:Jets_biased}, this choice is not biased by the MC truth distribution.\\

\begin{figure}[hbtp]
 \begin{center}
   \includegraphics[width=0.8\cmsFigWidth]{Figures/Unfolding/MCTests/Biased_Distributions/Mass_ZZTo4e_Pow_fr_SVD_2}     
   \includegraphics[width=0.8\cmsFigWidth]{Figures/Unfolding/MCTests/Biased_Distributions/Mass_ZZTo4m_Pow_fr_SVD_2}     
   \includegraphics[width=0.8\cmsFigWidth]{Figures/Unfolding/MCTests/Biased_Distributions/Mass_ZZTo2e2m_Pow_fr_SVD_2}
   \includegraphics[width=0.8\cmsFigWidth]{Figures/Unfolding/MCTests/Biased_Distributions/Mass_ZZTo4e_Pow_fr_SVD_4}     
   \includegraphics[width=0.8\cmsFigWidth]{Figures/Unfolding/MCTests/Biased_Distributions/Mass_ZZTo4m_Pow_fr_SVD_4}     
   \includegraphics[width=0.8\cmsFigWidth]{Figures/Unfolding/MCTests/Biased_Distributions/Mass_ZZTo2e2m_Pow_fr_SVD_4}
   \includegraphics[width=0.8\cmsFigWidth]{Figures/Unfolding/MCTests/Biased_Distributions/Mass_ZZTo4e_Pow_fr_bayes_4}     
   \includegraphics[width=0.8\cmsFigWidth]{Figures/Unfolding/MCTests/Biased_Distributions/Mass_ZZTo4m_Pow_fr_bayes_4}     
   \includegraphics[width=0.8\cmsFigWidth]{Figures/Unfolding/MCTests/Biased_Distributions/Mass_ZZTo2e2m_Pow_fr_bayes_4}
    \caption{A flat distribution (in red) is unfolded using the SVD algorithm with $k_{reg} = 2$ (top) and $k_{reg} = 4$ (center) and using the D'Agostini method with 4 iterations (bottom). Results are reported as a function of the invariant mass of the 4-lepton system  and for the $4e$ (left), $4\mu$ (center) and $2e2\mu$ (right) final state.}
   \label{fig:Mass_biased}
 \end{center}
\end{figure}

\begin{figure}[hbtp]
  \begin{center}
    \includegraphics[width=0.8\cmsFigWidth]{Figures/Unfolding/MCTests/Biased_Distributions/Jets_ZZTo4e_Mad_fr_SVD_2}     
    \includegraphics[width=0.8\cmsFigWidth]{Figures/Unfolding/MCTests/Biased_Distributions/Jets_ZZTo4m_Mad_fr_SVD_2}     
    \includegraphics[width=0.8\cmsFigWidth]{Figures/Unfolding/MCTests/Biased_Distributions/Jets_ZZTo2e2m_Mad_fr_SVD_2}
    \includegraphics[width=0.8\cmsFigWidth]{Figures/Unfolding/MCTests/Biased_Distributions/Jets_ZZTo4e_Mad_fr_SVD_4}     
    \includegraphics[width=0.8\cmsFigWidth]{Figures/Unfolding/MCTests/Biased_Distributions/Jets_ZZTo4m_Mad_fr_SVD_4}     
    \includegraphics[width=0.8\cmsFigWidth]{Figures/Unfolding/MCTests/Biased_Distributions/Jets_ZZTo2e2m_Mad_fr_SVD_4}
    \includegraphics[width=0.8\cmsFigWidth]{Figures/Unfolding/MCTests/Biased_Distributions/Jets_ZZTo4e_Mad_fr_bayes_4}     
    \includegraphics[width=0.8\cmsFigWidth]{Figures/Unfolding/MCTests/Biased_Distributions/Jets_ZZTo4m_Mad_fr_bayes_4}     
    \includegraphics[width=0.8\cmsFigWidth]{Figures/Unfolding/MCTests/Biased_Distributions/Jets_ZZTo2e2m_Mad_fr_bayes_4}
     \caption{A flat distribution (in red) is unfolded using the SVD algorithm with $k_{reg} = 2$ (top) and $k_{reg} = 4$ (center) and using the D'Agostini method with 4 iterations (bottom). Results are reported as a function of the number of jets and for the $4e$ (left), $4\mu$ (center) and $2e2\mu$ (right) final state.}
    \label{fig:Jets_biased}
  \end{center}
\end{figure}
\clearpage

%\subsection{Unfolding data distributions} FIXME: qui non c'e' granche' da dire...

\subsection{Systematic Uncertainties}
To evaluate the effects of the different sources of systematic uncertainty on the unfolded distributions, the whole unfolding procedure
is repeated for every contribution. The following systematic uncertainties are considered for the result and their contributions
are individually reported in Figures from~\ref{fig:Mass_syst_4e} to~\ref{fig:EtaJet2syst2e2m}, for all variables and final states.   
%In Figures~\ref{fig:mass_bkg_shift} and~\ref{fig:jets_bkg_shift} the shifted distributions are reported, together with the nominal one, for the studied observables.
\subsubsection{Unfolding (measurement and response matrix uncertainties)}
Uncertainties due to the unfolded procedure are computed by the \texttt{RooUnfold} toolkit and consist of two types, the 
``measurement uncertainties'' and the ``response matrix uncertainties''. The former come from the propagation of the statistical 
errors of the distribution that has to be unfolded through the unfolding matrix, in particular bin-to-bin correlations, and must be considered 
as statistical uncertainties. The latter are uncertainties on the response matrix elements due to the limited MC statistics but can be neglected.% that in case of more than one iteration can not be neglected –iterations depend on the result of the previous step and thus they depend on the response matrix as well

\subsubsection{$\sigma_{qq}$ and $\sigma_{gg}$ ratio}
The response matrix is built adding the \texttt{MCFM} sample, which describes $gg\to ZZ\to 4\ell$ processes, to  the \texttt{MadGraph} (or \texttt{Powheg})
one, that contains $qq/qg\to ZZ \to 4\ell$ events. The cross-sections used to simulate these processes are estimated with \texttt{MCFM} and are affected
by an uncertainties ($d\sigma_{qq\to ZZ}= \pm 4.44\%$ and $d\sigma_{gg\to ZZ}= \pm 25.36\%$). This leads to a systematic effect that is evaluated building 
up new response matrices by varying the cross-sections by $\pm d\sigma_{qq/gg}$. The combinations returning the largest discrepancy are those in which 
the corrections of $\sigma_{qq}$ and $\sigma_{gg}$ have opposite sign, since in this case the difference of shapes is enhanced. The unfolded procedure 
is repeated using these new matrices and the systematic uncertainty is evaluated taking the difference between the largest and the smallest results 
for each bin. This contribution is found to be less than 1\%. 
\subsubsection{Choice of MC generator}
This systematic uncertainty is computed comparing unfolded data distributions obtained applying the \texttt{Madgraph+MCFM+Phantom} or the \texttt{Powheg+MCFM+Phantom}
response matrix and taking the difference between the two results. It varies between 1\% and 10\%.

\subsubsection{Irreducible and reducible backgrounds}
Uncertainties due to background estimate are evaluated by creating a new ``data-background'' distribution, in which the background is shifted up and down by
its uncertainties, that is unfolded using standard matrices. This procedure is done for both reducible (1-16\%) and irreducible (0.1 - 5.5\%) contributions. 

\subsubsection{Unfolded Data over MC Truth ratio}
This systematic uncertainty is evaluated building up a new response matrix whose elements are weighted for the ratio between the unfolded data and the generator level 
information. The difference between the unfolded results obtained by the application of this weighted response matrix and the standard one is considered as a 
systematic error~\cite{CMS_AN_13-165}. This uncertainty ranges from 1\% to 29\%, in the last bins where the statistics is very low, and it is thus one
of the largest contributions.

\subsubsection{Lepton Efficiency}
This uncertainty comes from the measurement of data/MC efficiency scale factors used to weight MC distributions for lepton reconstruction. It is evaluated shfting up and down the efficiency by its error and building up two new response matrices to be used in the unfolding procedure. The difference between the unfolded results obtained applying these two ``up-'' and ``down-''matrices is taken as the systematic uncertainty. Its contributions is found to ranges between 1\% and 6\%.


\subsubsection{Jet energy resolutions (JER)}
This correction has been estimated for data and MC in~\cite{JECandJER}. MC slightly overestimates the resolution compared to data. The effect is propagated accordingly by smearing the samples used to build the response matrices. This uncertainty is less than 3.5\%.%is found to range from 0.2\% up to $\sim$30\% in the last bins, where the statistics is very low. 

\subsubsection{Jet energy scale corrections (JES)}
This uncertainty is calculated by rescaling the jet $p_T$ spectrum up and down by one standard deviation of the measured jet energy correction. This is done separately in data~\cite{CMS_PAS_13-007} and in MC reconstructed distributions used for the response matrix. The distributions obtained using the standard corrections and its variations are then compared and their difference is taken as systematic uncertainty. When the jet transverse momenta of reconstructed MC distributions are shifted, this uncertainty ranges from 1.4\% to 6.9\%, while in the case of modified data distributions, it grows up to 66\% in the last bins and it is thus the largest contribution affecting
the measurement.% in data~\cite{CMS_PAS_13-007} and, separately, in MC reconstructed distributions. It is $p_T$ and $\eta$ dependent and, together with the unfolded-over-MC-truth ratio, gives the largest contribution.
% for the phase space of interest, are estimated to be from $\sim$1 up to $\sim$30\% in the last bins, where the statistics is very low.



\begin{figure}[hbtp]
 \begin{center}
    \includegraphics[width=0.8\cmsFigWidth]{Figures/Unfolding/Systematics/ZZTo4e_Mass_qqgg_Pow_fr}     
    \includegraphics[width=0.8\cmsFigWidth]{Figures/Unfolding/Systematics/ZZTo4e_Mass_MCgen_Pow_fr}     
    \includegraphics[width=0.8\cmsFigWidth]{Figures/Unfolding/Systematics/ZZTo4e_Mass_IrrBkg_Pow_fr}
    \includegraphics[width=0.8\cmsFigWidth]{Figures/Unfolding/Systematics/ZZTo4e_Mass_RedBkg_Pow_fr}     
    \includegraphics[width=0.8\cmsFigWidth]{Figures/Unfolding/Systematics/ZZTo4e_Mass_UnfDataOverGenMC_Pow_fr}
    \includegraphics[width=0.8\cmsFigWidth]{Figures/Unfolding/Systematics/ZZTo4e_Mass_SFSq_Pow_fr}          
    %\includegraphics[width=0.8\cmsFigWidth]{Figures/Unfolding/Systematics/ZZTo4e_Mass_JER_fr}
   % \includegraphics[width=0.8\cmsFigWidth]{Figures/Unfolding/Systematics/ZZTo4e_Mass_JES_ModData_fr}     
   % \includegraphics[width=0.8\cmsFigWidth]{Figures/Unfolding/Systematics/ZZTo4e_Mass_JES_ModMat_fr}
    \caption{Effects of the different sources of systematic uncertainty on the unfolded distributions of the $4\ell$ mass, for the     
    $4e$ final state. From left to right, from top to bottom: $\sigma_{qq}$ and $\sigma_{gg}$ ratio, MC generator, irreducible background,
reducible background, unfolded/truth ratio, lepton efficiency. The systematic effect is superimposed on the nominal unfolded distribution, togheter with MC predictions from \texttt{MadGraph} and \texttt{Pohweg} sets of samples.}
    \label{fig:Mass_syst_4e}
  \end{center}
\end{figure}

\begin{figure}[hbtp]
  \begin{center}
    \includegraphics[width=0.8\cmsFigWidth]{Figures/Unfolding/Systematics/ZZTo4m_Mass_qqgg_Pow_fr}     
    \includegraphics[width=0.8\cmsFigWidth]{Figures/Unfolding/Systematics/ZZTo4m_Mass_MCgen_Pow_fr}     
    \includegraphics[width=0.8\cmsFigWidth]{Figures/Unfolding/Systematics/ZZTo4m_Mass_IrrBkg_Pow_fr}
    \includegraphics[width=0.8\cmsFigWidth]{Figures/Unfolding/Systematics/ZZTo4m_Mass_RedBkg_Pow_fr}     
    \includegraphics[width=0.8\cmsFigWidth]{Figures/Unfolding/Systematics/ZZTo4m_Mass_UnfDataOverGenMC_Pow_fr}  
    \includegraphics[width=0.8\cmsFigWidth]{Figures/Unfolding/Systematics/ZZTo4m_Mass_SFSq_Pow_fr}                  
   % \includegraphics[width=0.8\cmsFigWidth]{Figures/Unfolding/Systematics/ZZTo4m_Mass_JER_Pow_fr}
    %\includegraphics[width=0.8\cmsFigWidth]{Figures/Unfolding/Systematics/ZZTo4m_Mass_JES_ModData_Pow_fr}     
   % \includegraphics[width=0.8\cmsFigWidth]{Figures/Unfolding/Systematics/ZZTo4m_Mass_JES_ModMat_Pow_fr}
    \caption{Effects of the different sources of systematic uncertainty on the unfolded distributions of the $4\ell$ mass, for the     
    $4\mu$ final state. From left to right, from top to bottom: $\sigma_{qq}$ and $\sigma_{gg}$ ratio, MC generator, irreducible background,
reducible background, unfolded/truth ratio, lepton efficiency. The systematic effect is superimposed on the nominal unfolded distribution, togheter with MC predictions from \texttt{MadGraph} and \texttt{Pohweg} sets of samples.}
    \label{fig:Mass_syst_4m}
  \end{center}
\end{figure}

\begin{figure}[hbtp]
  \begin{center}
    \includegraphics[width=0.8\cmsFigWidth]{Figures/Unfolding/Systematics/ZZTo2e2m_Mass_qqgg_Pow_fr}     
    \includegraphics[width=0.8\cmsFigWidth]{Figures/Unfolding/Systematics/ZZTo2e2m_Mass_MCgen_Pow_fr}     
    \includegraphics[width=0.8\cmsFigWidth]{Figures/Unfolding/Systematics/ZZTo2e2m_Mass_IrrBkg_Pow_fr}
    \includegraphics[width=0.8\cmsFigWidth]{Figures/Unfolding/Systematics/ZZTo2e2m_Mass_RedBkg_Pow_fr}     
    \includegraphics[width=0.8\cmsFigWidth]{Figures/Unfolding/Systematics/ZZTo2e2m_Mass_UnfDataOverGenMC_Pow_fr}  
     \includegraphics[width=0.8\cmsFigWidth]{Figures/Unfolding/Systematics/ZZTo2e2m_Mass_SFSq_Pow_fr}     
   % \includegraphics[width=0.8\cmsFigWidth]{Figures/Unfolding/Systematics/ZZTo2e2m_Mass_JER_Pow}
   % \includegraphics[width=0.8\cmsFigWidth]{Figures/Unfolding/Systematics/ZZTo2e2m_Mass_JES_ModData_Pow}     
  % \includegraphics[width=0.8\cmsFigWidth]{Figures/Unfolding/Systematics/ZZTo2e2m_Mass_JES_ModMat_Pow}
    \caption{Effects of the different sources of systematic uncertainty on the unfolded distributions of the $4\ell$ mass, for the     
    $2e2\mu$ final state. From left to right, from top to bottom: $\sigma_{qq}$ and $\sigma_{gg}$ ratio, MC generator, irreducible background,
reducible background, unfolded/truth ratio, lepton efficiency. The systematic effect is superimposed on the nominal unfolded distribution, togheter with MC predictions from \texttt{MadGraph} and \texttt{Pohweg} sets of samples.}
    \label{fig:Mass_syst_2e2m}
  \end{center}
\end{figure}
\clearpage
\begin{figure}[hbtp]
  \begin{center}
    \includegraphics[width=0.8\cmsFigWidth]{Figures/Unfolding/Systematics/ZZTo4e_Jets_qqgg_Mad_fr}     
    \includegraphics[width=0.8\cmsFigWidth]{Figures/Unfolding/Systematics/ZZTo4e_Jets_MCgen_Mad_fr}     
    \includegraphics[width=0.8\cmsFigWidth]{Figures/Unfolding/Systematics/ZZTo4e_Jets_IrrBkg_Mad_fr}
    \includegraphics[width=0.8\cmsFigWidth]{Figures/Unfolding/Systematics/ZZTo4e_Jets_RedBkg_Mad_fr}     
    \includegraphics[width=0.8\cmsFigWidth]{Figures/Unfolding/Systematics/ZZTo4e_Jets_UnfDataOverGenMC_Mad_fr}  
     \includegraphics[width=0.8\cmsFigWidth]{Figures/Unfolding/Systematics/ZZTo4e_Jets_SFSq_Mad_fr}       
     \includegraphics[width=0.8\cmsFigWidth]{Figures/Unfolding/Systematics/ZZTo4e_Jets_JER_Mad_fr}
    \includegraphics[width=0.8\cmsFigWidth]{Figures/Unfolding/Systematics/ZZTo4e_Jets_JES_ModData_Mad_fr}     
    \includegraphics[width=0.8\cmsFigWidth]{Figures/Unfolding/Systematics/ZZTo4e_Jets_JES_ModMat_Mad_fr}
       \caption{Effects of the different sources of systematic uncertainty on the unfolded distributions of the number of jets (with $|\eta^{jet}|<4.7$), for the     
    $4e$ final state. From left to right, from top to bottom: $\sigma_{qq}$ and $\sigma_{gg}$ ratio, MC generator, irreducible background,
reducible background, unfolded/truth ratio, lepton efficiency, JER, JES modifying data distribution, JES modifying the response matrix. The systematic effect is superimposed on the nominal unfolded distribution, togheter with MC predictions from \texttt{MadGraph} and \texttt{Pohweg} sets of samples.}
    \label{fig:Jets_syst_4e}
  \end{center}
\end{figure}

\begin{figure}[hbtp]
  \begin{center}
    \includegraphics[width=0.8\cmsFigWidth]{Figures/Unfolding/Systematics/ZZTo4m_Jets_qqgg_Mad_fr}     
    \includegraphics[width=0.8\cmsFigWidth]{Figures/Unfolding/Systematics/ZZTo4m_Jets_MCgen_Mad_fr}     
    \includegraphics[width=0.8\cmsFigWidth]{Figures/Unfolding/Systematics/ZZTo4m_Jets_IrrBkg_Mad_fr}
    \includegraphics[width=0.8\cmsFigWidth]{Figures/Unfolding/Systematics/ZZTo4m_Jets_RedBkg_Mad_fr}     
    \includegraphics[width=0.8\cmsFigWidth]{Figures/Unfolding/Systematics/ZZTo4m_Jets_UnfDataOverGenMC_Mad_fr}  
    \includegraphics[width=0.8\cmsFigWidth]{Figures/Unfolding/Systematics/ZZTo4m_Jets_SFSq_Mad_fr}                    
    \includegraphics[width=0.8\cmsFigWidth]{Figures/Unfolding/Systematics/ZZTo4m_Jets_JER_Mad_fr}
    \includegraphics[width=0.8\cmsFigWidth]{Figures/Unfolding/Systematics/ZZTo4m_Jets_JES_ModData_Mad_fr}     
    \includegraphics[width=0.8\cmsFigWidth]{Figures/Unfolding/Systematics/ZZTo4m_Jets_JES_ModMat_Mad_fr}
        \caption{Effects of the different sources of systematic uncertainty on the unfolded distributions of the number of jets (with $|\eta^{jet}|<4.7$), for the     
    $4\mu$ final state. From left to right, from top to bottom: $\sigma_{qq}$ and $\sigma_{gg}$ ratio, MC generator, irreducible background,
reducible background, unfolded/truth ratio, lepton efficiency, JER, JES modifying data distribution, JES modifying the response matrix. The systematic effect is superimposed on the nominal unfolded distribution, togheter with MC predictions from \texttt{MadGraph} and \texttt{Pohweg} sets of samples.}
    \label{fig:Jets_syst_4m}
  \end{center}
\end{figure}

\begin{figure}[hbtp]
  \begin{center}
    \includegraphics[width=0.8\cmsFigWidth]{Figures/Unfolding/Systematics/ZZTo2e2m_Jets_qqgg_Mad_fr}     
    \includegraphics[width=0.8\cmsFigWidth]{Figures/Unfolding/Systematics/ZZTo2e2m_Jets_MCgen_Mad_fr}     
    \includegraphics[width=0.8\cmsFigWidth]{Figures/Unfolding/Systematics/ZZTo2e2m_Jets_IrrBkg_Mad_fr}
    \includegraphics[width=0.8\cmsFigWidth]{Figures/Unfolding/Systematics/ZZTo2e2m_Jets_RedBkg_Mad_fr}     
    \includegraphics[width=0.8\cmsFigWidth]{Figures/Unfolding/Systematics/ZZTo2e2m_Jets_UnfDataOverGenMC_Mad_fr}  
    \includegraphics[width=0.8\cmsFigWidth]{Figures/Unfolding/Systematics/ZZTo2e2m_Jets_SFSq_Mad_fr}                       
    \includegraphics[width=0.8\cmsFigWidth]{Figures/Unfolding/Systematics/ZZTo2e2m_Jets_JER_Mad_fr}
    \includegraphics[width=0.8\cmsFigWidth]{Figures/Unfolding/Systematics/ZZTo2e2m_Jets_JES_ModData_Mad_fr}     
    \includegraphics[width=0.8\cmsFigWidth]{Figures/Unfolding/Systematics/ZZTo2e2m_Jets_JES_ModMat_Mad_fr}
    \caption{Effects of the different sources of systematic uncertainty on the unfolded distributions of the number of jets (with $|\eta^{jet}|<4.7$), for the     
    $2e2\mu$ final state. From left to right, from top to bottom: $\sigma_{qq}$ and $\sigma_{gg}$ ratio, MC generator, irreducible background,
reducible background, unfolded/truth ratio, lepton efficiency, JER, JES modifying data distribution, JES modifying the response matrix. The systematic effect is superimposed on the nominal unfolded distribution, togheter with MC predictions from \texttt{MadGraph} and \texttt{Pohweg} sets of samples.}
    \label{fig:Jets_syst_2e2m}
  \end{center}
\end{figure}

\clearpage
\begin{figure}[hbtp]
  \begin{center}
    \includegraphics[width=0.8\cmsFigWidth]{Figures/Unfolding/Systematics/ZZTo4e_CentralJets_qqgg_Mad_fr}     
    \includegraphics[width=0.8\cmsFigWidth]{Figures/Unfolding/Systematics/ZZTo4e_CentralJets_MCgen_Mad_fr}     
    \includegraphics[width=0.8\cmsFigWidth]{Figures/Unfolding/Systematics/ZZTo4e_CentralJets_IrrBkg_Mad_fr}
    \includegraphics[width=0.8\cmsFigWidth]{Figures/Unfolding/Systematics/ZZTo4e_CentralJets_RedBkg_Mad_fr}     
    \includegraphics[width=0.8\cmsFigWidth]{Figures/Unfolding/Systematics/ZZTo4e_CentralJets_UnfDataOverGenMC_Mad_fr}     
    \includegraphics[width=0.8\cmsFigWidth]{Figures/Unfolding/Systematics/ZZTo4e_CentralJets_SFSq_Mad_fr}
    \includegraphics[width=0.8\cmsFigWidth]{Figures/Unfolding/Systematics/ZZTo4e_CentralJets_JER_Mad_fr}   
    \includegraphics[width=0.8\cmsFigWidth]{Figures/Unfolding/Systematics/ZZTo4e_CentralJets_JES_ModData_Mad_fr}     
    \includegraphics[width=0.8\cmsFigWidth]{Figures/Unfolding/Systematics/ZZTo4e_CentralJets_JES_ModMat_Mad_fr}
    \caption{Effects of the different sources of systematic uncertainty on the unfolded distributions of the number of central jets (with $|\eta^{jet}|<2.4$), for the     
    $4e$ final state. From left to right, from top to bottom: $\sigma_{qq}$ and $\sigma_{gg}$ ratio, MC generator, irreducible background,
reducible background, unfolded/truth ratio, lepton efficiency, JER, JES modifying data distribution, JES modifying the response matrix. The systematic effect is superimposed on the nominal unfolded distribution, togheter with MC predictions from \texttt{MadGraph} and \texttt{Pohweg} sets of samples.}
    \label{fig:CentralJets_syst_4e}
  \end{center}
\end{figure}
\clearpage
\begin{figure}[hbtp]
  \begin{center}
    \includegraphics[width=0.8\cmsFigWidth]{Figures/Unfolding/Systematics/ZZTo4m_CentralJets_qqgg_Mad_fr}     
    \includegraphics[width=0.8\cmsFigWidth]{Figures/Unfolding/Systematics/ZZTo4m_CentralJets_MCgen_Mad_fr}     
    \includegraphics[width=0.8\cmsFigWidth]{Figures/Unfolding/Systematics/ZZTo4m_CentralJets_IrrBkg_Mad_fr}
    \includegraphics[width=0.8\cmsFigWidth]{Figures/Unfolding/Systematics/ZZTo4m_CentralJets_RedBkg_Mad_fr}     
    \includegraphics[width=0.8\cmsFigWidth]{Figures/Unfolding/Systematics/ZZTo4m_CentralJets_UnfDataOverGenMC_Mad_fr}     
    \includegraphics[width=0.8\cmsFigWidth]{Figures/Unfolding/Systematics/ZZTo4m_CentralJets_SFSq_Mad_fr}        
    \includegraphics[width=0.8\cmsFigWidth]{Figures/Unfolding/Systematics/ZZTo4m_CentralJets_JER_Mad_fr}
    \includegraphics[width=0.8\cmsFigWidth]{Figures/Unfolding/Systematics/ZZTo4m_CentralJets_JES_ModData_Mad_fr}     
    \includegraphics[width=0.8\cmsFigWidth]{Figures/Unfolding/Systematics/ZZTo4m_CentralJets_JES_ModMat_Mad_fr}
    \caption{Effects of the different sources of systematic uncertainty on the unfolded distributions of the number of central jets (with $|\eta^{jet}|<2.4$), for the     
    $4\mu$ final state. From left to right, from top to bottom: $\sigma_{qq}$ and $\sigma_{gg}$ ratio, MC generator, irreducible background,
reducible background, unfolded/truth ratio, lepton efficiency, JER, JES modifying data distribution, JES modifying the response matrix. The systematic effect is superimposed on the nominal unfolded distribution, togheter with MC predictions from \texttt{MadGraph} and \texttt{Pohweg} sets of samples.}
    \label{fig:CentralJets_syst_4m}
  \end{center}
\end{figure}

\begin{figure}[hbtp]
  \begin{center}
    \includegraphics[width=0.8\cmsFigWidth]{Figures/Unfolding/Systematics/ZZTo2e2m_CentralJets_qqgg_Mad_fr}     
    \includegraphics[width=0.8\cmsFigWidth]{Figures/Unfolding/Systematics/ZZTo2e2m_CentralJets_MCgen_Mad_fr}     
    \includegraphics[width=0.8\cmsFigWidth]{Figures/Unfolding/Systematics/ZZTo2e2m_CentralJets_IrrBkg_Mad_fr}
    \includegraphics[width=0.8\cmsFigWidth]{Figures/Unfolding/Systematics/ZZTo2e2m_CentralJets_RedBkg_Mad_fr}     
    \includegraphics[width=0.8\cmsFigWidth]{Figures/Unfolding/Systematics/ZZTo2e2m_CentralJets_UnfDataOverGenMC_Mad_fr}     
    \includegraphics[width=0.8\cmsFigWidth]{Figures/Unfolding/Systematics/ZZTo2e2m_CentralJets_SFSq_Mad_fr}
    \includegraphics[width=0.8\cmsFigWidth]{Figures/Unfolding/Systematics/ZZTo2e2m_CentralJets_JER_Mad_fr}
    \includegraphics[width=0.8\cmsFigWidth]{Figures/Unfolding/Systematics/ZZTo2e2m_CentralJets_JES_ModData_Mad_fr}     
    \includegraphics[width=0.8\cmsFigWidth]{Figures/Unfolding/Systematics/ZZTo2e2m_CentralJets_JES_ModMat_Mad_fr}
    \caption{Effects of the different sources of systematic uncertainty on the unfolded distributions of the number of central jets (with $|\eta^{jet}|<2.4$), for the     
    $2e2\mu$ final state. From left to right, from top to bottom: $\sigma_{qq}$ and $\sigma_{gg}$ ratio, MC generator, irreducible background,
reducible background, unfolded/truth ratio, lepton efficiency, JER, JES modifying data distribution, JES modifying the response matrix. The systematic effect is superimposed on the nominal unfolded distribution, togheter with MC predictions from \texttt{MadGraph} and \texttt{Pohweg} sets of samples.}
    \label{fig:CentralJets_syst_2e2m}
  \end{center}
\end{figure}
\clearpage
\begin{figure}[hbtp]
  \begin{center}
    \includegraphics[width=0.8\cmsFigWidth]{Figures/Unfolding/Systematics/ZZTo4e_Mjj_qqgg_Mad_fr}     
    \includegraphics[width=0.8\cmsFigWidth]{Figures/Unfolding/Systematics/ZZTo4e_Mjj_MCgen_Mad_fr}     
    \includegraphics[width=0.8\cmsFigWidth]{Figures/Unfolding/Systematics/ZZTo4e_Mjj_IrrBkg_Mad_fr}
    \includegraphics[width=0.8\cmsFigWidth]{Figures/Unfolding/Systematics/ZZTo4e_Mjj_RedBkg_Mad_fr}     
    \includegraphics[width=0.8\cmsFigWidth]{Figures/Unfolding/Systematics/ZZTo4e_Mjj_UnfDataOverGenMC_Mad_fr}     
    \includegraphics[width=0.8\cmsFigWidth]{Figures/Unfolding/Systematics/ZZTo4e_Mjj_SFSq_Mad_fr}
    \includegraphics[width=0.8\cmsFigWidth]{Figures/Unfolding/Systematics/ZZTo4e_Mjj_JER_Mad_fr}
    \includegraphics[width=0.8\cmsFigWidth]{Figures/Unfolding/Systematics/ZZTo4e_Mjj_JES_ModData_Mad_fr}     
    \includegraphics[width=0.8\cmsFigWidth]{Figures/Unfolding/Systematics/ZZTo4e_Mjj_JES_ModMat_Mad_fr}
    \caption{Effects of the different sources of systematic uncertainty on the unfolded distributions of $m_{jj}$ (with $|\eta^{jet}|<4.7$), for the $4e$ final state. From left to right, from top to bottom: $\sigma_{qq}$ and $\sigma_{gg}$ ratio, MC generator, irreducible background,
reducible background, unfolded/truth ratio, lepton efficiency, JER, JES modifying data distribution, JES modifying the response matrix. The systematic effect is superimposed on the nominal unfolded distribution, togheter with MC predictions from \texttt{MadGraph} and \texttt{Pohweg} sets of samples.}
    \label{fig:Mjj_syst_4e}
  \end{center}
\end{figure}

\begin{figure}[hbtp]
  \begin{center}
    \includegraphics[width=0.8\cmsFigWidth]{Figures/Unfolding/Systematics/ZZTo4m_Mjj_qqgg_Mad_fr}     
    \includegraphics[width=0.8\cmsFigWidth]{Figures/Unfolding/Systematics/ZZTo4m_Mjj_MCgen_Mad_fr}     
    \includegraphics[width=0.8\cmsFigWidth]{Figures/Unfolding/Systematics/ZZTo4m_Mjj_IrrBkg_Mad_fr}
    \includegraphics[width=0.8\cmsFigWidth]{Figures/Unfolding/Systematics/ZZTo4m_Mjj_RedBkg_Mad_fr}     
    \includegraphics[width=0.8\cmsFigWidth]{Figures/Unfolding/Systematics/ZZTo4m_Mjj_UnfDataOverGenMC_Mad_fr}     
    \includegraphics[width=0.8\cmsFigWidth]{Figures/Unfolding/Systematics/ZZTo4m_Mjj_SFSq_Mad_fr}
    \includegraphics[width=0.8\cmsFigWidth]{Figures/Unfolding/Systematics/ZZTo4m_Mjj_JER_Mad_fr}
    \includegraphics[width=0.8\cmsFigWidth]{Figures/Unfolding/Systematics/ZZTo4m_Mjj_JES_ModData_Mad_fr}     
    \includegraphics[width=0.8\cmsFigWidth]{Figures/Unfolding/Systematics/ZZTo4m_Mjj_JES_ModMat_Mad_fr}
    \caption{Effects of the different sources of systematic uncertainty on the unfolded distributions of $m_{jj}$ (with $|\eta^{jet}|<4.7$), for the $4\mu$ final state. From left to right, from top to bottom: $\sigma_{qq}$ and $\sigma_{gg}$ ratio, MC generator, irreducible background, reducible background, unfolded/truth ratio, lepton efficiency, JER, JES modifying data distribution, JES modifying the response matrix. The shifted distributions due to the systematic effects are reported, together with the nominal one and the MC truth.}
    \label{fig:Mjj_syst_4m}
  \end{center}
\end{figure}

\begin{figure}[hbtp]
  \begin{center}
    \includegraphics[width=0.8\cmsFigWidth]{Figures/Unfolding/Systematics/ZZTo2e2m_Mjj_qqgg_Mad_fr}     
    \includegraphics[width=0.8\cmsFigWidth]{Figures/Unfolding/Systematics/ZZTo2e2m_Mjj_MCgen_Mad_fr}     
    \includegraphics[width=0.8\cmsFigWidth]{Figures/Unfolding/Systematics/ZZTo2e2m_Mjj_IrrBkg_Mad_fr}
    \includegraphics[width=0.8\cmsFigWidth]{Figures/Unfolding/Systematics/ZZTo2e2m_Mjj_RedBkg_Mad_fr}     
    \includegraphics[width=0.8\cmsFigWidth]{Figures/Unfolding/Systematics/ZZTo2e2m_Mjj_UnfDataOverGenMC_Mad_fr}     
    \includegraphics[width=0.8\cmsFigWidth]{Figures/Unfolding/Systematics/ZZTo2e2m_Mjj_SFSq_Mad_fr}
    \includegraphics[width=0.8\cmsFigWidth]{Figures/Unfolding/Systematics/ZZTo2e2m_Mjj_JER_Mad_fr}
    \includegraphics[width=0.8\cmsFigWidth]{Figures/Unfolding/Systematics/ZZTo2e2m_Mjj_JES_ModData_Mad_fr}     
    \includegraphics[width=0.8\cmsFigWidth]{Figures/Unfolding/Systematics/ZZTo2e2m_Mjj_JES_ModMat_Mad_fr}
    \caption{Effects of the different sources of systematic uncertainty on the unfolded distributions of $m_{jj}$ (with $|\eta^{jet}|<4.7$), for the $2e2\mu$ final state. From left to right, from top to bottom: $\sigma_{qq}$ and $\sigma_{gg}$ ratio, MC generator, irreducible background, reducible background, unfolded/truth ratio, lepton efficiency, JER, JES modifying data distribution, JES modifying the response matrix. The shifted distributions due to the systematic effects are reported, together with the nominal one and the MC truth.}
   \label{fig:Mjj_syst_2e2m}
  \end{center}
\end{figure}
\clearpage
\begin{figure}[hbtp]
  \begin{center}
    \includegraphics[width=0.8\cmsFigWidth]{Figures/Unfolding/Systematics/ZZTo4e_CentralMjj_qqgg_Mad_fr}     
    \includegraphics[width=0.8\cmsFigWidth]{Figures/Unfolding/Systematics/ZZTo4e_CentralMjj_MCgen_Mad_fr}     
    \includegraphics[width=0.8\cmsFigWidth]{Figures/Unfolding/Systematics/ZZTo4e_CentralMjj_IrrBkg_Mad_fr}
    \includegraphics[width=0.8\cmsFigWidth]{Figures/Unfolding/Systematics/ZZTo4e_CentralMjj_RedBkg_Mad_fr}     
    \includegraphics[width=0.8\cmsFigWidth]{Figures/Unfolding/Systematics/ZZTo4e_CentralMjj_UnfDataOverGenMC_Mad_fr}     
    \includegraphics[width=0.8\cmsFigWidth]{Figures/Unfolding/Systematics/ZZTo4e_CentralMjj_SFSq_Mad_fr}
    \includegraphics[width=0.8\cmsFigWidth]{Figures/Unfolding/Systematics/ZZTo4e_CentralMjj_JER_Mad_fr}
    \includegraphics[width=0.8\cmsFigWidth]{Figures/Unfolding/Systematics/ZZTo4e_CentralMjj_JES_ModData_Mad_fr}     
    \includegraphics[width=0.8\cmsFigWidth]{Figures/Unfolding/Systematics/ZZTo4e_CentralMjj_JES_ModMat_Mad_fr}
    \caption{Effects of the different sources of systematic uncertainty on the unfolded distributions of $m_{jj}$ (with $|\eta^{jet}|<2.4$), for the $4e$ final state. From left to right, from top to bottom: $\sigma_{qq}$ and $\sigma_{gg}$ ratio, MC generator, irreducible background, reducible background, unfolded/truth ratio, lepton efficiency, JER, JES modifying data distribution, JES modifying the response matrix. The systematic effect is superimposed on the nominal unfolded distribution, togheter with MC predictions from \texttt{MadGraph} and \texttt{Pohweg} sets of samples.}
    \label{fig:CentralMjj_syst_4e}
  \end{center}
\end{figure}

\begin{figure}[hbtp]
  \begin{center}
    \includegraphics[width=0.8\cmsFigWidth]{Figures/Unfolding/Systematics/ZZTo4m_CentralMjj_qqgg_Mad_fr}     
    \includegraphics[width=0.8\cmsFigWidth]{Figures/Unfolding/Systematics/ZZTo4m_CentralMjj_MCgen_Mad_fr}     
    \includegraphics[width=0.8\cmsFigWidth]{Figures/Unfolding/Systematics/ZZTo4m_CentralMjj_IrrBkg_Mad_fr}
    \includegraphics[width=0.8\cmsFigWidth]{Figures/Unfolding/Systematics/ZZTo4m_CentralMjj_RedBkg_Mad_fr}     
    \includegraphics[width=0.8\cmsFigWidth]{Figures/Unfolding/Systematics/ZZTo4m_CentralMjj_UnfDataOverGenMC_Mad_fr}     
    \includegraphics[width=0.8\cmsFigWidth]{Figures/Unfolding/Systematics/ZZTo4m_CentralMjj_SFSq_Mad_fr}
    \includegraphics[width=0.8\cmsFigWidth]{Figures/Unfolding/Systematics/ZZTo4m_CentralMjj_JER_Mad_fr}
    \includegraphics[width=0.8\cmsFigWidth]{Figures/Unfolding/Systematics/ZZTo4m_CentralMjj_JES_ModData_Mad_fr}     
    \includegraphics[width=0.8\cmsFigWidth]{Figures/Unfolding/Systematics/ZZTo4m_CentralMjj_JES_ModMat_Mad_fr}
    \caption{Effects of the different sources of systematic uncertainty on the unfolded distributions of $m_{jj}$ (with $|\eta^{jet}|<2.4$), for the $4\mu$ final state. From left to right, from top to bottom: $\sigma_{qq}$ and $\sigma_{gg}$ ratio, MC generator, irreducible background, reducible background, unfolded/truth ratio, lepton efficiency, JER, JES modifying data distribution, JES modifying the response matrix. The shifted distributions due to the systematic effects are reported, together with the nominal one and the MC truth.}
    \label{fig:CentralMjj_syst_4m}
  \end{center}
\end{figure}

\begin{figure}[hbtp]
  \begin{center}
    \includegraphics[width=0.8\cmsFigWidth]{Figures/Unfolding/Systematics/ZZTo2e2m_CentralMjj_qqgg_Mad_fr}     
    \includegraphics[width=0.8\cmsFigWidth]{Figures/Unfolding/Systematics/ZZTo2e2m_CentralMjj_MCgen_Mad_fr}     
    \includegraphics[width=0.8\cmsFigWidth]{Figures/Unfolding/Systematics/ZZTo2e2m_CentralMjj_IrrBkg_Mad_fr}
    \includegraphics[width=0.8\cmsFigWidth]{Figures/Unfolding/Systematics/ZZTo2e2m_CentralMjj_RedBkg_Mad_fr}     
    \includegraphics[width=0.8\cmsFigWidth]{Figures/Unfolding/Systematics/ZZTo2e2m_CentralMjj_UnfDataOverGenMC_Mad_fr}     
    \includegraphics[width=0.8\cmsFigWidth]{Figures/Unfolding/Systematics/ZZTo2e2m_CentralMjj_SFSq_Mad_fr}
    \includegraphics[width=0.8\cmsFigWidth]{Figures/Unfolding/Systematics/ZZTo2e2m_CentralMjj_JER_Mad_fr}
    \includegraphics[width=0.8\cmsFigWidth]{Figures/Unfolding/Systematics/ZZTo2e2m_CentralMjj_JES_ModData_Mad_fr}     
    \includegraphics[width=0.8\cmsFigWidth]{Figures/Unfolding/Systematics/ZZTo2e2m_CentralMjj_JES_ModMat_Mad_fr}
    \caption{Effects of the different sources of systematic uncertainty on the unfolded distributions of $m_{jj}$ (with $|\eta^{jet}|<2.4$), for the $2e2\mu$ final state. From left to right, from top to bottom: $\sigma_{qq}$ and $\sigma_{gg}$ ratio, MC generator, irreducible background, reducible background, unfolded/truth ratio, lepton efficiency, JER, JES modifying data distribution, JES modifying the response matrix. The shifted distributions due to the systematic effects are reported, together with the nominal one and the MC truth.}
   \label{fig:CentralMjj_syst_2e2m}
  \end{center}
\end{figure}


\clearpage
\begin{figure}[hbtp]
 \begin{center}
   \includegraphics[width=0.8\cmsFigWidth]{Figures/Unfolding/Systematics/ZZTo4e_Deta_qqgg_Mad_fr}     
   \includegraphics[width=0.8\cmsFigWidth]{Figures/Unfolding/Systematics/ZZTo4e_Deta_MCgen_Mad_fr}     
   \includegraphics[width=0.8\cmsFigWidth]{Figures/Unfolding/Systematics/ZZTo4e_Deta_IrrBkg_Mad_fr}
   \includegraphics[width=0.8\cmsFigWidth]{Figures/Unfolding/Systematics/ZZTo4e_Deta_RedBkg_Mad_fr}     
   \includegraphics[width=0.8\cmsFigWidth]{Figures/Unfolding/Systematics/ZZTo4e_Deta_UnfDataOverGenMC_Mad_fr}     
   \includegraphics[width=0.8\cmsFigWidth]{Figures/Unfolding/Systematics/ZZTo4e_Deta_SFSq_Mad_fr}
   \includegraphics[width=0.8\cmsFigWidth]{Figures/Unfolding/Systematics/ZZTo4e_Deta_JER_Mad_fr}
   \includegraphics[width=0.8\cmsFigWidth]{Figures/Unfolding/Systematics/ZZTo4e_Deta_JES_ModData_Mad_fr}     
   \includegraphics[width=0.8\cmsFigWidth]{Figures/Unfolding/Systematics/ZZTo4e_Deta_JES_ModMat_Mad_fr}
   \caption{Effects of the different sources of systematic uncertainty on the unfolded distributions of $\Delta\eta_{jj}$ (with $|\eta^{jet}|<4.7$), for the $4e$ final state. From left to right, from top to bottom: $\sigma_{qq}$ and $\sigma_{gg}$ ratio, MC generator, irreducible background, reducible background, unfolded/truth ratio, lepton efficiency, JER, JES modifying data distribution, JES modifying the response matrix. The systematic effect is superimposed on the nominal unfolded distribution, togheter with MC predictions from \texttt{MadGraph} and \texttt{Pohweg} sets of samples.}
   \label{fig:Deta_syst_4e}
 \end{center}
\end{figure}

\begin{figure}[hbtp]
 \begin{center}
   \includegraphics[width=0.8\cmsFigWidth]{Figures/Unfolding/Systematics/ZZTo4m_Deta_qqgg_Mad_fr}     
   \includegraphics[width=0.8\cmsFigWidth]{Figures/Unfolding/Systematics/ZZTo4m_Deta_MCgen_Mad_fr}     
   \includegraphics[width=0.8\cmsFigWidth]{Figures/Unfolding/Systematics/ZZTo4m_Deta_IrrBkg_Mad_fr}
   \includegraphics[width=0.8\cmsFigWidth]{Figures/Unfolding/Systematics/ZZTo4m_Deta_RedBkg_Mad_fr}     
   \includegraphics[width=0.8\cmsFigWidth]{Figures/Unfolding/Systematics/ZZTo4m_Deta_UnfDataOverGenMC_Mad_fr}     
   \includegraphics[width=0.8\cmsFigWidth]{Figures/Unfolding/Systematics/ZZTo4m_Deta_SFSq_Mad_fr}
   \includegraphics[width=0.8\cmsFigWidth]{Figures/Unfolding/Systematics/ZZTo4m_Deta_JER_Mad_fr}
   \includegraphics[width=0.8\cmsFigWidth]{Figures/Unfolding/Systematics/ZZTo4m_Deta_JES_ModData_Mad_fr}     
   \includegraphics[width=0.8\cmsFigWidth]{Figures/Unfolding/Systematics/ZZTo4m_Deta_JES_ModMat_Mad_fr}
   \caption{Effects of the different sources of systematic uncertainty on the unfolded distributions of $\Delta\eta_{jj}$ (with $|\eta^{jet}|<4.7$), for the $4\mu$ final state. From left to right, from top to bottom: $\sigma_{qq}$ and $\sigma_{gg}$ ratio, MC generator, irreducible background, reducible background, unfolded/truth ratio, lepton efficiency, JER, JES modifying data distribution, JES modifying the response matrix. The systematic effect is superimposed on the nominal unfolded distribution, togheter with MC predictions from \texttt{MadGraph} and \texttt{Pohweg} sets of samples.}
   \label{fig:Deta_syst_4m}
 \end{center}
\end{figure}

\begin{figure}[hbtp]
 \begin{center}
   \includegraphics[width=0.8\cmsFigWidth]{Figures/Unfolding/Systematics/ZZTo2e2m_Deta_qqgg_Mad_fr}     
   \includegraphics[width=0.8\cmsFigWidth]{Figures/Unfolding/Systematics/ZZTo2e2m_Deta_MCgen_Mad_fr}     
   \includegraphics[width=0.8\cmsFigWidth]{Figures/Unfolding/Systematics/ZZTo2e2m_Deta_IrrBkg_Mad_fr}
   \includegraphics[width=0.8\cmsFigWidth]{Figures/Unfolding/Systematics/ZZTo2e2m_Deta_RedBkg_Mad_fr}     
   \includegraphics[width=0.8\cmsFigWidth]{Figures/Unfolding/Systematics/ZZTo2e2m_Deta_UnfDataOverGenMC_Mad_fr}     
   \includegraphics[width=0.8\cmsFigWidth]{Figures/Unfolding/Systematics/ZZTo2e2m_Deta_SFSq_Mad_fr}
   \includegraphics[width=0.8\cmsFigWidth]{Figures/Unfolding/Systematics/ZZTo2e2m_Deta_JER_Mad_fr}
   \includegraphics[width=0.8\cmsFigWidth]{Figures/Unfolding/Systematics/ZZTo2e2m_Deta_JES_ModData_Mad_fr}     
   \includegraphics[width=0.8\cmsFigWidth]{Figures/Unfolding/Systematics/ZZTo2e2m_Deta_JES_ModMat_Mad_fr}
   \caption{Effects of the different sources of systematic uncertainty on the unfolded distributions of $\Delta\eta_{jj}$ (with $|\eta^{jet}|<4.7$), for the $2e2\mu$ final state. From left to right, from top to bottom: $\sigma_{qq}$ and $\sigma_{gg}$ ratio, MC generator, irreducible background, reducible background, unfolded/truth ratio, lepton efficiency, JER, JES modifying data distribution, JES modifying the response matrix. The systematic effect is superimposed on the nominal unfolded distribution, togheter with MC predictions from \texttt{MadGraph} and \texttt{Pohweg} sets of samples.}
   \label{fig:Detasyst2e2m}
 \end{center}
\end{figure}
\clearpage
\begin{figure}[hbtp]
 \begin{center}
   \includegraphics[width=0.8\cmsFigWidth]{Figures/Unfolding/Systematics/ZZTo4e_CentralDeta_qqgg_Mad_fr}     
   \includegraphics[width=0.8\cmsFigWidth]{Figures/Unfolding/Systematics/ZZTo4e_CentralDeta_MCgen_Mad_fr}     
   \includegraphics[width=0.8\cmsFigWidth]{Figures/Unfolding/Systematics/ZZTo4e_CentralDeta_IrrBkg_Mad_fr}
   \includegraphics[width=0.8\cmsFigWidth]{Figures/Unfolding/Systematics/ZZTo4e_CentralDeta_RedBkg_Mad_fr}     
   \includegraphics[width=0.8\cmsFigWidth]{Figures/Unfolding/Systematics/ZZTo4e_CentralDeta_UnfDataOverGenMC_Mad_fr}     
   \includegraphics[width=0.8\cmsFigWidth]{Figures/Unfolding/Systematics/ZZTo4e_CentralDeta_SFSq_Mad_fr}
   \includegraphics[width=0.8\cmsFigWidth]{Figures/Unfolding/Systematics/ZZTo4e_CentralDeta_JER_Mad_fr}
   \includegraphics[width=0.8\cmsFigWidth]{Figures/Unfolding/Systematics/ZZTo4e_CentralDeta_JES_ModData_Mad_fr}     
   \includegraphics[width=0.8\cmsFigWidth]{Figures/Unfolding/Systematics/ZZTo4e_CentralDeta_JES_ModMat_Mad_fr}
   \caption{Effects of the different sources of systematic uncertainty on the unfolded distributions of $\Delta\eta_{jj}$ (with $|\eta^{jet}|<2.4$), for the $4e$ final state. From left to right, from top to bottom: $\sigma_{qq}$ and $\sigma_{gg}$ ratio, MC generator, irreducible background, reducible background, unfolded/truth ratio, lepton efficiency, JER, JES modifying data distribution, JES modifying the response matrix. The systematic effect is superimposed on the nominal unfolded distribution, togheter with MC predictions from \texttt{MadGraph} and \texttt{Pohweg} sets of samples.}
   \label{fig:CentralDeta_syst_4e}
 \end{center}
\end{figure}

\begin{figure}[hbtp]
 \begin{center}
   \includegraphics[width=0.8\cmsFigWidth]{Figures/Unfolding/Systematics/ZZTo4m_CentralDeta_qqgg_Mad_fr}     
   \includegraphics[width=0.8\cmsFigWidth]{Figures/Unfolding/Systematics/ZZTo4m_CentralDeta_MCgen_Mad_fr}     
   \includegraphics[width=0.8\cmsFigWidth]{Figures/Unfolding/Systematics/ZZTo4m_CentralDeta_IrrBkg_Mad_fr}
   \includegraphics[width=0.8\cmsFigWidth]{Figures/Unfolding/Systematics/ZZTo4m_CentralDeta_RedBkg_Mad_fr}     
   \includegraphics[width=0.8\cmsFigWidth]{Figures/Unfolding/Systematics/ZZTo4m_CentralDeta_UnfDataOverGenMC_Mad_fr}     
   \includegraphics[width=0.8\cmsFigWidth]{Figures/Unfolding/Systematics/ZZTo4m_CentralDeta_SFSq_Mad_fr}
   \includegraphics[width=0.8\cmsFigWidth]{Figures/Unfolding/Systematics/ZZTo4m_CentralDeta_JER_Mad_fr}
   \includegraphics[width=0.8\cmsFigWidth]{Figures/Unfolding/Systematics/ZZTo4m_CentralDeta_JES_ModData_Mad_fr}     
   \includegraphics[width=0.8\cmsFigWidth]{Figures/Unfolding/Systematics/ZZTo4m_CentralDeta_JES_ModMat_Mad_fr}
   \caption{Effects of the different sources of systematic uncertainty on the unfolded distributions of $\Delta\eta_{jj}$ (with $|\eta^{jet}|<2.4$), for the $4\mu$ final state. From left to right, from top to bottom: $\sigma_{qq}$ and $\sigma_{gg}$ ratio, MC generator, irreducible background, reducible background, unfolded/truth ratio, lepton efficiency, JER, JES modifying data distribution, JES modifying the response matrix. The systematic effect is superimposed on the nominal unfolded distribution, togheter with MC predictions from \texttt{MadGraph} and \texttt{Pohweg} sets of samples.}
   \label{fig:CentralDeta_syst_4m}
 \end{center}
\end{figure}

\begin{figure}[hbtp]
 \begin{center}
   \includegraphics[width=0.8\cmsFigWidth]{Figures/Unfolding/Systematics/ZZTo2e2m_CentralDeta_qqgg_Mad_fr}     
   \includegraphics[width=0.8\cmsFigWidth]{Figures/Unfolding/Systematics/ZZTo2e2m_CentralDeta_MCgen_Mad_fr}     
   \includegraphics[width=0.8\cmsFigWidth]{Figures/Unfolding/Systematics/ZZTo2e2m_CentralDeta_IrrBkg_Mad_fr}
   \includegraphics[width=0.8\cmsFigWidth]{Figures/Unfolding/Systematics/ZZTo2e2m_CentralDeta_RedBkg_Mad_fr}     
   \includegraphics[width=0.8\cmsFigWidth]{Figures/Unfolding/Systematics/ZZTo2e2m_CentralDeta_UnfDataOverGenMC_Mad_fr}     
   \includegraphics[width=0.8\cmsFigWidth]{Figures/Unfolding/Systematics/ZZTo2e2m_CentralDeta_SFSq_Mad_fr}
   \includegraphics[width=0.8\cmsFigWidth]{Figures/Unfolding/Systematics/ZZTo2e2m_CentralDeta_JER_Mad_fr}
   \includegraphics[width=0.8\cmsFigWidth]{Figures/Unfolding/Systematics/ZZTo2e2m_CentralDeta_JES_ModData_Mad_fr}     
   \includegraphics[width=0.8\cmsFigWidth]{Figures/Unfolding/Systematics/ZZTo2e2m_CentralDeta_JES_ModMat_Mad_fr}
   \caption{Effects of the different sources of systematic uncertainty on the unfolded distributions of $\Delta\eta_{jj}$ (with $|\eta^{jet}|<2.4$), for the $2e2\mu$ final state. From left to right, from top to bottom: $\sigma_{qq}$ and $\sigma_{gg}$ ratio, MC generator, irreducible background, reducible background, unfolded/truth ratio, lepton efficiency, JER, JES modifying data distribution, JES modifying the response matrix. The systematic effect is superimposed on the nominal unfolded distribution, togheter with MC predictions from \texttt{MadGraph} and \texttt{Pohweg} sets of samples.}
   \label{fig:CentralDetasyst2e2m}
 \end{center}
\end{figure}
\clearpage
\begin{figure}[hbtp]
 \begin{center}
   \includegraphics[width=0.8\cmsFigWidth]{Figures/Unfolding/Systematics/ZZTo4e_PtJet1_qqgg_Mad_fr}     
   \includegraphics[width=0.8\cmsFigWidth]{Figures/Unfolding/Systematics/ZZTo4e_PtJet1_MCgen_Mad_fr}     
   \includegraphics[width=0.8\cmsFigWidth]{Figures/Unfolding/Systematics/ZZTo4e_PtJet1_IrrBkg_Mad_fr}
   \includegraphics[width=0.8\cmsFigWidth]{Figures/Unfolding/Systematics/ZZTo4e_PtJet1_RedBkg_Mad_fr}     
  \includegraphics[width=0.8\cmsFigWidth]{Figures/Unfolding/Systematics/ZZTo4e_PtJet1_UnfDataOverGenMC_Mad_fr}     
   \includegraphics[width=0.8\cmsFigWidth]{Figures/Unfolding/Systematics/ZZTo4e_PtJet1_SFSq_Mad_fr}
   \includegraphics[width=0.8\cmsFigWidth]{Figures/Unfolding/Systematics/ZZTo4e_PtJet1_JER_Mad_fr}
   \includegraphics[width=0.8\cmsFigWidth]{Figures/Unfolding/Systematics/ZZTo4e_PtJet1_JES_ModData_Mad_fr}     
   \includegraphics[width=0.8\cmsFigWidth]{Figures/Unfolding/Systematics/ZZTo4e_PtJet1_JES_ModMat_Mad_fr}
   \caption{Effects of the different sources of systematic uncertainty on the unfolded distributions of  $p_{T}^{jet1}$, for the     
   $4e$ final state. From left to right, from top to bottom: $\sigma_{qq}$ and $\sigma_{gg}$ ratio, MC generator, irreducible background, reducible background, unfolded/truth ratio, lepton efficiency, JER, JES modifying data distribution, JES modifying the response matrix. The systematic effect is superimposed on the nominal unfolded distribution, togheter with MC predictions from \texttt{MadGraph} and \texttt{Pohweg} sets of samples.}
   \label{fig:PtJet1_syst_4e}
 \end{center}
\end{figure}

\begin{figure}[hbtp]
 \begin{center}
   \includegraphics[width=0.8\cmsFigWidth]{Figures/Unfolding/Systematics/ZZTo4m_PtJet1_qqgg_Mad_fr}     
   \includegraphics[width=0.8\cmsFigWidth]{Figures/Unfolding/Systematics/ZZTo4m_PtJet1_MCgen_Mad_fr}     
   \includegraphics[width=0.8\cmsFigWidth]{Figures/Unfolding/Systematics/ZZTo4m_PtJet1_IrrBkg_Mad_fr}
   \includegraphics[width=0.8\cmsFigWidth]{Figures/Unfolding/Systematics/ZZTo4m_PtJet1_RedBkg_Mad_fr}     
  \includegraphics[width=0.8\cmsFigWidth]{Figures/Unfolding/Systematics/ZZTo4m_PtJet1_UnfDataOverGenMC_Mad_fr}     
   \includegraphics[width=0.8\cmsFigWidth]{Figures/Unfolding/Systematics/ZZTo4m_PtJet1_SFSq_Mad_fr}
   \includegraphics[width=0.8\cmsFigWidth]{Figures/Unfolding/Systematics/ZZTo4m_PtJet1_JER_Mad_fr}
   \includegraphics[width=0.8\cmsFigWidth]{Figures/Unfolding/Systematics/ZZTo4m_PtJet1_JES_ModData_Mad_fr}     
   \includegraphics[width=0.8\cmsFigWidth]{Figures/Unfolding/Systematics/ZZTo4m_PtJet1_JES_ModMat_Mad_fr}
   \caption{Effects of the different sources of systematic uncertainty on the unfolded distributions of  $p_{T}^{jet1}$, for the     
   $4\mu$ final state. From left to right, from top to bottom: $\sigma_{qq}$ and $\sigma_{gg}$ ratio, MC generator, irreducible background, reducible background, unfolded/truth ratio, lepton efficiency, JER, JES modifying data distribution, JES modifying the response matrix. The systematic effect is superimposed on the nominal unfolded distribution, togheter with MC predictions from \texttt{MadGraph} and \texttt{Pohweg} sets of samples.}
   \label{fig:PtJet1_syst_4m}
 \end{center}
\end{figure}

\begin{figure}[hbtp]
 \begin{center}
   \includegraphics[width=0.8\cmsFigWidth]{Figures/Unfolding/Systematics/ZZTo2e2m_PtJet1_qqgg_Mad_fr}     
   \includegraphics[width=0.8\cmsFigWidth]{Figures/Unfolding/Systematics/ZZTo2e2m_PtJet1_MCgen_Mad_fr}     
   \includegraphics[width=0.8\cmsFigWidth]{Figures/Unfolding/Systematics/ZZTo2e2m_PtJet1_IrrBkg_Mad_fr}
   \includegraphics[width=0.8\cmsFigWidth]{Figures/Unfolding/Systematics/ZZTo2e2m_PtJet1_RedBkg_Mad_fr}     
   \includegraphics[width=0.8\cmsFigWidth]{Figures/Unfolding/Systematics/ZZTo2e2m_PtJet1_UnfDataOverGenMC_Mad_fr}     
   \includegraphics[width=0.8\cmsFigWidth]{Figures/Unfolding/Systematics/ZZTo2e2m_PtJet1_SFSq_Mad_fr}
   \includegraphics[width=0.8\cmsFigWidth]{Figures/Unfolding/Systematics/ZZTo2e2m_PtJet1_JER_Mad_fr}
   \includegraphics[width=0.8\cmsFigWidth]{Figures/Unfolding/Systematics/ZZTo2e2m_PtJet1_JES_ModData_Mad_fr}     
   \includegraphics[width=0.8\cmsFigWidth]{Figures/Unfolding/Systematics/ZZTo2e2m_PtJet1_JES_ModMat_Mad_fr}
   \caption{Effects of the different sources of systematic uncertainty on the unfolded distributions of $p_{T}^{jet1}$, for the     
   $2e2\mu$ final state. From left to right, from top to bottom: $\sigma_{qq}$ and $\sigma_{gg}$ ratio, MC generator, irreducible background, reducible background, unfolded/truth ratio, lepton efficiency, JER, JES modifying data distribution, JES modifying the response matrix. The systematic effect is superimposed on the nominal unfolded distribution, togheter with MC predictions from \texttt{MadGraph} and \texttt{Pohweg} sets of samples.}
   \label{fig:PtJet1syst2e2m}
 \end{center}
\end{figure}

\begin{figure}[hbtp]
 \begin{center}
   \includegraphics[width=0.8\cmsFigWidth]{Figures/Unfolding/Systematics/ZZTo4e_PtJet2_qqgg_Mad_fr}     
   \includegraphics[width=0.8\cmsFigWidth]{Figures/Unfolding/Systematics/ZZTo4e_PtJet2_MCgen_Mad_fr}     
   \includegraphics[width=0.8\cmsFigWidth]{Figures/Unfolding/Systematics/ZZTo4e_PtJet2_IrrBkg_Mad_fr}
   \includegraphics[width=0.8\cmsFigWidth]{Figures/Unfolding/Systematics/ZZTo4e_PtJet2_RedBkg_Mad_fr}     
  \includegraphics[width=0.8\cmsFigWidth]{Figures/Unfolding/Systematics/ZZTo4e_PtJet2_UnfDataOverGenMC_Mad_fr}     
   \includegraphics[width=0.8\cmsFigWidth]{Figures/Unfolding/Systematics/ZZTo4e_PtJet2_SFSq_Mad_fr}
   \includegraphics[width=0.8\cmsFigWidth]{Figures/Unfolding/Systematics/ZZTo4e_PtJet2_JER_Mad_fr}
   \includegraphics[width=0.8\cmsFigWidth]{Figures/Unfolding/Systematics/ZZTo4e_PtJet2_JES_ModData_Mad_fr}     
   \includegraphics[width=0.8\cmsFigWidth]{Figures/Unfolding/Systematics/ZZTo4e_PtJet2_JES_ModMat_Mad_fr}
   \caption{Effects of the different sources of systematic uncertainty on the unfolded distributions of  $p_{T}^{jet2}$, for the     
   $4e$ final state. From left to right, from top to bottom: $\sigma_{qq}$ and $\sigma_{gg}$ ratio, MC generator, irreducible background, reducible background, unfolded/truth ratio, lepton efficiency, JER, JES modifying data distribution, JES modifying the response matrix. The systematic effect is superimposed on the nominal unfolded distribution, togheter with MC predictions from \texttt{MadGraph} and \texttt{Pohweg} sets of samples.}
   \label{fig:PtJet2_syst_4e}
 \end{center}
\end{figure}

\begin{figure}[hbtp]
 \begin{center}
   \includegraphics[width=0.8\cmsFigWidth]{Figures/Unfolding/Systematics/ZZTo4m_PtJet2_qqgg_Mad_fr}     
   \includegraphics[width=0.8\cmsFigWidth]{Figures/Unfolding/Systematics/ZZTo4m_PtJet2_MCgen_Mad_fr}     
   \includegraphics[width=0.8\cmsFigWidth]{Figures/Unfolding/Systematics/ZZTo4m_PtJet2_IrrBkg_Mad_fr}
   \includegraphics[width=0.8\cmsFigWidth]{Figures/Unfolding/Systematics/ZZTo4m_PtJet2_RedBkg_Mad_fr}     
  \includegraphics[width=0.8\cmsFigWidth]{Figures/Unfolding/Systematics/ZZTo4m_PtJet2_UnfDataOverGenMC_Mad_fr}     
   \includegraphics[width=0.8\cmsFigWidth]{Figures/Unfolding/Systematics/ZZTo4m_PtJet2_SFSq_Mad_fr}
   \includegraphics[width=0.8\cmsFigWidth]{Figures/Unfolding/Systematics/ZZTo4m_PtJet2_JER_Mad_fr}
   \includegraphics[width=0.8\cmsFigWidth]{Figures/Unfolding/Systematics/ZZTo4m_PtJet2_JES_ModData_Mad_fr}     
   \includegraphics[width=0.8\cmsFigWidth]{Figures/Unfolding/Systematics/ZZTo4m_PtJet2_JES_ModMat_Mad_fr}
   \caption{Effects of the different sources of systematic uncertainty on the unfolded distributions of  $p_{T}^{jet2}$, for the     
   $4\mu$ final state. From left to right, from top to bottom: $\sigma_{qq}$ and $\sigma_{gg}$ ratio, MC generator, irreducible background, reducible background, unfolded/truth ratio, lepton efficiency, JER, JES modifying data distribution, JES modifying the response matrix. The systematic effect is superimposed on the nominal unfolded distribution, togheter with MC predictions from \texttt{MadGraph} and \texttt{Pohweg} sets of samples.}
   \label{fig:PtJet2_syst_4m}
 \end{center}
\end{figure}

\begin{figure}[hbtp]
 \begin{center}
   \includegraphics[width=0.8\cmsFigWidth]{Figures/Unfolding/Systematics/ZZTo2e2m_PtJet2_qqgg_Mad_fr}     
   \includegraphics[width=0.8\cmsFigWidth]{Figures/Unfolding/Systematics/ZZTo2e2m_PtJet2_MCgen_Mad_fr}     
   \includegraphics[width=0.8\cmsFigWidth]{Figures/Unfolding/Systematics/ZZTo2e2m_PtJet2_IrrBkg_Mad_fr}
   \includegraphics[width=0.8\cmsFigWidth]{Figures/Unfolding/Systematics/ZZTo2e2m_PtJet2_RedBkg_Mad_fr}     
   \includegraphics[width=0.8\cmsFigWidth]{Figures/Unfolding/Systematics/ZZTo2e2m_PtJet2_UnfDataOverGenMC_Mad_fr}     
   \includegraphics[width=0.8\cmsFigWidth]{Figures/Unfolding/Systematics/ZZTo2e2m_PtJet2_SFSq_Mad_fr}
   \includegraphics[width=0.8\cmsFigWidth]{Figures/Unfolding/Systematics/ZZTo2e2m_PtJet2_JER_Mad_fr}
   \includegraphics[width=0.8\cmsFigWidth]{Figures/Unfolding/Systematics/ZZTo2e2m_PtJet2_JES_ModData_Mad_fr}     
   \includegraphics[width=0.8\cmsFigWidth]{Figures/Unfolding/Systematics/ZZTo2e2m_PtJet2_JES_ModMat_Mad_fr}
   \caption{Effects of the different sources of systematic uncertainty on the unfolded distributions of $p_{T}^{jet2}$, for the     
   $2e2\mu$ final state. From left to right, from top to bottom: $\sigma_{qq}$ and $\sigma_{gg}$ ratio, MC generator, irreducible background, reducible background, unfolded/truth ratio, lepton efficiency, JER, JES modifying data distribution, JES modifying the response matrix. The systematic effect is superimposed on the nominal unfolded distribution, togheter with MC predictions from \texttt{MadGraph} and \texttt{Pohweg} sets of samples.}
   \label{fig:PtJet2syst2e2m}
 \end{center}
\end{figure}
\begin{figure}[hbtp]
 \begin{center}
   \includegraphics[width=0.8\cmsFigWidth]{Figures/Unfolding/Systematics/ZZTo4e_EtaJet1_qqgg_Mad_fr}     
   \includegraphics[width=0.8\cmsFigWidth]{Figures/Unfolding/Systematics/ZZTo4e_EtaJet1_MCgen_Mad_fr}     
   \includegraphics[width=0.8\cmsFigWidth]{Figures/Unfolding/Systematics/ZZTo4e_EtaJet1_IrrBkg_Mad_fr}
   \includegraphics[width=0.8\cmsFigWidth]{Figures/Unfolding/Systematics/ZZTo4e_EtaJet1_RedBkg_Mad_fr}     
  \includegraphics[width=0.8\cmsFigWidth]{Figures/Unfolding/Systematics/ZZTo4e_EtaJet1_UnfDataOverGenMC_Mad_fr}     
   \includegraphics[width=0.8\cmsFigWidth]{Figures/Unfolding/Systematics/ZZTo4e_EtaJet1_SFSq_Mad_fr}
   \includegraphics[width=0.8\cmsFigWidth]{Figures/Unfolding/Systematics/ZZTo4e_EtaJet1_JER_Mad_fr}
   \includegraphics[width=0.8\cmsFigWidth]{Figures/Unfolding/Systematics/ZZTo4e_EtaJet1_JES_ModData_Mad_fr}     
   \includegraphics[width=0.8\cmsFigWidth]{Figures/Unfolding/Systematics/ZZTo4e_EtaJet1_JES_ModMat_Mad_fr}
   \caption{Effects of the different sources of systematic uncertainty on the unfolded distributions of  $\eta^{jet1}$, for the     
   $4e$ final state. From left to right, from top to bottom: $\sigma_{qq}$ and $\sigma_{gg}$ ratio, MC generator, irreducible background, reducible background, unfolded/truth ratio, lepton efficiency, JER, JES modifying data distribution, JES modifying the response matrix. The systematic effect is superimposed on the nominal unfolded distribution, togheter with MC predictions from \texttt{MadGraph} and \texttt{Pohweg} sets of samples.}
   \label{fig:EtaJet1_syst_4e}
 \end{center}
\end{figure}
\clearpage
\begin{figure}[hbtp]
 \begin{center}
   \includegraphics[width=0.8\cmsFigWidth]{Figures/Unfolding/Systematics/ZZTo4m_EtaJet1_qqgg_Mad_fr}     
   \includegraphics[width=0.8\cmsFigWidth]{Figures/Unfolding/Systematics/ZZTo4m_EtaJet1_MCgen_Mad_fr}     
   \includegraphics[width=0.8\cmsFigWidth]{Figures/Unfolding/Systematics/ZZTo4m_EtaJet1_IrrBkg_Mad_fr}
   \includegraphics[width=0.8\cmsFigWidth]{Figures/Unfolding/Systematics/ZZTo4m_EtaJet1_RedBkg_Mad_fr}     
  \includegraphics[width=0.8\cmsFigWidth]{Figures/Unfolding/Systematics/ZZTo4m_EtaJet1_UnfDataOverGenMC_Mad_fr}     
   \includegraphics[width=0.8\cmsFigWidth]{Figures/Unfolding/Systematics/ZZTo4m_EtaJet1_SFSq_Mad_fr}
   \includegraphics[width=0.8\cmsFigWidth]{Figures/Unfolding/Systematics/ZZTo4m_EtaJet1_JER_Mad_fr}
   \includegraphics[width=0.8\cmsFigWidth]{Figures/Unfolding/Systematics/ZZTo4m_EtaJet1_JES_ModData_Mad_fr}     
   \includegraphics[width=0.8\cmsFigWidth]{Figures/Unfolding/Systematics/ZZTo4m_EtaJet1_JES_ModMat_Mad_fr}
   \caption{Effects of the different sources of systematic uncertainty on the unfolded distributions of  $\eta^{jet1}$, for the     
   $4\mu$ final state. From left to right, from top to bottom: $\sigma_{qq}$ and $\sigma_{gg}$ ratio, MC generator, irreducible background, reducible background, unfolded/truth ratio, lepton efficiency, JER, JES modifying data distribution, JES modifying the response matrix. The systematic effect is superimposed on the nominal unfolded distribution, togheter with MC predictions from \texttt{MadGraph} and \texttt{Pohweg} sets of samples.}
   \label{fig:EtaJet1_syst_4m}
 \end{center}
\end{figure}

\begin{figure}[hbtp]
 \begin{center}
   \includegraphics[width=0.8\cmsFigWidth]{Figures/Unfolding/Systematics/ZZTo2e2m_EtaJet1_qqgg_Mad_fr}     
   \includegraphics[width=0.8\cmsFigWidth]{Figures/Unfolding/Systematics/ZZTo2e2m_EtaJet1_MCgen_Mad_fr}     
   \includegraphics[width=0.8\cmsFigWidth]{Figures/Unfolding/Systematics/ZZTo2e2m_EtaJet1_IrrBkg_Mad_fr}
   \includegraphics[width=0.8\cmsFigWidth]{Figures/Unfolding/Systematics/ZZTo2e2m_EtaJet1_RedBkg_Mad_fr}     
   \includegraphics[width=0.8\cmsFigWidth]{Figures/Unfolding/Systematics/ZZTo2e2m_EtaJet1_UnfDataOverGenMC_Mad_fr}     
   \includegraphics[width=0.8\cmsFigWidth]{Figures/Unfolding/Systematics/ZZTo2e2m_EtaJet1_SFSq_Mad_fr}
   \includegraphics[width=0.8\cmsFigWidth]{Figures/Unfolding/Systematics/ZZTo2e2m_EtaJet1_JER_Mad_fr}
   \includegraphics[width=0.8\cmsFigWidth]{Figures/Unfolding/Systematics/ZZTo2e2m_EtaJet1_JES_ModData_Mad_fr}     
   \includegraphics[width=0.8\cmsFigWidth]{Figures/Unfolding/Systematics/ZZTo2e2m_EtaJet1_JES_ModMat_Mad_fr}
   \caption{Effects of the different sources of systematic uncertainty on the unfolded distributions of $\eta^{jet1}$, for the     
   $2e2\mu$ final state. From left to right, from top to bottom: $\sigma_{qq}$ and $\sigma_{gg}$ ratio, MC generator, irreducible background, reducible background, unfolded/truth ratio, lepton efficiency, JER, JES modifying data distribution, JES modifying the response matrix. The systematic effect is superimposed on the nominal unfolded distribution, togheter with MC predictions from \texttt{MadGraph} and \texttt{Pohweg} sets of samples.}
   \label{fig:EtaJet1syst2e2m}
 \end{center}
\end{figure}

\begin{figure}[hbtp]
 \begin{center}
   \includegraphics[width=0.8\cmsFigWidth]{Figures/Unfolding/Systematics/ZZTo4e_EtaJet2_qqgg_Mad_fr}     
   \includegraphics[width=0.8\cmsFigWidth]{Figures/Unfolding/Systematics/ZZTo4e_EtaJet2_MCgen_Mad_fr}     
   \includegraphics[width=0.8\cmsFigWidth]{Figures/Unfolding/Systematics/ZZTo4e_EtaJet2_IrrBkg_Mad_fr}
   \includegraphics[width=0.8\cmsFigWidth]{Figures/Unfolding/Systematics/ZZTo4e_EtaJet2_RedBkg_Mad_fr}     
  \includegraphics[width=0.8\cmsFigWidth]{Figures/Unfolding/Systematics/ZZTo4e_EtaJet2_UnfDataOverGenMC_Mad_fr}     
   \includegraphics[width=0.8\cmsFigWidth]{Figures/Unfolding/Systematics/ZZTo4e_EtaJet2_SFSq_Mad_fr}
   \includegraphics[width=0.8\cmsFigWidth]{Figures/Unfolding/Systematics/ZZTo4e_EtaJet2_JER_Mad_fr}
   \includegraphics[width=0.8\cmsFigWidth]{Figures/Unfolding/Systematics/ZZTo4e_EtaJet2_JES_ModData_Mad_fr}     
   \includegraphics[width=0.8\cmsFigWidth]{Figures/Unfolding/Systematics/ZZTo4e_EtaJet2_JES_ModMat_Mad_fr}
   \caption{Effects of the different sources of systematic uncertainty on the unfolded distributions of  $\eta^{jet2}$, for the     
   $4e$ final state. From left to right, from top to bottom: $\sigma_{qq}$ and $\sigma_{gg}$ ratio, MC generator, irreducible background, reducible background, unfolded/truth ratio, lepton efficiency, JER, JES modifying data distribution, JES modifying the response matrix. The systematic effect is superimposed on the nominal unfolded distribution, togheter with MC predictions from \texttt{MadGraph} and \texttt{Pohweg} sets of samples.}
   \label{fig:EtaJet2_syst_4e}
 \end{center}
\end{figure}

\begin{figure}[hbtp]
 \begin{center}
   \includegraphics[width=0.8\cmsFigWidth]{Figures/Unfolding/Systematics/ZZTo4m_EtaJet2_qqgg_Mad_fr}     
   \includegraphics[width=0.8\cmsFigWidth]{Figures/Unfolding/Systematics/ZZTo4m_EtaJet2_MCgen_Mad_fr}     
   \includegraphics[width=0.8\cmsFigWidth]{Figures/Unfolding/Systematics/ZZTo4m_EtaJet2_IrrBkg_Mad_fr}
   \includegraphics[width=0.8\cmsFigWidth]{Figures/Unfolding/Systematics/ZZTo4m_EtaJet2_RedBkg_Mad_fr}     
   \includegraphics[width=0.8\cmsFigWidth]{Figures/Unfolding/Systematics/ZZTo4m_EtaJet2_UnfDataOverGenMC_Mad_fr}     
   \includegraphics[width=0.8\cmsFigWidth]{Figures/Unfolding/Systematics/ZZTo4m_EtaJet2_SFSq_Mad_fr}
   \includegraphics[width=0.8\cmsFigWidth]{Figures/Unfolding/Systematics/ZZTo4m_EtaJet2_JER_Mad_fr}
   \includegraphics[width=0.8\cmsFigWidth]{Figures/Unfolding/Systematics/ZZTo4m_EtaJet2_JES_ModData_Mad_fr}     
   \includegraphics[width=0.8\cmsFigWidth]{Figures/Unfolding/Systematics/ZZTo4m_EtaJet2_JES_ModMat_Mad_fr}
   \caption{Effects of the different sources of systematic uncertainty on the unfolded distributions of  $\eta^{jet2}$, for the     
   $4\mu$ final state. From left to right, from top to bottom: $\sigma_{qq}$ and $\sigma_{gg}$ ratio, MC generator, irreducible background, reducible background, unfolded/truth ratio, lepton efficiency, JER, JES modifying data distribution, JES modifying the response matrix. The systematic effect is superimposed on the nominal unfolded distribution, togheter with MC predictions from \texttt{MadGraph} and \texttt{Pohweg} sets of samples.}
   \label{fig:EtaJet2_syst_4m}
 \end{center}
\end{figure}

\begin{figure}[hbtp]
 \begin{center}
   \includegraphics[width=0.8\cmsFigWidth]{Figures/Unfolding/Systematics/ZZTo2e2m_EtaJet2_qqgg_Mad_fr}     
   \includegraphics[width=0.8\cmsFigWidth]{Figures/Unfolding/Systematics/ZZTo2e2m_EtaJet2_MCgen_Mad_fr}     
   \includegraphics[width=0.8\cmsFigWidth]{Figures/Unfolding/Systematics/ZZTo2e2m_EtaJet2_IrrBkg_Mad_fr}
   \includegraphics[width=0.8\cmsFigWidth]{Figures/Unfolding/Systematics/ZZTo2e2m_EtaJet2_RedBkg_Mad_fr}     
   \includegraphics[width=0.8\cmsFigWidth]{Figures/Unfolding/Systematics/ZZTo2e2m_EtaJet2_UnfDataOverGenMC_Mad_fr}     
   \includegraphics[width=0.8\cmsFigWidth]{Figures/Unfolding/Systematics/ZZTo2e2m_EtaJet2_SFSq_Mad_fr}
   \includegraphics[width=0.8\cmsFigWidth]{Figures/Unfolding/Systematics/ZZTo2e2m_EtaJet2_JER_Mad_fr}
   \includegraphics[width=0.8\cmsFigWidth]{Figures/Unfolding/Systematics/ZZTo2e2m_EtaJet2_JES_ModData_Mad_fr}     
   \includegraphics[width=0.8\cmsFigWidth]{Figures/Unfolding/Systematics/ZZTo2e2m_EtaJet2_JES_ModMat_Mad_fr}
   \caption{Effects of the different sources of systematic uncertainty on the unfolded distributions of $\eta^{jet2}$, for the     
   $2e2\mu$ final state. From left to right, from top to bottom: $\sigma_{qq}$ and $\sigma_{gg}$ ratio, MC generator, irreducible background, reducible background, unfolded/truth ratio, lepton efficiency, JER, JES modifying data distribution, JES modifying the response matrix. The systematic effect is superimposed on the nominal unfolded distribution, togheter with MC predictions from \texttt{MadGraph} and \texttt{Pohweg} sets of samples.}
   \label{fig:EtaJet2syst2e2m}
 \end{center}
\end{figure}


% >> acknowledgements (for journal papers)
% Please include the latest version from https://twiki.cern.ch/twiki/bin/viewauth/CMS/Internal/PubAcknow.
%\section*{Acknowledgements}
% ack-text

%% **DO NOT REMOVE BIBLIOGRAPHY**
\bibliography{auto_generated}   % will be created by the tdr script.

%% examples of appendices. **DO NOT PUT \end{document} at the end
%\clearpage

%%% DO NOT ADD \end{document}!

