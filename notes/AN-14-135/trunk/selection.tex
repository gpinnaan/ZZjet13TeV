%The identification and reconstruction of the leptons produced in the event is the same as the one described in~\cite{ZZXSPaper}, including the final state radiation (FSR) treatment, which corrects the energy and isolation variables of the leptons from Z boson candidates when the radiated photons are
%identified. In addition, jets are reconstructed using the anti-$k_{T}$ clustering algorithm~\cite{antiKT}, as implemented in the \texttt{FASTJET} package~\cite{FASTJET}, with a distance parameter of 0.5. In order to assure a good reconstruction efficiency, identification quality criteria are imposed on jets based on the energy fraction of the charged, electromagnetic, and neutral hadronic components.\\
%Once reconstructed, jets are required to have a spatial separation from lepton candidates of $\Delta R = \sqrt{(\Delta\eta)^2+(\Delta\phi)^2}> 0.5$. Several effects contribute to bias the measured jet energy, such as pileup interactions, detector noise, and detector response non-uniformities in $\eta$ and non-linearities in $p_T$. The jet energy scale (JES) calibration~\cite{JESCalib, Khachatryan:2016kdb} relies on corrections which are parameterized in terms of the uncorrected $p_T$ and $\eta$ of the jet, and applied as multiplicative factors scaling the four-momentum vector of each jet. Furthermore, the jet energy resolution (JER) in data is known to be worse than in the simulation; therefore the simulated resolution is degraded to compensate for this effect. 
%A minimum threshold of 30 GeV on the transverse
%momentum of jets is required to ensure that they are well measured and
%to reduce the pileup contamination. Only jets with $|\eta| < $ 4.7 are considered.\\
%The selection is the same as~\cite{ZZXSPaper} and is here summarized briefly for convenience. Events are selected online from the presence of a pair of electrons or muons, or a triplet of electrons. Triggers requiring an electron together with a muon are also used.
%Electrons are selected if they have $|\eta^e| < 2.5$ and transverse momentum greater than 7 GeV,  while muons are considered if they have $|\eta^{\mu}| < 2.4$ and $p^{\mu}_T > 5~\mathrm{GeV}$.
%Moreover, electrons and muons must be isolated from the other particles and are required to originate from the primary vertex of the collision.  Leptons coming from tau decays are not considered as signal leptons. Each of the two Z candidates is formed of a pair of leptons of the same flavor and opposite charge that satisfies $60 < m_{\ell\ell} < 120$ GeV.

%\begin{table*}[htbH]
%\begin{center}
%\topcaption{Scaling factors applied on MC jets to described the data resolution\label{tab:c_JER}}
%\begin{tabular}{cccccccc}
%\hline $|\eta|$-region & $0.0-0.5$ & $0.5-1.1$ & $1.1-1.7$ & $1.7-2.3$ & $2.3-2.8$ & $2.8-3.2$ & $3.2-5.0$\\
%\hline $c$ & 1.079 & 1.099 & 1.121 & 1.208 & 1.254 & 1.395 & 1.056\\
%\hline
%\end{tabular}
%\end{center}
%\end{table*}

