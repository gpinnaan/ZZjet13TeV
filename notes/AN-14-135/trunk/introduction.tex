In this note we present the measurement of the differential cross section of two Z bosons produced in association with jets in pp collisions at $\sqrt{s}= 13~\mathrm{TeV}$, extending the analysis performed at $\sqrt{s}= 8~\mathrm{TeV}$ in~\cite{CMS-PAS-SMP-15-012}. The dependence of the cross section on jet multiplicity tests the current understanding of the QCD corrections to ZZ production. Moreover, the study of the invariant mass of the two $p_T$-leading jets ($m_{jj}$) establishes the basis for future multi-boson final state searches and for the investigation of phenomena involving interactions with four bosons in a single vertex. Finally, the measurement of the separation in pseudorapidity between the two $p_T$-leading jets ($\Delta\eta_{jj}$) is instrumental for the vector boson scattering analysis in this channel, the key process to probe the nature of the electroweak symmetry breaking mechanism, which becomes potentially accessible at 13 TeV.\\
 The measurement is performed considering the \Z bosons decaying leptonically to either electrons or muons. The analysis is based on a data set that corresponds to an integrated luminosity of $12.9~\mathrm{fb}^{-1}$, collected by the CMS experiment at $\sqrt{s} =$~ 13 TeV.\\

The implant of the analysis is largely the same as~\cite{CMS-PAS-SMP-15-012} and, for what concerns the reconstruction of the two Z bosons, it is explicitly
based on the algorithms and the techniques used for the $H\rightarrow ZZ \rightarrow 4\ell$ analysis~\cite{HiggsLegacyPaper} (as it was done for~\cite{CMS-PAS-SMP-15-012}), with the obvious change in the mass window ($60 < m_{Z_{1,2}} < 120~\mathrm{GeV}$). %used for the lowest quality Z.


This note is organized as follow: section~\ref{sec:samples} summarizes the Monte Carlo and the dataset used for this analysis, section~\ref{sec:selection} describes the event selection, 
while section~\ref{sec:background} deals with the background estimation and the related uncertainties. Section~\ref{sec:systunc} lists the several sources of systematic uncertainty and, finally, in section~\ref{sec:results} we present the result of this analysis,
including the unfolding of the detector resolution for the differential cross-section that we measured.
 
